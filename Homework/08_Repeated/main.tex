\documentclass[addpoints]{exam}

\printanswers
\CorrectChoiceEmphasis{\color{red}\bfseries}
\usepackage{amssymb, amsmath, amsfonts}
\usepackage{geometry}
\usepackage{graphicx}
\usepackage{tikz}
\usetikzlibrary{calc}
\usepackage{pgfplots}
\usepackage{multirow,array} % for payoff matrix formatting
\usepackage[colorlinks,pdfusetitle,urlcolor=blue,citecolor=blue,linkcolor=blue]{hyperref}
\usepackage{bbding} % dingbats fonts

\usepackage{adjustbox}

\definecolor{crimson}{RGB}{ 170, 4, 36 }
\definecolor{darkblue}{RGB}{ 4, 47, 170 }
\definecolor{brown}{RGB}{ 111, 71, 2 }
\definecolor{periwinkle}{RGB}{ 90, 177, 204 }
\definecolor{ducksgreen}{HTML}{007030}

\geometry{left=1.0in,right=1.0in,top=1.0in,bottom=1.0in}
\pagestyle{headandfoot}
\lhead{EC327 Game Theory}
\chead{Homework 5}
\rhead{Fall 2025}
\runningheadrule

\title{
    \textbf{Econ 327: Game Theory} \\ 
    Homework $\#8$
    }
\author{University of Oregon}
\date{Due: Dec. 5$^{th}$}

% exam-type question formatting
\renewcommand{\thequestion}{\textbf{Q\arabic{question}}}
\bracketedpoints

\begin{document}

\maketitle

\begin{center}
  \gradetable[h][questions]
\end{center}

\vspace{0.5in}

\begin{center}
  \textbf{For homework assignments:}
\end{center}

\begin{itemize}

%  \item DO NOT write your name:
%  this assignment will be graded anonymously. 
%  If you want to, you can include your student ID instead.

  \item Complete \textit{all} questions and parts.

  % I will select one question at random to be graded
  % according to the rubric on Canvas.

  \item You will be graded on not only the content of your work
    but on how clearly you present your ideas.
    Make sure that your handwriting is legible.
    Please use extra pages if you run out of space 
    but make sure that all parts of a question 
    are in the correct order when you submit.

  \item You may choose to work with others,
  but everyone must submit to Canvas individually.

  Please include the names of everyone who you worked with 
  below your own name.
 
\end{itemize}

\vspace{1.0in}

\makebox[.6\textwidth]{Name\enspace\hrulefill}


\vspace{0.5in}

\begin{center}
  \fbox{\fbox{\parbox{5.5in}{\centering
  \textbf{Note:}
  All Questions are adapted from problems in  
  Dixit, Skeath and Reiley, \textit{Games of Strategy}, Fourth Edition. 
  }}}
\end{center}

\newpage

\begin{questions}
  \question
  Recall the example from the midterm, 
  where South Korea and Japan compete in the market for production of VLCCs. 
  Assume now that the cost of building ships is \$30 (million) in each country, 
  and the demand for ships is $P = 180 - Q$,
  where $Q = q_{\text{Korea}} + q_{\text{Japan}}$.
  \begin{parts}
    \part [4]
    Solve for all Nash equilibria in this continuous strategy game.
    \part [4] 
    Now find the collusive outcome.
    What total quantity should be set by the two countries 
    in order to maximize their joint profit?
    \part [4]
    Suppose the two countries produce equal quantities of VLCCs, so
    that they earn equal shares of this collusive profit.
    How much profit would each country earn? 
    Compare this profit with the amount they would earn in the Nash equilibrium.
    \part [4] 
    Now suppose the two countries are in a repeated relationship.
    Once per year, they choose production quantities,
    and each can observe the amount its rival produced in the previous year. 
    They wish to cooperate to sustain the collusive profit levels found in part (c). 
    In any one year, one of them can defect from the agreement.
    If one of them holds the quantity at the agreed level, 
    what is the best defecting quantity for the other? 
    What are the resulting profits?
    \part[4]
    Write down a matrix that represents this game as a prisoners’ dilemma.
    \part[4] 
    For what interest rates will collusion be sustainable 
    when the two countries use grim (defect forever) strategies?
  \end{parts}

  \newpage 

  \question
  Glassworks and Clearsmooth compete in the local market for windshield repairs.
  The market size (total available profits) is \$10 million per year.
  Each firm can choose whether to advertise on local television. 
  If a firm chooses to advertise in a given year, 
  it costs that firm \$3 million.
  If one firm advertises and the other doesn’t, 
  then the former captures the whole market.
  If both firms advertise, they split the market 50:50.
  If both firms choose not to advertise, they also split the market 50:50.

  \begin{parts}
    \part[4]
    Suppose the two windshield-repair firms know they will compete for just one year.
    Write down the payoff matrix for this game.
    Find the Nash equilibrium strategies.
    \part[4]
    Suppose the firms play this game for five years in a row, 
    and they know that at the end of five years,
    both firms plan to go out of business.
    What is the subgame-perfect equilibrium for this five-period game? 
    Explain. 
    \part[4]
    What would be a tit-for-tat strategy in the game described in part (b)?
    \part[4]
    Suppose the firms play this game repeatedly forever, 
    and suppose that future profits are discounted
    with an interest rate of 20\% per year.
    Can you find a subgame-perfect equilibrium that involves higher annual payoffs than the equilibrium in part (b)?
    If so, explain what strategies are involved.
    If not, explain why not.
  \end{parts}
  
\end{questions}
\end{document}