\documentclass[addpoints]{exam}

% \printanswers
\CorrectChoiceEmphasis{\color{red}\bfseries}
\usepackage{amssymb, amsmath, amsfonts}
\usepackage{geometry}
\usepackage{graphicx}
\usepackage{tikz}
\usetikzlibrary{calc}
\usepackage{multirow,array} % for payoff matrix formatting
\usepackage[colorlinks,pdfusetitle,urlcolor=blue,citecolor=blue,linkcolor=blue]{hyperref}

\definecolor{crimson}{RGB}{ 170, 4, 36 }
\definecolor{darkblue}{RGB}{ 4, 47, 170 }
\definecolor{brown}{RGB}{ 111, 71, 2 }
\definecolor{periwinkle}{RGB}{ 90, 177, 204 }
\definecolor{ducksgreen}{HTML}{007030}

\geometry{left=1.0in,right=1.0in,top=1.0in,bottom=1.0in}
\pagestyle{headandfoot}
\lhead{EC327 Game Theory}
\chead{Homework 3}
\rhead{Fall 2025}
\runningheadrule

\title{
    \textbf{Econ 327: Game Theory} \\ 
    Homework $\#3$
    }
\author{University of Oregon}
\date{Due: Oct. 17$^{th}$}

% exam-type question formatting
\renewcommand{\thequestion}{\textbf{Q\arabic{question}}}
\bracketedpoints

\begin{document}

\maketitle

\begin{center}
  \gradetable[h][questions]
\end{center}

\vspace{0.5in}

\begin{center}
  \textbf{For homework assignments:}
\end{center}

\begin{itemize}

%  \item DO NOT write your name:
%  this assignment will be graded anonymously. 
%  If you want to, you can include your student ID instead.

  \item Complete \textit{all} questions and parts.

  % I will select one question at random to be graded
  % according to the rubric on Canvas.

  \item You will be graded on not only the content of your work
    but on how clearly you present your ideas.
    Make sure that your handwriting is legible.
    Please use extra pages if you run out of space 
    but make sure that all parts of a question 
    are in the correct order when you submit.

  \item You may choose to work with others,
  but everyone must submit to Canvas individually.

  Please include the names of everyone who you worked with 
  below your own name.
 
\end{itemize}

\vspace{1.0in}

\makebox[.6\textwidth]{Name\enspace\hrulefill}

\vspace{0.5in}

% \begin{center}
%   \fbox{\fbox{\parbox{5.5in}{\centering
%     Answer the questions in the spaces provided on the
%     question sheets. If you run out of room for an answer,
%     continue on the back of the page or another sheet of paper.}}}
% \end{center}

\begin{questions}

\newpage

\question

Consider the strategic form game below:

\begin{table}[!h]
  \begin{center}
    \begin{tabular}{*{6}{c|}}
      \multicolumn{2}{c}{} & \multicolumn{4}{c}{$P_2$} \\ \cline{3-6}
      \multicolumn{1}{c}{} &  & $A$ & $B$ & $C$ & $D$ \\ \cline{2-6} 
      \multirow{4}*{$P_1$}
      & $H$ & 8,  1 & -3,  1 &  0,  1 &  3,  1 \\ \cline{2-6}
      & $J$ & 10, -2 &  0,  6 &  1, -1 &  4,  0 \\ \cline{2-6} 
      & $K$ & 9,  1 &  6,  3 &  2,  2 & 7, 4 \\ \cline{2-6} 
      & $L$ & 11, 10 & -1, 16 &  4, 12 &  5, 5 \\ \cline{2-6} 
  \end{tabular}
  \end{center}
\end{table}

\begin{parts}

  \part[2]
  Which strategy would Player 1 \textit{never} play if they are rational?
  \begin{solution}
    Player 1's strategy of $H$ always gives them a lower payoff compared to the payoffs of $J$.
    So if they are rational, they would never play $H$.
  \end{solution}
 
  \part[4] 
  Use Iterated Deletion of Strictly Dominated Strategies
  and write out a simplified game table with any remaining cells.

  \begin{solution}
    
    \begin{itemize}
      \item Step 1: $H$ is strictly dominated by $J$, eliminate $H$
      \item Step 2: $A$ and $C$ are both strictly dominated by $B$, eliminate $A$ and $C$
      \item Step 3: $J$ and $L$ are both strictly dominated by $K$, eliminate $J$ and $L$
      \item Step 4: $B$ is now strictly dominated by $D$, eliminate $B$.
    \end{itemize}

    \begin{center}
      \begin{tabular}{*{3}{c|}}
        \multicolumn{2}{c}{} & $P_2$ \\ \cline{3-3}
        \multicolumn{1}{c}{} & & $D$ \\ \cline{2-3}
        $P_1$
        & $K$ & 7, 4 \\ \cline{2-3}
      \end{tabular}
    \end{center}
  \end{solution}


  \part[4]
  Find all Nash equilibria in \textit{pure strategies}.

  \begin{solution}
      
    $\mathbf{(K,D)}$ is the only pure strategy Nash equilibrium.

    \par\noindent\rule{\textwidth}{0.4pt}

    Partial credit may be awarded for an answer that is consistent with mistakes made in 
    eliminating strictly dominated strategies in part $a$.

  \end{solution}
  
  \part[4]
  Explain why you know that the strategy profile(s) you found in part b are Nash equilibria.

  \begin{solution}
  
    This is the only pure strategy NE because we eliminated all strategies in part (a)
    that will never be played in any NE.
    We also found the intersection of either players best responses in the table from (a)
    which is the definition of a NE.

  \end{solution}

  \part[2]
  Is the outcome of the Nash equilibrium of this game Pareto optimal?
  Why or why not?

  \begin{solution}
    The Nash equilirium results in a payoff of 7 for Player 1 and a payoff of 4 for Player 2 which is not Pareto optimal.
    It is possible for both players to earn higher payoffs of (11,10) by playing the strategy profile $\mathbf{(L,A)}$
    which Pareto dominates the (7,1) payoffs of the Nash outcome.
  \end{solution}
\end{parts}

\newpage

\question
Here's a little ditty, about Jack and Diane,
two American kids growing up in the heartland.
\footnote{Cliff Bekar, Lewis and Clark College}

Suppose that Jack and Diane move \textit{simutaneously}.

  \begin{table}[h!]
    \centering
    \setlength{\extrarowheight}{2pt}
    \begin{tabular}{*{5}{c|}}
      \multicolumn{2}{c}{} & \multicolumn{3}{c}{Diane} \\\cline{3-5}
      \multicolumn{1}{c}{} &     & $x$ & $y$ & $z$ \\\cline{2-5}
      \multirow{3}*{Jack}  & $a$ & 1,1 & 2,1 & 2,0 \\\cline{2-5}
                           & $b$ & 2,3 & 0,2 & 2,1 \\\cline{2-5}
                           & $c$ & 2,1 & 1,2 & 3,0 \\\cline{2-5}
    \end{tabular}
  \end{table}

\begin{parts}

  \part[4] 
  Describe all of Jack's best responses.
  
  \begin{solution}
    For Jack, $b$ and $c$ are best responses to $x$,
    $a$ is BR to $y$,  
    and $c$ is BR to $z$.
  \end{solution}
    
  \part[4]
  Describe all of Diane's best responses.

  \begin{solution}
    For Diane, $x$ and $y$ are BR to $a$,
    $x$ is BR to $b$,
    and $z$ is BR to $c$.
  \end{solution}

  \part[4]
  List any strictly dominated strategies, or if there are none, explain why not.

  \begin{solution}
    There are no strictly dominated strategies for either player because every strategy is a best response to at least one of the opponents' strategies.
  \end{solution}

  \part[4]
  Find all strategy profiles which are Nash equilibria.
  Explain why each strategy profile is a Nash.

  \begin{solution}
    There are two strategy profiles where each player's best responses intersect:

    \begin{itemize}
      \item $N_1 = \left( b, x \right)$ results in payoffs (2,3)

      Jack's strategy is to choose $b$, 
      Diane's strategy is to choose $x$.
      Neither have regrets about their strategy choice;
      given that Jack is choosing $b$, Diane can't get a higher payoff by deviating.
      Given that Diane is playing $x$, Jack is indifferent between playing $b$ and $c$ but he can't get a strictly higher payoff by deviating. 
      The resulting outcome is that Jack gets $2$, 
      Diane gets $3$.

      \item $N_2 = \left( a, y \right)$ results in payoffs (2,1)

      Jack's strategy is $a$, Diane's is $y$. 
      When Jack plays $a$, Dianne is indifferent between $x$ and $y$, but
      still cannot deviate to a strictly higher payoff.
      When Diane plays $y$, Jack's best response is $a$ because $2>\{0,1\}$. 
      The outcome is that Jack gets $2$ and Diane gets $1$.

    \end{itemize}

    These are the only two \textit{pure strategy} Nash equilibria
    because there are no other intersections of pure strategy best responses.
    Note that even though $N_1$ Pareto dominates $N_2$ 
    (Jack is indifferent; $2=2$, and Diane is better off; $3>1$),
    there is no \textit{unilateral} deviation that would reach $N_1$
    from $N_2$.

  \end{solution}
\end{parts}

\newpage
\question[4]
Solve for all Nash equilibria in pure strategies.

\begin{table}[h!]
  \centering
  \setlength{\extrarowheight}{2pt}
  \begin{tabular}{*{4}{c|}}
    \multicolumn{2}{c}{} & \multicolumn{2}{c}{Column} \\\cline{3-4}
    \multicolumn{1}{c}{} &      & Left & Right \\\cline{2-4}
    \multirow{2}*{Row}   & Up   & 1,3  & 2,4   \\\cline{2-4}
                         & Down & 2,5  & 3,2   \\\cline{2-4}
  \end{tabular}
\end{table}

\begin{solution}
  \{Down, Left\}
\end{solution}

\question[4]
Solve for all Nash equilibria in pure strategies.

\begin{table}[h!]
  \centering
  \setlength{\extrarowheight}{2pt}
  \begin{tabular}{*{4}{c|}}
    \multicolumn{2}{c}{} & \multicolumn{2}{c}{Column} \\\cline{3-4}
    \multicolumn{1}{c}{} &      & Left & Right \\\cline{2-4}
    \multirow{2}*{Row}   & Up   & 5,3  & 1,1   \\\cline{2-4}
                         & Down & 0,0  & 2,2   \\\cline{2-4}
  \end{tabular}
\end{table}

\begin{solution}
  \{Up, Left\} and \{Down, Right\}
\end{solution}

\question[4]
Solve for all Nash equilibria in pure strategies.
\begin{table}[h!]
  \centering
  \setlength{\extrarowheight}{2pt}
  \begin{tabular}{*{5}{c|}}
    \multicolumn{2}{c}{} & \multicolumn{3}{c}{Column} \\\cline{3-5}
    \multicolumn{1}{c}{} &        & Left & Middle & Right \\\cline{2-5}
    \multirow{3}*{Row}   & Up     & 3,8 & 6,4 & 7,1 \\\cline{2-5}
                         & Center & 7,3 & 10,1 & 6,2 \\\cline{2-5}
                         & Down   & 5,2 & 3,6 & 8,0 \\\cline{2-5}
  \end{tabular}
\end{table}

\begin{solution}
  \{Center, Left\}
\end{solution}

\question[4]
Solve for all Nash equilibria in pure strategies.
\begin{table}[h!]
  \centering
  \setlength{\extrarowheight}{2pt}
  \begin{tabular}{*{5}{c|}}
    \multicolumn{2}{c}{} & \multicolumn{3}{c}{Column} \\\cline{3-5}
    \multicolumn{1}{c}{} &        & Left & Middle & Right \\\cline{2-5}
    \multirow{3}*{Row}   & Up     & 2,2 & 3,1 & 2,0 \\\cline{2-5}
                         & Center & 1,5 & 2,2 & 7,4 \\\cline{2-5}
                         & Down   & 0,1 & 4,0 & 6,2 \\\cline{2-5}
  \end{tabular}
\end{table}

\begin{solution}
  \{Up, Left\}
\end{solution}

\end{questions}

\end{document}
