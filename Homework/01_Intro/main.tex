\documentclass[addpoints]{exam}

% \printanswers
\CorrectChoiceEmphasis{\color{red}\bfseries}
\usepackage{amssymb, amsmath, amsfonts}
\usepackage{geometry}
\usepackage{graphicx}
\usepackage{tikz}
\usetikzlibrary{calc}
\usepackage{multirow,array} % for payoff matrix formatting
\usepackage[colorlinks,pdfusetitle,urlcolor=blue,citecolor=blue,linkcolor=blue]{hyperref}

\geometry{left=1.0in,right=1.0in,top=1.0in,bottom=1.0in}
\pagestyle{headandfoot}
\lhead{EC327 Game Theory}
\chead{Homework 1}
\rhead{Fall 2025}
\runningheadrule

\title{
    \textbf{Econ 327: Game Theory} \\ 
    Homework $\#1$
    }
\author{University of Oregon}
\date{Due: Oct. 3$^{rd}$}

% exam-type question formatting
\renewcommand{\thequestion}{\textbf{Question \arabic{question}}}
\bracketedpoints

\begin{document}

\maketitle

\begin{center}
  \gradetable[h][questions]
\end{center}

\vspace{0.5in}

\begin{center}
  \textbf{For homework assignments:}
\end{center}

\begin{itemize}

%  \item DO NOT write your name:
%  this assignment will be graded anonymously. 
%  If you want to, you can include your student ID instead.

  \item Complete \textit{all} questions and parts.

  % I will select one question at random to be graded
  % according to the rubric on Canvas.

  \item You will be graded on not only the content of your work
    but on how clearly you present your ideas.
    Make sure that your handwriting is legible.
    Please use extra pages if you run out of space 
    but make sure that all parts of a question 
    are in the correct order when you submit.

  \item You may choose to work with others,
  but everyone must submit to Canvas individually.

  Please include the names of everyone who you worked with 
  below your own name.
 
\end{itemize}

\vspace{1.0in}

\makebox[.6\textwidth]{Name\enspace\hrulefill}

\vspace{0.5in}

% \begin{center}
%   \fbox{\fbox{\parbox{5.5in}{\centering
%     Answer the questions in the spaces provided on the
%     question sheets. If you run out of room for an answer,
%     continue on the back of the page or another sheet of paper.}}}
% \end{center}

\begin{questions}

\newpage


%------------------------------------------------------------------

\question%[20]

Consider a (heavily) simplified version of the popular board game \textit{Catan}.
In this version, there are only five map tiles numbered one through five.
There are two players, Andy and Booth, and each has two settlements placed on a tile.
Andy's settlements are on tiles 1 and 2 and Booth's settlements are on tiles 3 and 4.
On a players' turn, they roll a single fairly weighted die and if it lands face up on a number which matches the tile of one of their settlements, they earn one sheep card.
However, if the die lands on a 6, then the player who rolled it can steal one sheep from their opponent.

Suppose that it is Andy's turn to roll the die.

\begin{parts}

\part[4] What is Andy's expected payoff (in sheep)?
  \begin{solution}
    Andy's expected payoff is $\frac{1}{3} (1) + \frac{1}{3} (0) + \frac{1}{6} (0) + \frac{1}{6} (1) = \frac{1}{2}$
  \end{solution}

\part[4] What is Booth's expected payoff?
  \begin{solution}
    Booth's expected payoff is $\frac{1}{3} (0) + \frac{1}{3} (1) + \frac{1}{6} (0) + \frac{1}{3} (-1) = \frac{1}{6} $

    \textbf{Note:} I realize I could have been more clear on the rules of Catan.
    If you said Booth's expected payoff was $-\frac{1}{6}$, I gave you credit.
  \end{solution}

\end{parts}

\question

Watch the clip from the British Game show Golden Balls.
\url{https://youtu.be/bOwh9_lAeLA}
The specific game is called Golden Balls and is played between the two contestents after they have collected a pool of prize money together in other challenges.

Respond to the following questions about the game and briefly justify each answer.

\begin{parts}
  \part[2] 
  Do the players take their actions \textit{sequentially} or \textit{simultaneously}?
  \begin{solution}
    Simultaneous
  \end{solution}

  \part[2]
  Is Golden Balls a \textit{zero-sum} or a \textit{constant-sum} game? 
  \begin{solution}
    Neither, it is possible for the sum of both players' payoffs to either be zero or \pounds10,000,
    depending on what strategies they choose.
  \end{solution}

  \part[2]
  What relevant \textit{information} is available to the players when taking their actions?
  \begin{solution}
    The rules of the game, payoffs, etc. but not the strategies their opponent will take
    (or how honest they are).
  \end{solution}

  \part[2]
  How important is the stage where the constestants are told to talk with each other?
  Can communication be strategic?
  \begin{solution}
    This is up to interpretation.
    You might say that because there is an incentive to lie and say you will split,
    there is no real point in listening to anything your opponent could say
    (`cheap talk').

    But maybe there are some ways in which you could \textit{credibly} signal your true intent.
    For example, making a legally binding pledge with enforcable penalties, etc.
  \end{solution}

\end{parts}

\newpage

\question%[20]

Read the 2022 Policito Opinion article, 
\textit{How Game Theory Explains Why We Have to Sanction Putin : Even If It’s Costly}.

\url{https://www.politico.com/news/magazine/2022/04/21/russia-sanctions-game-theory-00026566}

\begin{parts}

  \part[4]
  List the tools from game theory that we've learned about in class
  which the authors use to argue their point.
  \begin{solution}
    Relevant tools/terms which we have discussed include backwards induction,
    common knowledge of rationality, perfect information, Pareto dominance/efficiency,
    game tree/extensive form, forward-looking, subgame perfect Nash Equilibrium/rollback equilibrium.
    A complete answer should relate at least on of these terms to the international sanctions placed on Russia.
  \end{solution}

  \part[2] 
  What assumptions do the authors make in their simplified `Repeated Sanctions Game'?
  \begin{solution}
    Some potential answers are: 
    rationality of Putin, 
    the relative rankings of the payoffs in the game (i.e., international community prefers Don't Punish after Putin Transgresses over Sanctioning),
    the potential future players care about future iterations of the game enough,
    etc.
  \end{solution}

  \part[2]
  Do you find the authors' arguments convincing? Why or why not?
  \begin{solution}
    For example, consider what happens if the payoff to Player 2 of Don't Punish after a Transgression is 
    $ -c - f $ where $f$ is the cost in the future to other Players observing that Transgressions will go unpunished.
    If $f > s$, i.e., if the future cost of additional repeated Transgressions outweighs the cost of sanctions today,
    then the equilibrium of the game changes to { Transgress, (Sanction if Transgress, Don't Punish if Don't Transgress) } because now Player 2 it is rational for Player 2 to Sanction after a Transgression, 
    where it wasn't rational in the original one-shot game.
  \end{solution}

\end{parts}

\end{questions}

\end{document}
