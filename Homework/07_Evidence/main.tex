\documentclass[addpoints]{exam}

\printanswers
\CorrectChoiceEmphasis{\color{red}\bfseries}
\usepackage{amssymb, amsmath, amsfonts}
\usepackage{geometry}
\usepackage{graphicx}
\usepackage{tikz}
\usetikzlibrary{calc}
\usepackage{pgfplots}
\usepackage{multirow,array} % for payoff matrix formatting
\usepackage[colorlinks,pdfusetitle,urlcolor=blue,citecolor=blue,linkcolor=blue]{hyperref}
\usepackage{bbding} % dingbats fonts

\usepackage{adjustbox}

\definecolor{crimson}{RGB}{ 170, 4, 36 }
\definecolor{darkblue}{RGB}{ 4, 47, 170 }
\definecolor{brown}{RGB}{ 111, 71, 2 }
\definecolor{periwinkle}{RGB}{ 90, 177, 204 }
\definecolor{ducksgreen}{HTML}{007030}

\geometry{left=1.0in,right=1.0in,top=1.0in,bottom=1.0in}
\pagestyle{headandfoot}
\lhead{EC327 Game Theory}
\chead{Homework 7}
\rhead{Fall 2025}
\runningheadrule

\title{
    \textbf{Econ 327: Game Theory} \\ 
    Homework $\#7$
    }
\author{University of Oregon}
\date{Due: Nov. 26$^{th}$}

% exam-type question formatting
\renewcommand{\thequestion}{\textbf{Q\arabic{question}}}
\bracketedpoints

\begin{document}

\maketitle

\begin{center}
  \gradetable[h][questions]
\end{center}

\vspace{0.5in}

\begin{center}
  \textbf{For homework assignments:}
\end{center}

\begin{itemize}

%  \item DO NOT write your name:
%  this assignment will be graded anonymously. 
%  If you want to, you can include your student ID instead.

  \item Complete \textit{all} questions and parts.

  % I will select one question at random to be graded
  % according to the rubric on Canvas.

  \item You will be graded on not only the content of your work
    but on how clearly you present your ideas.
    Make sure that your handwriting is legible.
    Please use extra pages if you run out of space 
    but make sure that all parts of a question 
    are in the correct order when you submit.

  \item You may choose to work with others,
  but everyone must submit to Canvas individually.

  Please include the names of everyone who you worked with 
  below your own name.
 
\end{itemize}

\vspace{1.0in}

\makebox[.6\textwidth]{Name\enspace\hrulefill}


\newpage

\begin{questions}


  \question
  Find an example of a strategic setting somewhere outside of this class.
  It can relate to a news article you read, 
  something that happened to you or a friend,
  a topic or setting in another class, etc.
  Please be creative and pick something that you care about
  and that you think would be a good place to apply
  your new skills as an applied game theorist.
  \begin{parts}
    \part[4]
    Explain the setting you chose using the vocaublary of strategic games
    you have learned in this class.
    Clearly and briefly introduce your game to someone who is familiar with basic game theory,
    but not with your setting.
    \part[4]
    Create a simplified model of your game using one or more of the representations from class.
    Clearly state all relevant assumptions you are making in the process.
    My advice is to start as simple as possible.
    \part[4]
    Use the methods from this class to make a prediction
    for what equilibrium behavior might look like.
    \part[4]
    Compare your predictions to observations from the real world.
    Which of your assumptions do you think you would have to adapt
    to better reflect the complexities of the actual strategic setting?
  \end{parts}

  \newpage

  \question 
  Take a look at the data collected in class from 
  \textbf{Activity 6 - Penalty Shootout}.
  \begin{parts}
    \part[4]
    Generate a testable hypothesis using mixed-strategy Nash equilibria 
    to compare against the data from our in-class activity.
    Your hypothesis should include a specific numerical value or values.
    For example, you could find the probability of a goal being scored in equilibrium,
    or the overall distribution of strategies by kickers and goalies playing equilibrium strategy profiles.
    \part[4]
    Now compare your theoretical prediction to some statistic or set of values from class.
    Do you \textit{reject} or \textit{fail to reject} the null hypothesis that the class data
    were significantly different from the theoretical distribution?
    Feel free to use any graphical, statistical, or computational tools to help your analysis.
    \part[4]
    Evaluate the predictions generated by the theory. 
    If students' behavior in class resembled equilibrium behavior, 
    under what circumstances might you expect the predictions to break down?
    If students' behavior didn't fit with the predictions, 
    what assumptions in the theory do you think failed to hold up in real life?
  \end{parts}

  \newpage

  \question
  Reflect on the activities (besides penalty kick shootout) 
  that you have been asked to participate in during this class.
  \begin{parts}
    \part[4]
    What concepts of game theory did the activities relate to?
    Feel free to just talk about a specific one as an example here.
    \part[4]
    How well did the behavior among your classmates fit with the predictions
    of Nash equilibria in these activities?
    Would you have changed your strategy knowing what you know now?
    \part[4]
    In your opinion, which activity or activities were most helpful
    in being able to understand class material?
    Which aspects were more (or less) helpful?
    \part
    What types of activities would you recommend for a future version of this class?
    Any other feedback?
  \end{parts}
\end{questions}
\end{document}