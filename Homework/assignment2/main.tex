\documentclass[addpoints ]{exam}

\usepackage{amssymb, amsmath, amsfonts}
\usepackage{geometry}
\usepackage{graphicx}
\usepackage{tikz}
\usetikzlibrary{calc}
\usepackage{multirow,array} % for payoff matrix formatting

\definecolor{crimson}{RGB}{ 170, 4, 36 }
\definecolor{darkblue}{RGB}{ 4, 47, 170 }
\definecolor{brown}{RGB}{ 111, 71, 2 }
\definecolor{periwinkle}{RGB}{ 90, 177, 204 }
\definecolor{ducksgreen}{HTML}{007030}

\geometry{left=1.0in,right=1.0in,top=1.0in,bottom=1.0in}
\pagestyle{headandfoot}
\lhead{EC327 Game Theory}
\chead{Homework 1}
\rhead{Winter 2024}
\runningheadrule

\title{
    \textbf{Econ 327: Game Theory} \\ 
    Homework $\#2$
    }
\author{University of Oregon}
\date{Due: Feb. 6$^{th}$}

% exam-type question formatting
\renewcommand{\thequestion}{\textbf{Question \arabic{question}}}
\bracketedpoints

\begin{document}

\maketitle

\begin{center}
  \gradetable[h][questions]
\end{center}

\vspace{0.5in}

\begin{center}
  \textbf{For homework assignments:}
\end{center}

\begin{itemize}

%  \item DO NOT write your name:
%  this assignment will be graded anonymously. 
%  If you want to, you can include your student ID instead.

  \item Complete \textit{all} questions and parts.
  I will select one question at random to be graded
  according to the rubric on Canvas.

  \item You may choose to work with others,
  but everyone must submit to Canvas individually.
  Please include the names of everyone who you worked with 
  below your own name.
 
\end{itemize}

\vspace{1.0in}

\makebox[.6\textwidth]{Name\enspace\hrulefill}

\vspace{0.5in}

% \begin{center}
%   \fbox{\fbox{\parbox{5.5in}{\centering
%     Answer the questions in the spaces provided on the
%     question sheets. If you run out of room for an answer,
%     continue on the back of the page or another sheet of paper.}}}
% \end{center}

\newpage

\begin{questions}

%------------------------------------------------------------------%

\question[20]
\textbf{Multiple Choice}

\begin{parts}

  \part Consider the strategic form game below:
  \begin{table}[h!]
    \begin{center}
    \begin{tabular}{*{3}{c|}}
      \multicolumn{2}{c}{} & \multicolumn{1}{c}{$Oregon~Driver$} \\\cline{3-5}
      \multicolumn{1}{c}{} &     & $Swerve$ & $Straight$ \\\cline{2-5}
      \multirow{2}*{$California~Driver$}  & $Swerve$ & -1,-1 & 1,1 \\\cline{2-5}
                           & $Straight$ & 1,1 & -1,-1 \\\cline{2-5}
    \end{tabular}
    \end{center}
  \end{table}
  What type of game is this?
  \begin{choices}
    \choice A zero-sum game
    \choice A coordination game 
    \CorrectChoice An anti-coordination game
    \choice A prisoners' dilemma
  \end{choices}

  \part Consider the strategic form game below:
  \begin{table}[!h]
  \begin{center}
    \begin{tabular}{*{3}{c|}}
      \multicolumn{2}{c}{} &
      \multicolumn{1}{c}{Navratilova} \\ \cline{3-5}
      \multicolumn{1}{c}{} & & $DL$ & $CC$ \\ \cline{2-5}
      \multirow{2}*{Evert} & $DL$ * 50, 50 & 80, 20 \\ \cline{2-5}
        & $CC$ & 90,10 & 20, 80 \\ \cline{2-5} 
    \end{tabular}
  \end{center}
  \end{table}
  Which method would you use to solve for Nash equilibria?
  \begin{choices}
    \CorrectChoice Graphing mixed strategies
    \choice Iterative deletion of strictly dominated strategies 
    \choice Backwards induction
    \choice There are no Nash equilibria of this game.
  \end{choices}

\end{parts}

\newpage

%------------------------------------------------------------------%

\question[20] 
A game theorist is walking down the street in his neighborhood and finds \$20.
Just as he picks it up, two neighborhood kids, 
Jane and Tim,
run up to him, asking if they can have it.
Because game theorists are generous by nature, 
he says he's willing to let them have the \$20,
but only according to the following procedure:
Jane and Tim are each to submit a written request 
as to their share of the \$20. 
Let $t$ denote the amount that Tim requests for himself
and $j$ be the amount that Jane requests for herself.
Tim and Jane must choose $j$ and $t$ from the interval
$[0,20]$.
If $j + t \leq 20$, then the two receive what they requested,
and the remainder, $20 - j - t$, is split equally between them.
If, however, $j + t > 20$, then they get nothing, and the game theorist keeps the \$20.
Tim and Jane are the players in this game.
Assume that each of them has a payoff equal to the amount of money that he or she receives. 
Find all Nash equilibria.
\footnote{Harrington \textit{Games, Strategies, and Decision Making}}

%\vspace{-5cm}



%------------------------------------------------------------------

%------------------------------------------------------------------

\end{questions}

\end{document}
