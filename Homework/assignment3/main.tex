\documentclass[addpoints ]{exam}

\usepackage{amssymb, amsmath, amsfonts}
\usepackage{geometry}
\usepackage{graphicx}
\usepackage{tikz}
\usetikzlibrary{calc}
\usepackage{multirow,array} % for payoff matrix formatting
\usepackage{hyperref}
\usepackage{xcolor}
\hypersetup{
  colorlinks = true,
  linkcolor  = gray,
  urlcolor   = blue
}

\definecolor{crimson}{RGB}{ 170, 4, 36 }
\definecolor{darkblue}{RGB}{ 4, 47, 170 }
\definecolor{brown}{RGB}{ 111, 71, 2 }
\definecolor{periwinkle}{RGB}{ 90, 177, 204 }
\definecolor{ducksgreen}{HTML}{007030}

\geometry{left=1.0in,right=1.0in,top=1.0in,bottom=1.0in}
\pagestyle{headandfoot}
\lhead{EC327 Game Theory}
\chead{Homework 3}
\rhead{Winter 2024}
\runningheadrule

\title{
    \textbf{Econ 327: Game Theory} \\ 
    Homework $\#3$
    }
\author{University of Oregon}
\date{Due: Mar. 1$^{st}$}

% exam-type question formatting
\renewcommand{\thequestion}{\textbf{Question \arabic{question}}}
\bracketedpoints

\begin{document}

\maketitle

\begin{center}
  \gradetable[h][questions]
\end{center}

\vspace{0.5in}

\begin{center}
  \textbf{For homework assignments:}
\end{center}

\begin{itemize}

%  \item DO NOT write your name:
%  this assignment will be graded anonymously. 
%  If you want to, you can include your student ID instead.

  \item Complete \textit{all} questions and parts.
  I will select one question at random to be graded
  according to the rubric on Canvas.

  \item You may choose to work with others,
  but everyone must submit to Canvas individually.
  Please include the names of everyone who you worked with 
  below your own name.
 
\end{itemize}

\vspace{1.0in}

\makebox[.6\textwidth]{Name\enspace\hrulefill}

\vspace{0.5in}

\begin{center}
  \fbox{\fbox{\parbox{5.5in}{\centering
  \textbf{Note:}
  All parts of Questions 1 and 2 are either directly from 
  or adapted from problems in  
  Dixit, Skeath and Reiley, \textit{Games of Strategy}, Fourth Edition. 
  }}}
\end{center}

\newpage

\begin{questions}

%------------------------------------------------------------------%

\question[10]
\textbf{Multiple Choice}

\begin{parts}

  \part Consider the game of Rock, Paper, Scissors:
  \begin{table}[h!]
    \begin{center}
    \begin{tabular}{*{5}{c|}}
      \multicolumn{2}{c}{} & \multicolumn{3}{c}{Lisa} \\\cline{3-5}
      \multicolumn{1}{c}{} &          & Rock    & Scissors & Paper   \\\cline{2-5}
      \multirow{3}*{Bart}  & Rock     &   0,  0 &  10, -10 & -10, 10 \\\cline{2-5}
                           & Scissors & -10, 10 &   0,   0 &  10,-10 \\\cline{2-5}
                           & Paper    &  10,-10 & -10,  10 &   0   0 \\\cline{2-5} 
    \end{tabular}
    \end{center}
  \end{table}

  Suppose that Lisa announced she would use a mixture in which 
  she chooses Rock with 40\% probability, Scissors with 30\% probability,
  and Paper with 30\% probability.
  What is Bart's best response to this strategy choice?
  \begin{choices}
    \choice Rock
    \CorrectChoice Paper
    \choice Scissors
    \choice Mixed strategy of Rock with 1/3, Scissors with 1/3,
    and Paper with 1/3.
  \end{choices}

  \part Consider the following game:
  \begin{table}[!h]
  \begin{center}
    \begin{tabular}{*{4}{c|}}
      \multicolumn{2}{c}{} &
      \multicolumn{2}{c}{Colin} \\ \cline{3-4}
      \multicolumn{1}{c}{} &       & $DL$   & $CC$   \\ \cline{2-4}
      \multirow{2}*{Rowena} & $DL$ & 4, 4 & 4, 1 \\ \cline{2-4}
                            & $CC$ & 1, 1 & 6, 6 \\ \cline{2-4} 
    \end{tabular}
  \end{center}
  \end{table}

  How many Nash Equilibria exist in the simultaneous game?
  \begin{choices}
    \CorrectChoice one equilibrium
    \choice two equilbria
    \choice three equilibria
    \choice an infinite number of equilibria
  \end{choices}

  \part 
  Jack is a talented investor, 
  but his earnings vary considerably from year to year.
  In the coming year he expects to earn either $\$ 250,000$ with good luck
  or $ \$ 90,000 $ with bad luck.
  Somewhat oddly, given his chosen profession, Jack is risk averse, 
  so that his utility is equal to the square root of his income.
  The probability of Jack's having good luck is 0.5.

  What is Jack's expected utility for the coming year?
  \begin{choices}
    \choice 170
    \CorrectChoice 400
    \choice 412 
    \choice 583
  \end{choices}


  \part 
  What amount of certain income would yield the same level of utility 
  for Jack as the expected utility in part (c)?

  \begin{choices}
    \choice \$28,900 
    \CorrectChoice \$160,000
    \choice \$169,744 
    \choice \$339,889
  \end{choices}

  \newpage
  \part 
  You are the Dean of the Faculty at St. Anford University. You hire Assistant Professors for a probationary period of 7 years, after which they come up for tenure and are either promoted and gain a job for life or turned down, in which case they must find another job elsewhere.
Your Assistant Professors come in two types, Good and Brilliant. Any types worse than Good have already been weeded out in the hiring process, but you cannot directly distinguish between Good and Brilliant types. Each individual Assistant Professor knows whether he or she is Brilliant or merely Good. You would like to tenure only the Brilliant types.
The payoff from a tenured career at St. Anford is \$2 million; think of this as the expected discounted present value of salaries, consulting fees, and book royalties, plus the monetary equivalent of the pride and joy that the faculty member and his or her family would get from being tenured at St. Anford. Anyone denied tenure at St. Anford will get a fac- ulty position at Boondocks College, and the present value of that career is \$0.5 million.
Your faculty can do research and publish the findings. But each publication requires effort and time and causes strain on the family; all these are costly to the faculty member. The monetary equivalent of this cost is \$30,000 per publication for a Brilliant Assistant Professor and \$60,000 per publication for a Good one. You can set a minimum number, $N$, of publications that an Assistant Professor must produce in order to achieve tenure.
  
  What will happen in a \textit{separating equilibrium}?
  \begin{choices}
    \choice St. Anford requires $N=0$ publications, \\
    both Good and Brilliant professors are tenured.
    \CorrectChoice St. Anford requires $25 < N \leq 50$ publications, \\
    Brilliant professors are tenured, Good ones are not.
    \choice St. Anford requires $25 < N \leq 50$ publications, \\
    Good professors are tenured, Brilliant ones are not.
    \choice St. Anford requires $N>50$ publications, \\
    neither Good nor Brilliant professors are tenured.
  \end{choices}
\end{parts}
  
\newpage

%------------------------------------------------------------------%

\question
A local charity has been given a grant to serve free meals to the homeless in its community, but it is worried that its program might be exploited by nearby college students, who are always on the lookout for a free meal.
Both a homeless person and a college student receive a payoff of 10 for a free meal.
The cost of standing in line for the meal is 
$t^2 / 320$ for a homeless person and 
$t^2 / 160$ for a college student,
where t is the amount of time in line measured in minutes.
Assume that the staff of the charity cannot observe the true type of those coming for free meals.

\begin{parts}
  \part[5] What is the minimum wait time that will achieve separation of types?

  \part[5] 
  After a while, the charity finds that it can successfully identify 
  and turn away college students half of the time.
  College students who are turned away receive no free meal and,
  further, incur a cost of 5 for their time and embarrassment.
  Will the partial identification of college students 
  reduce or increase the answer in part (a)? 
  Explain.

\end{parts}

\newpage

\question 

The table below contains the counts and frequencies of each combination of
kicker and goalie strategies across 410 match-ups (so 820 plays) 
from Activity 4 - Penalty Shootout Tournament.

\begin{table}[!h]
\centering
\begin{tabular}{|c|c|c|c|}
\hline
Kick & Intercept & count & proportion\\
\hline
L & L & 37 & 0.095\\
\hline
L & M & 51 & 0.131\\
\hline
L & R & 37 & 0.095\\
\hline
M & L & 44 & 0.113\\
\hline
M & M & 53 & 0.136\\
\hline
M & R & 51 & 0.131\\
\hline
R & L & 39 & 0.100\\
\hline
R & M & 33 & 0.085\\
\hline
R & R & 45 & 0.115\\
\hline
\end{tabular}
\end{table}

Recall that the payout matrix for this rule set was different 
than the payout matrix from the worksheet 
(the probability of scoring when the kicker chose a different side than the goalie was 1 in our class activity).
 
\begin{parts}
  \part[5]
  Calculate the \textbf{Mixed Strategy Nash Equilibrium} of this version of the game 
  \part[5]
  Use a \href{https://en.wikipedia.org/wiki/Chi-squared_test}{Chi-squared test}
  to test the null hypothesis that our class data 
  are the same as the theoretical equilibrium.
\end{parts}
\end{questions}

\end{document}
