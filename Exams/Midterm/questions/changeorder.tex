\begin{question}[ID=meangirls,type=exam]{30}
Consider a group project where both players, \textbf{Regina} and \textbf{Gretchen}
can put in some level of effort, $High$, $Moderate$, or $Low$.\\
Both players earn the same grade on the project,
but their shared grade depends on the total level of effort:\\
\begin{itemize} 
  \item If both put in $Low$ effort, they earn an $F$ which both value at 0 utils
  \item If one puts in $Low$ and the other puts in $High$ effort, they earn a $D$ which they value at 1 util
  \item If either both put in $Moderate$ effort, or if one puts in $High$ while the other puts $Low$ effort,
  they earn a $B$ which they value at 3 utils
  \item If one puts $Moderate$ and the other puts $High$, they earn an $A$ which they value at 4 utils
  \item And finally, if both put in $High$, they earn an $A^+$ which they value at 5 utils
\end{itemize}
\underline{Costs to effort:}
Putting in $High$ effort costs 2.5 util,
putting in $Moderate$ effort costs 1 util,
putting in $Low$ effort costs 0 utils \\
\textbf{Each player's payoff in this game is equal to their payoff from the grade they earn,
minus the cost of the level of effort they put in.}
\begin{tasks}
%  \item (\points{10})
%  Find all Nash equilibria when both players choose their level of effort \textit{simultaneously}.
%
%  \PrintSolutionsTF{
%  \fbox{
%    \parbox{\linewidth}{
%  
%    \begin{center}
%      \begin{tabular}{*{5}{c|}}
%        \multicolumn{2}{c}{} & \multicolumn{4}{c}{\textbf{Gretchen}} \\ \cline{3-5}
%        \multicolumn{1}{c}{} & & Low & Mod & High \\ \cline{2-5}
%        \multirow{3}*{Regina}
%          & Low  & 0, 0                           & 1, 0                         & \underline{3}, \underline{1} \\ \cline{2-5}
%          & Mod  & 0, 1                           & \underline{2}, \underline{2} & \underline{3}, 1.5           \\ \cline{2-5}
%          & High & \underline{0.5}, \underline{3} & 1.5,\underline{3}            & 2.5,2.5                      \\ \cline{2-5}
%      \end{tabular}
%    \end{center}
%
%    There are three NE in the simultaneous game:
%
%    \begin{itemize}
%      \item ($High_{R}$, $Low_{G}$)
%      \item ($Mod_{R}$, $Mod_{G}$)
%      \item ($Low_{R}$, $High_{G}$)
%    \end{itemize}
%
%    }}
%  }
  \task (\points{10})
  Write out the extensive form tree when \textbf{Gretchen} has to choose her level of effort and then \textbf{Regina} chooses her level of effort.
  \PrintSolutionsTF{
  \fbox{
    \parbox{\linewidth}{
    SOLUTION
    \vspace{10cm}
    }}
  }{
  \vspace{10cm}
  }
  \task (\points{10})
  Find all subgame perfect Nash equilibria in the game when Gretchen moves first.
  A complete answer will consist of a \textbf{a complete plan of action} for both players for every decision which could possibly be reached.

  \PrintSolutionsTF{
  \fbox{
    \parbox{\linewidth}{
    SOLUTION
    \vspace{2cm}
    }}
  }
\newpage
  \task (\points{10})
  Find all Nash equilibria when \textbf{Regina} has to choose her level of effort and then \textbf{Gretchen} chooses her level of effort.
  Compare your answer to your answers in parts (a) and (b) and explain the reason for the differences.
\end{tasks}
\end{question}

\begin{question}[ID=subgame,type=exam]
  Consider the extensive form game below: \\
  \begin{adjustbox}{width=\textwidth,center}
  \vary{
  \includegraphics{questions/figures/mixed-tree1.png}
  }{
  \includegraphics{questions/figures/mixed-tree2.png}
  }
  \end{adjustbox}
  \begin{tasks}
    \task (\points{4})
    First, focus on the subgame outlined in the dashed box and labelled SG($b\ell$)\footnote{the subgame following the branches labelled b and l}. \\
    Write out this sequential subgame as a normal form table.
    \task (\points{4}) Solve for all Nash equilibria in SG($b\ell$).
    \task (\points{4}) Now based on your answer above,
    what can you say about any subgame perfect Nash equilibria in the whole game?
    \task (\points{4}) Solve for all subgame perfect Nash in the entire game.
    Outline your work to justify your answer.
    If you cannot find any Nash, explain why not.
  \end{tasks}
  \begin{solution}
    \begin{tasks}
      \task Either version acceptable:
      \begin{center}
        \begin{tabular}{*{4}{c|}}
          \multicolumn{2}{c}{} & \multicolumn{2}{c}{$P_1$} \\ \cline{3-4}
          \multicolumn{1}{c}{} & & e & f \\ \cline{2-4}
          \multirow{2}*{$P_3$}
            & q & \underline{1},3 & \underline{3},\underline{5} \\ \cline{2-4}
            & r & 0,\underline{2} & 1,1 \\ \cline{2-4}
        \end{tabular}
        \begin{tabular}{*{4}{c|}}
          \multicolumn{2}{c}{} & \multicolumn{2}{c}{$P_3$} \\ \cline{3-4}
          \multicolumn{1}{c}{} & & q & r \\ \cline{2-4}
          \multirow{2}*{$P_1$}
            & e & 3,\underline{1} & \underline{2},1 \\ \cline{2-4}
            & f & \underline{5},\underline{3} & 1,1 \\ \cline{2-4}
        \end{tabular}
      \end{center}
      \task For subgame $b\ell$, (q,f) is the only pure strategy Nash.
      \task SPE means that players are rational in every subgame.
      So a Nash in SG(bl) must be part of a SPE in the whole game.
      \task Backwards induction:
      \begin{itemize}
        \item $P_3$ chooses y over x, v over u if h or i, but u over v if j.
        Based on argument above, only rational for them to play q if $\ell$.
        \item Based on $P_3$, $P_2$ chooses l if b and j if a.
        \item Based on argument from (c), $P_1$ plays f if $\ell$,
        and using previous backwards induction,
        they should also choose b.
      \end{itemize}
      SPE is \{(b,f),($\ell$,j),(x,q,v,v,u)\}
    \end{tasks}
  \end{solution}
\end{question}