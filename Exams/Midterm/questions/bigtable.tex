\begin{question}[ID=tbl1,type=exam]{20}
Consider the strategic form game below:
\begin{table}[!h]
  \begin{center}
    \begin{tabular}{*{6}{c|}}
      \multicolumn{2}{c}{} & \multicolumn{4}{c}{$P_2$} \\ \cline{3-6}
      \multicolumn{1}{c}{} & & \vary{W}{i} & \vary{X}{j} & \vary{Y}{k} & \vary{Z}{l} \\ \cline{2-6}
      \multirow{5}*{$P_1$} 
          & \vary{A}{q} & \vary{10}{8}, \vary{1}{2} & \vary{5}{3}, \vary{4}{5}  & \vary{12}{10}, \vary{6}{7}   & \vary{2}{0}, \vary{7}{8} \\ \cline{2-6}
          & \vary{B}{r} & \vary{2}{0},  \vary{3}{4} & \vary{6}{4}, \vary{4}{5}  & \vary{13}{11}, \vary{5}{6}   & \vary{9}{7}, \vary{12}{11} \\ \cline{2-6}
          & \vary{C}{s} & \vary{11}{9}, \vary{8}{9} & \vary{3}{1}, \vary{9}{10} & \vary{10}{8},  \vary{10}{11} & \vary{6}{4}, \vary{11}{13} \\ \cline{2-6}
          & \vary{D}{t} & \vary{10}{8}, \vary{6}{7} & \vary{7}{5}, \vary{9}{10} & \vary{11}{9},  \vary{6}{7}   & \vary{6}{4}, \vary{7}{9} \\ \cline{2-6}
          & \vary{E}{v} & \vary{5}{3},  \vary{3}{4} & \vary{5}{3}, \vary{4}{5}  & \vary{8}{6},   \vary{5}{6}   & \vary{5}{3}, \vary{14}{16} \\ \cline{2-6}
  \end{tabular}
  \end{center}
\end{table}
\begin{tasks}
  \task (\points{10})
  Use Iterated Deletion of Strictly Dominated Strategies
  and write out a simplified game table with any remaining cells.
  \PrintSolutionsTF{
  \fbox{
    \parbox{\linewidth}{
    \begin{itemize}
      \item Step 1: \vary{E}{v} is strictly dominated by \vary{D}{t}, eliminate \vary{E}{v}.
      \item Step 2: \vary{W}{i} and \vary{Y}{k} are strictly dominated by \vary{Z}{l}, eliminate \vary{W}{i} and \vary{Y}{k}
      \item Step 3: \vary{A}{q} and \vary{C}{s} are strictly dominated by \vary{B}{r}, eliminate \vary{A}{q} and \vary{C}{s}
    \end{itemize}
    \begin{center}
      \begin{tabular}{*{4}{c|}}
        \multicolumn{2}{c}{} & \multicolumn{2}{c}{$P_2$} \\ \cline{3-4}
        \multicolumn{1}{c}{} & & \vary{X}{j} & \vary{Z}{l} \\ \cline{2-4}
        \multirow{2}*{$P_1$}
          & \vary{B}{r} & \vary{6}{4}, \vary{4}{5}  & \underline{\vary{13}{11}}, \underline{\vary{5}{6}} \\ \cline{2-4}
          & \vary{D}{t} & \underline{\vary{7}{5}}, \underline{\vary{9}{10}} & \vary{11}{9},  \vary{6}{7} \\ \cline{2-4}
      \end{tabular}
    \end{center}
    }}
  }{
  \vspace{8cm}
  }
  \task (\points{10})
  Find all Nash equilibria in this strategic form game.
  Explain why you know they are Nash equilibria.
  \PrintSolutionsTF{
  \fbox{
    \parbox{\linewidth}{
    The best response to \vary{X}{j} is \vary{D}{t},
    and the best response to \vary{D}{t} is \vary{X}{j} 
    so (\vary{X}{j}, \vary{D}{t}) is one Nash equilibrium.
    The best response to \vary{Z}{l} is \vary{B}{r},
    and the best response to \vary{B}{r} is \vary{Z}{l} 
    so (\vary{Z}{l}, \vary{B}{r}) is the other NE.
    These are they only pure strategy NE because we eliminated all strategies in part (a)
    that will never be played in any NE.
    We also found the intersection of either players best responses in the table from (a)
    which is the definition of a NE.
    \par\noindent\rule{\linewidth}{0.4pt}
    Partial credit may be awarded for an answer that is consistent with mistakes made in 
    eliminating strictly dominated strategies in part $a$.
    }}
  }
\end{tasks}
\end{question}

\begin{question}[ID=tbl2,type=exam]
Consider the strategic form game below:
\begin{table}[!h]
  \begin{center}
    \begin{tabular}{*{6}{c|}}
      \multicolumn{2}{c}{} & \multicolumn{4}{c}{$P_2$} \\ \cline{3-6}
      \multicolumn{1}{c}{} & & \vary{W}{i} & \vary{X}{j} & \vary{Y}{k} & \vary{Z}{l} \\ \cline{2-6}
      \multirow{5}*{$P_1$} 
          & \vary{A}{q} & \vary{9}{8}, \vary{5}{7} & \vary{6}{5}, \vary{8}{10} & \vary{10}{9},  \vary{5}{7}   & \vary{6}{4}, \vary{7}{9} \\ \cline{2-6}
          & \vary{B}{r} & \vary{9}{8}, \vary{0}{2} & \vary{4}{3}, \vary{3}{5}  & \vary{11}{10}, \vary{5}{7}   & \vary{2}{0}, \vary{7}{8} \\ \cline{2-6}
          & \vary{C}{s} & \vary{10}{9}, \vary{7}{9} & \vary{2}{1}, \vary{8}{10} & \vary{9}{8},  \vary{9}{11} & \vary{6}{4}, \vary{11}{13} \\ \cline{2-6}
          & \vary{D}{t} & \vary{1}{0},  \vary{2}{4} & \vary{5}{4}, \vary{3}{5}  & \vary{12}{11}, \vary{4}{6}   & \vary{9}{7}, \vary{12}{11} \\ \cline{2-6}
          & \vary{E}{v} & \vary{10}{3},  \vary{2}{4} & \vary{4}{3}, \vary{3}{5}  & \vary{7}{6},   \vary{4}{6}   & \vary{4}{3}, \vary{14}{4} \\ \cline{2-6}
  \end{tabular}
  \end{center}
\end{table}
\begin{tasks}
  \task (\points{4})
  Does either player have any strategies they would never play if they are rational?
  State any strictly dominated strategies and explain why they would not be played.
  \task (\points{4})
  Use Iterated Deletion of Strictly Dominated Strategies
  and write out a simplified game table with any remaining cells.
  \task (\points{4})
  Find all Nash equilibria in pure strategies.
  Explain why you know that these strategy profiles are Nash,
  or if you cannot find any explain why not.
\end{tasks}
\begin{solution}
  \begin{tasks}
    \task \vary{W}{v} is strictly dominated by \vary{X,Z}{q}
    because conditional on $\vary{P_1}{P_2}$'s strategy,
    it always results in a higher payoff.
    \task 
    \begin{enumerate}
      \item Eliminate \vary{W}{v} as a non-rational strategy
      \item Eliminate \vary{E}{i} because it's now strictly dominated by \vary{A}{j,l}
      \item Eliminate \vary{B}{k}, S.D. by \vary{D}{l}
      \item Eliminate \vary{Y}{r}, S.D. by \vary{Z}{t}
      \item Eliminate \vary{C}{s}, S.D. by \vary{D}{t}
      \item Stop, no more stritcly dominated strategies
    \end{enumerate}
    \begin{center}
      \begin{tabular}{*{4}{c|}}
        \multicolumn{2}{c}{} & \multicolumn{2}{c}{$P_2$} \\ \cline{3-4}
        \multicolumn{1}{c}{} & & \vary{X}{j} & \vary{Z}{l} \\ \cline{2-4}
        \multirow{2}*{$P_1$}
          & \vary{A}{q} & \underline{\vary{6}{5}}, \underline{\vary{8}{10}}  & \vary{6}{4}, \vary{7}{9} \\ \cline{2-4}
          & \vary{D}{t} & \vary{5}{4}, \vary{3}{5} & \underline{\vary{9}{7}},  \underline{\vary{12}{11}}\\ \cline{2-4}
      \end{tabular}
    \end{center}
    \task NE: \vary{$\{A,X\}$, $\{D,Z\}$}{$\{q,j\}$, $\{t,l\}$}
    These are they only pure strategy NE because we eliminated all strategies in part (a)
    that will never be played in any NE.
    We also found the intersection of either players best responses in the table from (b)
    which is the definition of a NE.
  \end{tasks}
\end{solution}
\end{question}
