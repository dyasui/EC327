\begin{question}[ID=meangirls,type=exam]{30}
考虑一个小组项目,其中两个玩家,\textbf{Regina} 和 \textbf{Gretchen},
可以付出不同程度的努力:$High$、$Moderate$ 或 $Low$。\\
双方在项目中获得相同的成绩,
但共享成绩取决于总努力程度:\\
\begin{itemize} 
  \item 若双方均付出$Low$努力,获得$F$等(双方效用估值0)
  \item 若一方$Low$另一方$High$努力,获得$D$等(效用估值1)
  \item 若双方均$Moderate$,或一方$High$另一方$Low$努力,获得$B$等(效用估值3)
  \item 若一方$Moderate$另一方$High$,获得$A$等(效用估值4)
  \item 若双方均$High$努力,获得$A^+$等(效用估值5)
\end{itemize}
\underline{努力成本:}
$High$努力成本为2.5效用,
$Moderate$努力成本为1效用,
$Low$努力成本为0效用 \\
\textbf{每位玩家的支付等于从成绩中获得的效用减去自身努力成本。}
\begin{tasks}
  \task (\points{10})
  写出扩展式博弈树,当\textbf{Gretchen}先选择努力程度,随后\textbf{Regina}选择努力程度。
  \PrintSolutionsTF{
  \fbox{
    \parbox{\linewidth}{
    解答
    \vspace{10cm}
    }}
  }{
  \vspace{10cm}
  }
  \task (\points{10})
  求解当Gretchen先行动时的所有子博弈完美纳什均衡。
  完整答案需包含\textbf{双方在每种可能决策路径上的完整行动计划}。
  \PrintSolutionsTF{
  \fbox{
    \parbox{\linewidth}{
    解答
    \vspace{2cm}
    }}
  }
\newpage
  \task (\points{10})
  求解当\textbf{Regina}先选择努力程度,随后\textbf{Gretchen}选择时的所有纳什均衡。
  将结果与(a)(b)部分对比,并解释差异原因。
\end{tasks}
\end{question}

\begin{question}[ID=subgame,type=exam]
  考虑以下扩展式博弈: \\
  \begin{adjustbox}{width=\textwidth,center}
  \vary{
  \includegraphics{../questions/figures/mixed-tree1.png}
  }{
  \includegraphics{../questions/figures/mixed-tree2.png}
  }
  \end{adjustbox}
  \begin{tasks}
    \task (\points{4})
    首先关注虚线框内标记为SG($b\ell$)\footnote{分支标记b和l对应的子博弈}的子博弈。\\
    将该序贯子博弈写成标准型表格。
    \task (\points{4}) 求解SG($b\ell$)中的所有纳什均衡。
    \task (\points{4}) 根据上述结果,
    关于整个游戏的子博弈完美纳什均衡你能得出什么结论?
    \task (\points{4}) 求解整个游戏的所有子博弈完美纳什均衡。
    说明推导过程。若不存在纳什均衡,解释原因。
  \end{tasks}
  \begin{solution}
    \begin{tasks}
      \task 以下任一版本均可:
      \begin{center}
        \begin{tabular}{*{4}{c|}}
          \multicolumn{2}{c}{} & \multicolumn{2}{c}{$P_1$} \\ \cline{3-4}
          \multicolumn{1}{c}{} & & e & f \\ \cline{2-4}
          \multirow{2}*{$P_3$}
            & q & \underline{1},3 & \underline{3},\underline{5} \\ \cline{2-4}
            & r & 0,\underline{2} & 1,1 \\ \cline{2-4}
        \end{tabular}
        \begin{tabular}{*{4}{c|}}
          \multicolumn{2}{c}{} & \multicolumn{2}{c}{$P_3$} \\ \cline{3-4}
          \multicolumn{1}{c}{} & & q & r \\ \cline{2-4}
          \multirow{2}*{$P_1$}
            & e & 3,\underline{1} & \underline{2},1 \\ \cline{2-4}
            & f & \underline{5},\underline{3} & 1,1 \\ \cline{2-4}
        \end{tabular}
      \end{center}
      \task 子博弈$b\ell$中,(q,f) 是唯一的纯策略纳什均衡。
      \task SPE要求玩家在每个子博弈中理性行动,
      因此SG(bl)的纳什均衡必是整个博弈SPE的组成部分。
      \task 逆向归纳:
      \begin{itemize}
        \item $P_3$ 在x/y分支选y,在h/i分支选v(优于u),但在j分支选u(优于v)。
        根据前述,仅当$\ell$时选择q是理性的。
        \item 基于$P_3$行为,$P_2$在b分支选l、在a分支选j。
        \item 基于(b)结论,$P_1$在$\ell$时选f,
        结合逆向归纳结果,其在首节点应选b。
      \end{itemize}
      SPE为 \{(b,f),($\ell$,j),(x,q,v,v,u)\}
    \end{tasks}
  \end{solution}
\end{question}
