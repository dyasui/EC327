\begin{question}[ID=1,type=exam]{4}
  理性意味着:
  \begin{tasks}
    \task 玩家的偏好是连续且独立的
    \task 玩家总是战胜对手
    \task 玩家总是基于完全信息行动
    \task \correct 玩家的偏好是完备且可传递的
  \end{tasks}
\end{question}

\begin{question}[ID=2,type=exam]{4}
  两名玩家轮流从一个装有 100 颗石子的罐子中取石子。
  每位玩家每回合可以取 1 至 5 颗石子。
  取走最后一颗石子的玩家\textbf{输掉}游戏。\\
  先手玩家在第一回合应取多少颗石子?
  \begin{tasks}
    \task \vary{1}{\correct 3}
    \task \vary{2}{4}
    \task \vary{5}{2}
    \task \vary{4}{1} 
    \task \vary{\correct 3}{5}
  \end{tasks}
\end{question}

\begin{question}[ID=3,type=exam]{4}
  如果一个结果是 \rule{1cm}{0.15mm},
  那么它 \rule{1cm}{0.15mm}
  \begin{tasks}
    \task \vary
    {\correct 从不被帕累托占优,是帕累托最优}
    {不是纳什均衡,不是帕累托最优} 
    \task \vary
    {不是纳什均衡,不是帕累托最优} 
    {\correct 从不被帕累托占优,是帕累托最优}
    \task \vary
    {被帕累托占优,是纳什均衡}
    {对所有玩家都是严格占优,是帕累托最优}
    \task \vary
    {对所有玩家都是严格占优,是帕累托最优}
    {被帕累托占优,是纳什均衡}
  \end{tasks}
\end{question}

\begin{question}[ID=4,type=exam]{4}
  考虑以下策略式博弈:
  \begin{table}[h!]
    \begin{center}
    \setlength{\extrarowheight}{2pt}
    \begin{tabular}{*{5}{c|}}
      \multicolumn{2}{c}{} & \multicolumn{3}{c}{$P_2$} \\\cline{3-5}
      \multicolumn{1}{c}{} &     & $x$ & $y$ & $z$ \\\cline{2-5}
      \multirow{3}*{$P_1$}  & $a$ & 2,0 & 6,2 & 4,2 \\\cline{2-5}
                            & $b$ & 2,5 & 8,9 & 2,2 \\\cline{2-5}
                            & $c$ & 1,4 & 5,3 & 5,1 \\\cline{2-5}
    \end{tabular}
    \end{center}
  \end{table}
  在上面的博弈中,哪个策略是严格劣势策略?
  \begin{tasks}
    \task \vary{b}{$y$}
    \task \vary{a}{\correct $z$}
    \task \vary{c}{$b$}
    \task \vary{\correct $z$}{$x$}
    \task \vary{$x$}{$a$}
  \end{tasks}
\end{question}

\begin{question}[ID=5,type=exam]{4} 
  对以下策略式博弈执行严格劣势策略迭代删除(IDSDS)直至完成。\\
  \begin{table}[h!]
    \begin{center}
    \setlength{\extrarowheight}{2pt}
    \begin{tabular}{*{5}{c|}}
      \multicolumn{2}{c}{} & \multicolumn{3}{c}{$P_2$} \\\cline{3-5}
      \multicolumn{1}{c}{} &     & 宽式 & 窄式 & 细式 \\\cline{2-5}
      \multirow{3}*{$P_1$}  & 高式  & 5,9 & 6,2 & 6,2 \\\cline{2-5}
                            & 中式   & 2,5 & 8,8 & 2,2 \\\cline{2-5}
                            & 矮式 & 6,4 & 2,3 & 5,9 \\\cline{2-5}
    \end{tabular}
    \end{center}
  \end{table}
  IDSDS 关于该博弈的纳什均衡说明了什么?
  \begin{tasks}
    \task \vary{纳什均衡为(高式,宽式)}{纳什均衡为(中式,窄式)}
    \task \vary{\correct IDSDS 本身无法确定该博弈的纳什均衡}{纳什均衡为(矮式,细式)}
    \task \vary{纳什均衡为(中式,窄式)}{纳什均衡为(高式,宽式)}
    \task \vary{纳什均衡为(矮式,细式)}{\correct IDSDS 本身无法确定该博弈的纳什均衡}
    \task \vary{纳什均衡为(高式,窄式)}{ 纳什均衡为(矮式,宽式)}
  \end{tasks}
\end{question}

\begin{question}[ID=6,type=exam]{4}
  无论其他玩家如何行动,对某个玩家而言都是最优的选择被称为:
  \begin{tasks}
    \task \vary
    {纳什策略}
    {严格劣势策略}
    \task \vary
    {帕累托最优策略}
    {置信策略}
    \task \vary
    {\correct 严格优势策略}
    {帕累托最优策略}
    \task \vary
    {严格劣势策略}
    {\correct 严格优势策略}
  \end{tasks}
\end{question}

\begin{question}[ID=7,type=exam]{4}
  \setlist{nolistsep}
  严格劣势策略迭代删除(IDSDS)的有用性在于
  \begin{tasks}
    \task \vary
    {它总能找到任何策略式博弈的所有纳什均衡}
    {它没有用处,因为起始玩家不同会导致不同结果}
    \task \vary
    {它消除了不可信威胁}
    {\correct 它可以删除纳什均衡中永远不会被使用的策略}
    \task \vary
    {\correct 它可以删除纳什均衡中永远不会被使用的策略}
    {它总能找到任何策略式博弈的所有纳什均衡}
    \task \vary
    {它没有用处,因为起始玩家不同会导致不同结果}
    {它消除了不可信威胁}
  \end{tasks}
\end{question}

\begin{question}[ID=8,type=exam]{4}
  考虑以下策略式博弈:
  \begin{table}[h!]
    \begin{center}
    \begin{tabular}{*{5}{c|}}
      \multicolumn{2}{c}{} & \multicolumn{3}{c}{$P_2$} \\\cline{3-5}
      \multicolumn{1}{c}{} &         & 左 & 中 & 右 \\\cline{2-5}
      \multirow{3}*{$P_1$}& 上       & 0,1  & 9,0    & 2,3 \\\cline{2-5}
                          & 直行 & 5,9  & 7,3    & 1,7 \\\cline{2-5}
                          & 下     & 7,5  & 10,10  & 3,5 \\\cline{2-5}
    \end{tabular}
    \end{center}
  \end{table}
  纳什均衡是什么?
  \begin{tasks}
    \task \vary{上,左}{上,中}
    \task \vary{直行,中}{直行,右}
    \task \vary{下,左}{\correct 下,中}
    \task \vary{\correct 下,中}{下,右}
  \end{tasks}
\end{question}

\begin{question}[ID=9,type=exam]{4}
  考虑以下扩展式博弈:
  \begin{figure}[!h]
    \centering
    \begin{tikzpicture}[scale=1.5,font=\footnotesize]
        \tikzstyle{solid node}=[circle,draw,inner sep=1.5,fill=black]
        \tikzstyle{hollow node}=[circle,draw,inner sep=1.5]
        \tikzstyle{level 1}=[level distance=15mm,sibling distance=3.5cm]
        \tikzstyle{level 2}=[level distance=15mm,sibling distance=1.5cm]
        \tikzstyle{level 3}=[level distance=15mm,sibling distance=1cm]
        \node(0)[solid node,label=above:{玩家1}]{}
            child{node[solid node,label=above left:{玩家2}]{}
                child{node[hollow node,label=below:{$(1,2)$}]{} edge from parent node[left]{$C$}}
                child{node[hollow node,label=below:{$(1,-1)$}]{} edge from parent node[left]{$D$}}
                child{node[hollow node,label=below:{$(0,2)$}]{} edge from parent node[right]{$E$}}
                edge from parent node[left,xshift=-5]{$A$}
            }
            child{node[solid node,label=above right:{玩家2}]{}
                child{node[hollow node,label=below:{$(2,2)$}]{} edge from parent node[left]{$F$}}
                child{node[hollow node,label=below:{$(1,3)$}]{} edge from parent node[right]{$G$}}
                edge from parent node[right,xshift=5]{$B$}
            };
    \end{tikzpicture}
  \end{figure}
  以下哪项是子博弈精炼纳什均衡?
  \begin{tasks}
    \task \vary
    {(B, CF)}
    {(A, EF)}
    \task \vary
    {(A, EG)}
    {\correct (B, CG)}
    \task \vary
    {\correct (B, CG)}
    {(B, CF)}
    \task \vary
    {(A, EF)}
    {(B, DG)}
  \end{tasks}
\end{question}

\begin{question}[ID=10,type=exam]{4}
  我对两场独立比赛进行投注。
  如果俄勒冈大学男子篮球队击败华盛顿队,我赢取 6 美元,
  如果华盛顿队获胜,我损失 6 美元。
  如果俄勒冈大学女子篮球队击败科罗拉多队,我赢取 12 美元,
  如果科罗拉多队获胜,我损失 12 美元。
  假设俄勒冈大学男子击败华盛顿的概率为 $\frac{1}{3}$,
  俄勒冈大学女子击败科罗拉多的概率为 $\frac{1}{2}$。\\
  我在两场比赛中的预期(美元)收益是多少?\\
  \begin{tasks}
    \task \vary{\correct -2\$}{0\$}
    \task \vary{-4\$}{2\$}
    \task \vary{2\$}{-4\$}
    \task \vary{0\$}{\correct -2\$}
  \end{tasks}
\end{question}

\begin{question}[ID=11,type=exam]{4}
  在\textbf{序贯博弈}中,合适的分析方法是:
  \begin{tasks}
    \task \vary
    {严格劣势策略迭代删除}
    {混合策略纳什均衡}
    \task \vary
    {\correct 逆向归纳(倒推分析)}
    {最优反应动态}
    \task \vary
    {古诺调整过程}
    {\correct 逆向归纳(倒推分析)}
  \end{tasks}
\end{question}

\begin{question}[ID=12,type=exam]{4}
  在\textbf{囚徒困境}中,相互合作:
  \begin{tasks}
    \task \vary
    {是优势策略均衡}
    {\correct 帕累托占优于相互背叛的结果}
    \task \vary
    {\correct 帕累托占优于相互背叛的结果}
    {是稳定的}
    \task \vary
    {是稳定的}
    {是可信威胁}
    \task \vary
    {是可信威胁}
    {是优势策略均衡}
  \end{tasks}
\end{question}

\begin{question}[ID=13,type=exam]{4}
  当我们通过逆向归纳求解序贯博弈时,以下哪项是我们所假设的?
  \begin{tasks}
    \task \vary{\correct 所有玩家相信其他玩家都是理性的}{所有玩家拥有完全信息}
    \task \vary{所有玩家具有相同偏好}{某些玩家具有先动优势}
    \task \vary{所有玩家拥有完全信息}{所有玩家具有相同偏好}
    \task \vary{某些玩家具有先动优势}{\correct 所有玩家相信其他玩家都是理性的}
  \end{tasks}
\end{question}

\begin{question}[ID=14,type=exam]{4}
  在两名玩家各有 2 种策略的双人博弈中,
  可能存在 4 个纯策略纳什均衡。
  \begin{tasks}
    \task \vary{错误,纯策略纳什必须唯一}{错误,2x2博弈只能使用混合策略}
    \task \vary{错误,2x2博弈只能使用混合策略}{错误,纯策略纳什必须唯一}
    \task \vary{\correct 正确,如果所有都是弱纳什均衡}{正确,如果所有都被帕累托占优}
    \task \vary{正确,如果所有均衡都被帕累托占优}{\correct 正确,如果所有均衡均为弱纳什均衡}
  \end{tasks}
\end{question}
