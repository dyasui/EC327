\documentclass{article}

\usepackage{amssymb, amsmath, amsfonts}
\usepackage{geometry}
\usepackage{graphicx}
\usepackage{enumitem}
\usepackage{tikz}
\usetikzlibrary{calc}
\usepackage{pgfplots}
\pgfplotsset{compat=1.18} 
\usepackage{multirow,array} % for payoff matrix formatting

% use exsheets to create different versions with \vary{}{}
\usepackage[load-headings]{exsheets}
\SetupExSheets{
  headings = block-wp
}

\definecolor{darkgray}{RGB}{169, 169, 169}
\definecolor{dimgray}{RGB}{105, 105, 105}

\geometry{left=1.0in,right=1.0in,top=0.75in,bottom=1.0in}
% \pagestyle{headandfoot}
% \lhead{EC327 Game Theory}
% \chead{Final Exam}
% \rhead{Winter 2024}
% \runningheadrule

\title{
    \textbf{Econ 327: Game Theory} \\ 
    Final Exam
    }
\author{University of Oregon}
\date{March 20, 2024}

\begin{document}

\maketitle

% SET VERSION NUMBER HERE: 
\variant{1}
\begin{center}
  \Large{\textbf{Version \vary{1}{2}}}
\end{center}

\begin{center}
\begin{tabular}{|l|*{\numberofquestions}{c|}c|}\hline
  Question &
    \ForEachQuestion{\QuestionNumber{#1}\iflastquestion{}{&}} &
    Total \\ \hline
  Points  &
    \ForEachQuestion{\GetQuestionProperty{points}{#1}\iflastquestion{}{&}} &
    \pointssum* \\ \hline
    Score & 
      \ForEachQuestion{\iflastquestion{}{&}} & \\ \hline 
\end{tabular}
\end{center}

\begin{itemize}
  
  \item Complete \textit{all} questions and parts. 
  All questions will be graded.

  \item Carefully explain all your answers on short and long answer questions.

  An incorrect answer with clear explanation will earn partial credit,
  an incorrect answer with no work will get zero points.

  \item 
  If you do not understand what a question is asking for, 
  ask for clarification. 

\end{itemize}

\underline{Allowed Materials:}

\begin{itemize}
 
  \item A single 5" by 3" note card

  \item A non-programmable calculator

  \item Pencils, color pens, eraser, ruler/straight-edge etc.
\end{itemize}

\vspace{1.0in}

\makebox[.6\textwidth]{Name\enspace\hrulefill}

\vspace{0.5in}

\begin{center}
  \fbox{\fbox{\parbox{5.5in}{\centering
    Answer the questions in the spaces provided on the
    question sheets. If you run out of room for an answer,
    continue on the back of the page or another sheet of paper.}}}
\end{center}

\newpage


%------------------------------------------------------------------------------%


\section*{Multiple Choice}

% MC question format:
% \begin{question}[type=exam]{2}
%   Question: 
%   \setlist{nolistsep}
%   \begin{enumerate}[label=\alph*), noitemsep]
%     \item \vary
%     {}
%     {}
%     \item \vary
%     {}
%     {}
%     \item \vary
%     {}
%     {}
%     \item \vary
%     {}
%     {}
%   \end{enumerate}
% \end{question}

\textit{For the following questions, match the term to the correct definition}

\begin{question}[type=exam]{4}
  A \textbf{mixed strategy Nash equilibrium} is a strategy profile in which:
  \setlist{nolistsep}
  \begin{enumerate}[label=\alph*), noitemsep]
    \item \vary
    {all players are indifferent between any of their pure strategies} % correct
    {each player is playing their only strictly dominant pure strategy}
    \item \vary
    {a player can reveal their own private information through their actions}
    {nobody knows anything}
    \item \vary
    {each player is playing their only strictly dominant pure strategy}
    {all players are indifferent between any of their pure strategies} % correct
    \item \vary
    {a player can reveal the private information held by others by designing an incentive mechanism}
    {a player can reveal their own private information through their actions}
  \end{enumerate}
\end{question}

\begin{question}[type=exam]{4}
  \textbf{asymmetric information}
  \setlist{nolistsep}
  \begin{enumerate}[label=\alph*), noitemsep]
    \item \vary
    {when all players know everything about the game}
    {when some players have private information that is not available to other players} % correct
    \item \vary
    {when all players are uncertain about the same things}
    {when all players know everything about the game}
    \item \vary
    {when some players have private information that is not available to other players} % correct
    {when nobody knows anything}
    \item \vary
    {when nobody knows anything}
    {when all players are uncertain about the same things}
  \end{enumerate}
\end{question}

\begin{question}[type=exam]{4}
  \textbf{\vary{screening}{signaling}}
  \setlist{nolistsep}
  \begin{enumerate}[label=\alph*), noitemsep]
    \item \vary
    {a player can reveal the private information held by others by designing an incentive mechanism} % correct
    {when only players with the `bad' condition sort into a market}
    \item \vary
    {a player can reveal their own private information through their actions}
    {a player can reveal their own private information through their actions} % correct
    \item \vary
    {when only players with the `bad' condition sort into a market}
    {a player can reveal the private information held by others by designing an incentive mechanism}
    \item \vary
    {a type of equilibrium in which only mixed strategies are played}
    {a type of equilibrium in which only mixed strategies are played}
  \end{enumerate}
\end{question}

\begin{question}[type=exam]{4}
  \textbf{Folk Theorem}
  \setlist{nolistsep}
  \begin{enumerate}[label=\alph*), noitemsep]
    \item \vary
    {No matter how hard you try, some folks will just never cooperate}
    {No matter what, any socially optimal outcome can \textit{always} be reached in equilibrium}
    \item \vary
    {The Prisoners' Dilemma is the only game with a unique Nash equilibrium}
    {No matter how hard you try, some folks will just never cooperate}
    \item \vary
    {Any \textit{individually rational} and \textit{feasible} outcome can be reached in \textit{repeated games} for \textit{some} sufficiently high enough discount factor} % correct
    {The Prisoners' Dilemma is the only game with a unique Nash equilibrium}
    \item \vary
    {No matter what, any socially optimal outcome can \textit{always} be reached in equilibrium}
    {Any \textit{individually rational} and \textit{feasible} outcome can be reached in \textit{repeated games} for \textit{some} sufficiently high enough discount factor} % correct
  \end{enumerate}
\end{question}

\textit{For the next three questions, 
match the example given with the concept from class that is the most closely related to the specific strategic setting.}

\begin{question}[type=exam]{4}
  Web servers want to provide their content to real human users 
  while also keeping out `bots' which are programs that can mimic human users to bump up engagement numbers, etc. 
  Sometimes websites will make you fill out a `captcha' test,
  like clicking all of the pictures with street lights,
  in an effort to tell which users are humans and which are bots.
  \setlist{nolistsep}
  \begin{enumerate}[label=\alph*), noitemsep]
    \item \vary{Risk Aversion}{Brinksmanship}
    \item \vary{Brinksmanship}{Grim Trigger}
    \item \vary{Risk Aversion}{Screening}
    \item \vary{Screening}{Risk Aversion}
  \end{enumerate}
\end{question}

\begin{question}[type=exam]{4}
  Every year during Oscars season, Hollywood studios spend millions on media campaigns to hype up their movies to potential Oscars awards voters. 
  If no-one did any advertising, then these studios would all spend less money. 
  But if one studio was the only one advertising, then their movies would probably win more often than competing studios' movies.
  \setlist{nolistsep}
  \begin{enumerate}[label=\alph*), noitemsep]
    \item \vary{Prisoners' Dilemma}{Credible threats}
    \item \vary{Second-mover's advantage}{Expected utility}
    \item \vary{Expected utility}{Prisoners' Dilemma}
    \item \vary{Credible threats}{Babbling equilibrium}
  \end{enumerate}
\end{question}

\begin{question}[type=exam]{4}
  When I am writing these multiple choice problems
  I try to vary the order in which the correct choice appears among the options
  to try to make it harder to guess all of the right answers.
  \setlist{nolistsep}
  \begin{enumerate}[label=\alph*), noitemsep]
    \item \vary{Mixed strategy}{Mixed strategy}
    \item \vary{Pure strategy}{Pure strategy} 
    \item \vary{Tit-for-Tat strategy}{Screening strategy} 
    \item \vary{Cheap talk strategy}{Grim Trigger strategy}
  \end{enumerate}
\end{question}

\textit{The next few questions are related to examples we saw in class}

\begin{question}[type=exam]{4}
  Consider the tennis match below:
  \begin{table}[!h]
    \centering
    \begin{tabular}{cc|c|c|}
    & \multicolumn{1}{c}{} & \multicolumn{2}{c}{Navratilova}\\
    & \multicolumn{1}{c}{} & \multicolumn{1}{c}{$DL$}  & \multicolumn{1}{c}{$CC$} \\\cline{3-4}
    \multirow{2}*{Evert}  & $DL$ & $50, 50$ & $80,20$ \\\cline{3-4}
                          & $CC$ & $90, 10$ & $20,80$ \\\cline{3-4}
  \end{tabular}
  \end{table}

  Suppose that Evert can add a first stage move in which she can either say 
  ``I will serve Down the Line (DL)'' or ``I will serve Cross-Court (CC)''.

  What will happen in equilibrium?
  \setlist{nolistsep}
  \begin{enumerate}[label=\alph*), noitemsep]
    \item \vary
    {Evert will truthfully signal her strategy $DL$, Navratilova will believe her and play $DL$}
    {No matter what Evert says, Navratilova will never believe her. Both will randomize between $DL$ and $CC$} % correct
    \item \vary
    {Evert will lie by saying she will play $DL$ but actually play $CC$, Navratilova will believe her and play $DL$}
    {Evert will lie by saying she will play $DL$ but actually play $CC$, Navratilova won't believe her so she'll play $CC$}
    \item \vary
    {Evert will lie by saying she will play $DL$ but actually play $CC$, Navratilova won't believe her so she'll play $CC$}
    {Evert will lie by saying she will play $DL$ but actually play $CC$, Navratilova will believe her and play $DL$}
    \item \vary
    {No matter what Evert says, Navratilova will never believe her. Both will randomize between $DL$ and $CC$} % correct
    {Evert will truthfully signal her strategy $DL$, Navratilova will believe her and play $DL$}
  \end{enumerate}
\end{question}

\begin{question}[type=exam]{4}
  In Akerlof's original Market for Lemons, there were only two types of used cars; good cars and lemons (bad cars).
  Only the seller of the used car knows the true quality of their car, potential buyers do not.

  What happened in the Adverse Selection equilibrium?
  \setlist{nolistsep}
  \begin{enumerate}[label=\alph*), noitemsep]
    \item \vary
    {Good cars sold for high prices, bad cars sold for low prices}
    {No cars were sold}
    \item \vary
    {No cars were sold}
    {Good cars sold for high prices, bad cars sold for low prices}
    \item \vary
    {Only bad cars were sold, good cars were never sold} % correct
    {All cars were sold for high prices}
    \item \vary
    {All cars were sold for high prices}
    {Only bad cars were sold, good cars were never sold} % correct
  \end{enumerate}
\end{question}

\begin{question}[type=exam]{4}
  In the in-class activity \textit{Penalty Shootout Tournament}, 
  we broke into pairs of one kicker and one goalie who had to simultaneously choose either \textit{Left}, \textit{Middle}, or \textit{Right}.
  The goalie scored 1 point if they chose the same direction as the kicker, otherwise the kicker scored 1 point.

  Compare the theoretical prediction to what we actually observed in the data:
  \setlist{nolistsep}
  \begin{enumerate}[label=\alph*), noitemsep]
    \item \vary
    {The theory told us that the kicker should only choose \textit{Left}, but in the data many people actually chose \textit{Right}}
    {The theory told us that a mixed strategy profile of both players playing 1/3, 1/3, 1/3 was an equilibrium, and the class data were very close to the frequencies we predicted} % correct
    \item \vary
    {The theory told us that both types of players should mix strategies, but in the data most people only stuck to one choice}
    {The theory predicted the exact same outcomes as we saw in the class data}
    \item \vary
    {The theory told us that a mixed strategy profile of both players playing 1/3, 1/3, 1/3 was an equilibrium, and the class data were very close to the frequencies we predicted} % correct
    {The theory told us that both types of players should mix strategies, but in the data most people only stuck to one choice}
    \item \vary
    {The theory predicted the exact same outcomes as we saw in the class data}
    {The theory told us that the kicker should only choose \textit{Left}, but in the data many people actually chose \textit{Right}}
  \end{enumerate}
\end{question}

\newpage
%------------------------------------------------------------------------------%

\section*{Short Answer}
\begin{question}[type=exam]{8}

  You are the Dean of the Faculty at St. Anford University. You hire Assistant Professors for a probationary period of 7 years, after which they come up for tenure and are either promoted and gain a job for life or turned down, in which case they must find another job elsewhere.
Your Assistant Professors come in two types, Good and Brilliant. Any types worse than Good have already been weeded out in the hiring process, but you cannot directly distinguish between Good and Brilliant types. Each individual Assistant Professor knows whether he or she is Brilliant or merely Good. You would like to tenure only the Brilliant types.
The payoff from a tenured career at St. Anford is \$6 million; think of this as the expected discounted present value of salaries, consulting fees, and book royalties, plus the monetary equivalent of the pride and joy that the faculty member and his or her family would get from being tenured at St. Anford. Anyone denied tenure at St. Anford will get a fac- ulty position at Boondocks College, and the present value of that career is \$1 million.
Your faculty can do research and publish the findings. But each publication requires effort and time and causes strain on the family; all these are costly to the faculty member. The monetary equivalent of this cost is \$25,000 per publication for a Brilliant Assistant Professor and \$50,000 per publication for a Good one. You can set a minimum number, $N$, of publications that an Assistant Professor must produce in order to achieve tenure.

  \begin{enumerate}[label=\Alph*)]
    
    \item (\points{4})
    What is the minimum number $N$ you could require so that only \textit{brilliant} professors apply and \textit{good} professors don't apply?

    \vspace{4cm}

    \item (\points{4}) 
    What is the maximum number $N$ that you could require so that \textit{brilliant} professors still want to apply?
    
    \vspace{4cm}

  \end{enumerate}

\end{question}


\newpage

\section*{Long Answer}
\begin{question}[type=exam]{16}

Consider the strategic form game below: 

\begin{table}[!h]
  \begin{center}
    \begin{tabular}{*{6}{c|}}
      \multicolumn{2}{c}{} & \multicolumn{4}{c}{$P_2$} \\ \cline{3-6}
      \multicolumn{1}{c}{} &  & \textbf{Hall} & \textbf{Office} & \textbf{Library} & \textbf{Bathroom} \\ \cline{2-6} 
      \multirow{4}*{$P_1$}
      & \textbf{Roof}      & 0 , 2 & 1 , 1 & 0 , 2 & 5, 0 \\ \cline{2-6}
      & \textbf{Mezzanine} & 1 , 1 & 0 , 2 & 0 , 2 & 4, 0 \\ \cline{2-6}
      & \textbf{Ground}    & 0 , 2 & 0 , 2 & 1 , 0 & 3, -1\\ \cline{2-6} 
    \end{tabular}
  \end{center}
\end{table}

\begin{enumerate}[label=\Alph*)]
  \item (\points{4})
  Find any \textbf{pure strategy} Nash equilibria
  \vspace{3cm}

  \item (\points{6})
  Consider the following mixed strategy profile: 
  \begin{itemize}
    \item Player 1 plays 1/3 \textbf{Roof}, 0 \textbf{Mezzanine}, and 2/3 \textbf{Ground}
    \item Player 2 plays 0 \textbf{Hall}, 1/2 \textbf{Office}, 1/2 \textbf{Library}, and 0 \textbf{Bathroom}
  \end{itemize}
  Check whether this is a \textbf{mixed strategy Nash equilibrium} and explain why or why not. 
  \vspace{5cm}

  \item (\points{6})
  Now consider the strategy profile:
  \begin{itemize}
    \item Player 1 plays 1/4 \textbf{Roof}, 1/2 \textbf{Mezzanine}, and 1/4 \textbf{Ground}
    \item Player 2 plays 1/3 \textbf{Hall}, 1/3 \textbf{Office}, 1/3 \textbf{Library}, and 0 \textbf{Bathroom}
  \end{itemize}
  Check whether this is a \textbf{mixed strategy Nash equilibrium} and explain why or why not.
  \vspace{5cm}

\end{enumerate}

\end{question}


\begin{question}[type=exam]{20}
  
  Consider a Wild West shootout between Earp and the Stranger.

  With probability .75, the Stranger is a Gunslinger type and the table shows Earp's and the Stranger's payoffs
  \begin{table}[!h]
  \begin{center}
    \begin{tabular}{*{4}{c|}}
      \multicolumn{2}{c}{} &
      \multicolumn{2}{c}{Gunslinger Stranger} \\ \cline{3-4}
      \multicolumn{1}{c}{} &                & Draw & Wait \\ \cline{2-4}
      \multirow{2}*{Earp} & Draw & 2, 3 & 3, 1 \\ \cline{2-4}
                                     & Wait & 1, 4 & 8, 2 \\ \cline{2-4} 
    \end{tabular}
  \end{center}
  \end{table}

  But with probability .25, the Stranger is a Cowpoke type and the table shows Earp's and the Stranger's payoffs
  \begin{table}[!h]
  \begin{center}
    \begin{tabular}{*{4}{c|}}
      \multicolumn{2}{c}{} &
      \multicolumn{2}{c}{Cowpoke Stranger} \\ \cline{3-4}
      \multicolumn{1}{c}{} &             & Draw & Wait \\ \cline{2-4}
      \multirow{2}*{Earp} & Draw & 5, 2 & 4, 1 \\ \cline{2-4}
                                  & Wait & 6, 3 & 8, 4 \\ \cline{2-4} 
    \end{tabular}
  \end{center}
  \end{table}
  \begin{enumerate}[label=\Alph*)]
    
    \item (\points{4}) 
    What is the Nash equilibrium \textbf{when Earp is always a Gunslinger}?
    \vspace{2cm}

    \item (\points{4})
    What is the Nash equilibrium \textbf{when Earp is always a Cowpoke}?
    \vspace{2cm}

    \item (\points{6}) 
    Show whether there is a \textit{Separating} equilibrium where a Gunslinger Stranger will choose Draw 
    and a Cowpoke Stranger will choose Wait.
    \vspace{3cm}

    \item (\points{6})
    Consider a strategic move variation where the Gunslinger can commit to only playing Wait 
    before Nature has assigned them a type. 

    Is this type of commitment \textit{credible}? Why or why not?

  \end{enumerate}
\end{question}

\newpage

\begin{question}[type=exam]{16}
  
  Consider the strategic form game below: 

  \begin{table}[!h]
    \centering
    \begin{tabular}{*{4}{c|}}
      \multicolumn{2}{c}{} & \multicolumn{2}{c}{Column} \\ \cline{3-4}
      \multicolumn{1}{c}{} &         & Cooperate & Defect \\ \cline{2-4}
      \multirow{2}*{Row} & Cooperate & 8  , 8 & 0, 10 \\ \cline{2-4}
                         & Defect    & 10 , 0 & 3 , 3 \\ \cline{2-4} 
    \end{tabular} 
  \end{table} 

  \begin{enumerate}[label=\Alph*)]

    \item (\points{4})
    What will happen when this game is a \textit{one-shot} game and neither player can make any strategic moves?
    \vspace{2cm}

    \item (\points{2})
    Will this outcome be \textit{Pareto optimal}?
    \vspace{2cm}

    \item (\points{4})
    What could you change about the structure of this game to ensure that a socially optimal outcome will be reached in equilibrium? 
    \vspace{3cm}

    \item (\points{6})
    Suppose that both players have a \textit{discount factor} of $\delta=3/4$.
    Can a strategy profile of both players using \textit{grim trigger} strategies 
    be sustained in the game where the strategic form game above is repeated infinitely?

    Show all calculations and explain your answer.

    \vspace{5cm}
    
  \end{enumerate}

\end{question}


\section*{Extra Credit}
Choose from either \textbf{0}, \textbf{1}, \textbf{2}, or \textbf{3} possible bonus points to receive.
I will randomly pair you with another student. 
If the sum of the points chosen from both of you is four or less, you will each receive the number of bonus points you individually selected.
However if the sum is more than four, you will receive 0 bonus points.

%------------------------------------------------------------------------------%


\end{document}
