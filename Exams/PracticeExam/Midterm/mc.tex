\begin{question}
  Rationality means that:
  \begin{tasks}
    \task Players' preferences are continuous and independent
    \task players always win over their opponents
    \task players always act on perfect information
    \task \correct players' preferences are complete and transitive
  \end{tasks}
\end{question}

\begin{question}
  Two player take turns emptying pebbles from a jar containing 100 pebbles total.
  Each player can take any number of pebbles between 1 and 5 on their turn.
  The player who takes the last pebble \textbf{wins} the game. \\
  How many pebbles should the first player take on their first turn?
  \begin{tasks}
    \task \vary{1}{3}
    \task \vary{2}{\correct 4}
    \task \vary{5}{2}
    \task \vary{\correct 4}{1} 
    \task \vary{3}{5}
  \end{tasks}
\end{question}

\begin{question}
  If an outcome is \rule{1cm}{0.15mm}, 
  then it is \rule{1cm}{0.15mm}
  \begin{tasks}
    \task \vary
    {\correct never Pareto dominated, Pareto Optimal}
    {not a Nash equilibrium, not Pareto optimal} 
    \task \vary
    {not a Nash equilibrium, not Pareto optimal} 
    {\correct never Pareto dominated, Pareto Optimal}
    \task \vary
    {Pareto dominated, a Nash equilibrium}
    {strictly dominated for everyone, Pareto optimal}
    \task \vary
    {strictly dominated for everyone, Pareto optimal}
    {Pareto dominated, a Nash equilibrium}
  \end{tasks}
\end{question}

  \begin{question}
  Consider the strategic form game below:
  \begin{table}[h!]
    \begin{center}
    \setlength{\extrarowheight}{2pt}
    \begin{tabular}{*{5}{c|}}
      \multicolumn{2}{c}{} & \multicolumn{3}{c}{$P_2$} \\\cline{3-5}
      \multicolumn{1}{c}{} &     & $x$ & $y$ & $z$ \\\cline{2-5}
      \multirow{3}*{$P_1$}  & $a$ & 2,0 & 6,3 & 4,2 \\\cline{2-5}
                            & $b$ & 2,5 & 8,9 & 2,2 \\\cline{2-5}
                            & $c$ & 1,4 & 5,3 & 5,1 \\\cline{2-5}
    \end{tabular}
    \end{center}
  \end{table}
  
  In the game above, which strategy is strictly dominated?
  
  \begin{tasks}
    \task \vary{b}{y}
    \task \vary{a}{\correct z}
    \task \vary{c}{b}
    \task \vary{\correct z}{x}
    \task \vary{x}{a}
  \end{tasks}
\end{question}

  \begin{question} 
  Perform Iterative Deletion of Strictly Dominated Strategies for the same game as above all the way to completion.
  What does IDSDS tell you about the Nash equilibrium of this game?
  \begin{tasks}
    \task \vary{\correct The NE is (b,y)}{The NE is (a, x)}
    \task \vary{The NE is (a, x)}{The NE is (a, y)}
    \task \vary{The NE is (a, y)}{\correct The NE is (b,y)}
    \task \vary{The NE is (Y, z)}{IESDS by itself does not reveal the NE of this game.}
    \task \vary{IESDS by itself does not reveal the NE of this game.}{The NE is (Y, z)}
  \end{tasks}
\end{question}

\begin{question}
  A choice that is the best for a player \textbf{no matter what everyone else is doing}
  is referred to as a:
  \begin{tasks}
    \task \vary
    {Nash strategy}
    {strictly dominated strategy}
    \task \vary
    {Pareto optimal strategy}
    {confidence strategy}
    \task \vary
    {\correct strictly dominant strategy}
    {Pareto optimal strategy}
    \task \vary
    {strictly dominated strategy}
    {\correct strictly dominant strategy}
  \end{tasks}
\end{question}

% \begin{question}[type=exam]{5} 
%   \setlist{nolistsep}
%   Iterative Deletion of Strictly Dominated Strategies 
%   is useful because
%   \begin{tasks}
%     \task \vary
%     {it will always find all Nash Equilibria of any strategic form game}
%     {it's not useful because you will get different answers depending on which player you start with}
%     \task \vary
%     {it removes non-credible threats}
%     {it can remove strategies which will never be played in a Nash Equilibrium}
%     \task \vary
%     {it can remove strategies which will never be played in a Nash Equilibrium}
%     {it will always find all Nash Equilibria of any strategic form game}
%     \task \vary
%     {it's not useful because you will get different answers depending on which player you start with}
%     {it removes non-credible threats}
%   \end{tasks}

\begin{question}
  Consider the strategic form game below:

  \begin{table}[h!]
    \begin{center}
    \begin{tabular}{*{5}{c|}}
      \multicolumn{2}{c}{} & \multicolumn{3}{c}{$P_2$} \\\cline{3-5}
      \multicolumn{1}{c}{} &         & Left & Middle & Right \\\cline{2-5}
      \multirow{3}*{$P_1$}& Up       & 0,1  & 9,0    & 2,3 \\\cline{2-5}
                          & Straight & 5,9  & 7,3    & 1,7 \\\cline{2-5}
                          & Down     & 7,5  & 10,10  & 3,5 \\\cline{2-5}
    \end{tabular}
    \end{center}
  \end{table}
  What is the Nash Equilibrium?
  \begin{tasks}
    \task \vary{Up, Left}{Up, Middle}
    \task \vary{Straight, Middle}{Straight, Right}
    \task \vary{Down, Left}{\correct Down, Middle}
    \task \vary{\correct Down, Middle}{Down, Right}
  \end{tasks}
\end{question}

\begin{question}
  Consider the extensive form game below:
  \begin{figure}[!h]
    \centering
    % example from: https://www.sfu.ca/~haiyunc/notes/Game_Trees_with_TikZ.pdf
    
    \begin{tikzpicture}[scale=1.5,font=\footnotesize]
        \tikzstyle{solid node}=[circle,draw,inner sep=1.5,fill=black]
        \tikzstyle{hollow node}=[circle,draw,inner sep=1.5]
        \tikzstyle{level 1}=[level distance=15mm,sibling distance=3.5cm]
        \tikzstyle{level 2}=[level distance=15mm,sibling distance=1.5cm]
        \tikzstyle{level 3}=[level distance=15mm,sibling distance=1cm]
        
        \node(0)[solid node,label=above:{$P1$}]{}
            child{node[solid node,label=above left:{$P2$}]{}
                child{node[hollow node,label=below:{$(1,2)$}]{} edge from parent node[left]{$C$}}
                child{node[hollow node,label=below:{$(1,-1)$}]{} edge from parent node[left]{$D$}}
                child{node[hollow node,label=below:{$(0,2)$}]{} edge from parent node[right]{$E$}}
                edge from parent node[left,xshift=-5]{$A$}
            }
            child{node[solid node,label=above right:{$P2$}]{}
                child{node[hollow node,label=below:{$(2,2)$}]{} edge from parent node[left]{$F$}}
                child{node[hollow node,label=below:{$(1,3)$}]{} edge from parent node[right]{$G$}}
                edge from parent node[right,xshift=5]{$B$}
            };
    \end{tikzpicture}
  \end{figure}

  Which of the following is a subgame-perfect Nash equilibrium?
  \begin{tasks}
    \task \vary
    {(B, CF)}
    {(A, EF)}
    \task \vary
    {(A, EG)}
    {\correct (B, CG)}
    \task \vary
    {\correct (B, CG)}
    {(B, CF)}
    \task \vary
    {(A, EF)}
    {(B, DG)}
  \end{tasks}
\end{question}

% \begin{question}
%   I make two bets on separate games.
%   If the UO men's basketball team beats Washington, I win \$6, 
%   but if Washington wins I lose \$6.
%   If the UO women's team beats Colorado, I win \$12,
%   but if Colorado wins I lose \$12.
%   Suppose that the probability of the UO men beating Washington is $\frac{1}{3}$ 
%   and the probability of the UO women beating Colorado is $\frac{1}{2}$. \\
%   What is my expected (dollar) payout across both games? \\
%   \begin{tasks}
%     \task \$0
%     \task -\$2
%     \task -\$4
%     \task \$2 
%   \end{tasks}
% \end{question}


\begin{question}
  In a \textbf{sequential-move game}, the appropriate method of analysis is:
  \begin{tasks}
    \task \vary
    {Iterated elimination of dominated strategies}
    {Nash equilibrium in mixed strategies}
    \task \vary
    {\correct backward induction (rollback analysis)}
    {best response dynamics}
    \task \vary
    {Cournot adjustment process}
    {\correct backward induction (rollback analysis)}
  \end{tasks}
\end{question}

\begin{question}
  In the \textbf{Prisoner’s Dilemma}, mutual cooperation:
  \begin{tasks}
    \task \vary
    {is a dominant strategy equilibrium}
    {\correct Pareto dominates the outcome of mutual defection}
    \task \vary
    {\correct Pareto dominates the outcome of mutual defection}
    {is stable}
    \task \vary
    {is stable}
    {is a credible threat}
    \task \vary
    {is a credible threat}
    {is a dominant strategy equilibrium}
  \end{tasks}
\end{question}