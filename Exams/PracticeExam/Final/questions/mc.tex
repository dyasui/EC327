\begin{question}[ID=1,type=exam]{4}
  For the strategic form game below:
  \begin{table}[!h]
    \begin{center}
    \setlength{\extrarowheight}{2pt}
    \begin{tabular}{*{4}{c|}}
      \multicolumn{2}{c}{} & \multicolumn{2}{c}{$P_2$} \\\cline{3-4}
      \multicolumn{1}{c}{} &      & Left & Right \\\cline{2-4}
      \multirow{2}*{$P_1$} & Up   & 3,3  & 9,4   \\\cline{2-4}
                           & Down & 5,2  & 6,1   \\\cline{2-4}
    \end{tabular}
    \end{center}
  \end{table} \\
  let $p$ be the \textbf{probability Player 1 chooses Up}
  and let $q$ be the probability \textbf{Player 2 chooses Left}.
  Choose the correct Expected Utility expression for Player 1's strategy Up.
  \begin{tasks}
    \task \vary{$3p + 5(1-q)$}{\correct $3q + 9(1-q)$}
    \task \vary{$3q + 4(1-q)$}{$3p + 5(1-q)$}
    \task \vary{$3p + 9(1-p)$}{$3q + 4(1-q)$} 
    \task \vary{\correct $3q + 9(1-q)$}{$3p + 9(1-p)$}
    \PrintSolutionsTF{Correct}{}
  \end{tasks}
\end{question}

\begin{question}[ID=2,type=exam]{4}
  For the strategic form game below:
  \begin{table}[!h]
    \begin{center}
    \setlength{\extrarowheight}{2pt}
    \begin{tabular}{*{4}{c|}}
      \multicolumn{2}{c}{} & \multicolumn{2}{c}{$P_2$} \\\cline{3-4}
      \multicolumn{1}{c}{} &      & Close & Far \\\cline{2-4}
      \multirow{2}*{$P_1$} & High & 9,5   & 5,1 \\\cline{2-4}
                           & Low  & 6,2   & 6,8 \\\cline{2-4}
    \end{tabular}
    \end{center}
  \end{table} \\
  let $p$ be the \textbf{probability Player 1 chooses High}
  and let $q$ be the probability \textbf{Player 2 chooses Close}.
  Choose the correct Expected Utility expression for \textbf{Player 2's strategy Close}.
  \begin{tasks}
    \task \vary{$5q + 1(1-q)$}{$2p + 8(1-q)$}
    \task \vary{$2p + 8(1-q)$}{\correct $5p + 2(1-p)$}
    \task \vary{\correct $5p + 2(1-p)$}{$1p + 8(1-p)$} \PrintSolutionsTF{Correct}{}
    \task \vary{$1p + 8(1-p)$}{$5q + 1(1-q)$}
  \end{tasks}
\end{question}

\begin{question}[ID=3,type=exam]{4}
  For the strategic form game below: \\
  \begin{table}[!h]
    \begin{center}
    \setlength{\extrarowheight}{2pt}
    \begin{tabular}{*{4}{c|}}
      \multicolumn{2}{c}{} & \multicolumn{2}{c}{$P_2$} \\\cline{3-4}
      \multicolumn{1}{c}{} &      & Push & Pull  \\\cline{2-4}
      \multirow{2}*{$P_1$} & Give & 6, 9 & 12, 8 \\\cline{2-4}
                           & Take & 9, 4 & 7 ,10 \\\cline{2-4}
    \end{tabular}
    \end{center}
  \end{table} \\
  Which of the following mixed strategy profiles is a Nash equilibrium?
  \begin{tasks}
    \task \vary{
      $\sigma_1$ = (1/2 Give, 1/2 Take), $\sigma_2$ = (1/2 Push, 1/2 Pull)
    }{
      \correct $\sigma_1$ = (6/7 Give, 1/7 Take), $\sigma_2$ = (5/8 Push, 3/8 Pull)
    }
    \task \vary{
      $\sigma_1$ = (2/5 Give, 3/5 Take), $\sigma_2$ = (1/2 Push, 1/2 Pull)
    }{
      $\sigma_1$ = (1/3 Give, 2/3 Take), $\sigma_2$ = (5/6 Push, 1/6 Pull)
    }
    \task \vary{
      $\sigma_1$ = (1/3 Give, 2/3 Take), $\sigma_2$ = (5/6 Push, 1/6 Pull)
    }{
      $\sigma_1$ = (1/2 Give, 1/2 Take), $\sigma_2$ = (1/2 Push, 1/2 Pull)
    }
    \task \vary{
      \correct $\sigma_1$ = (6/7 Give, 1/7 Take), $\sigma_2$ = (5/8 Push, 3/8 Pull)
    }{
      $\sigma_1$ = (2/5 Give, 3/5 Take), $\sigma_2$ = (1/2 Push, 1/2 Pull)
    }
  \end{tasks}
\end{question}

  \begin{question}[ID=4,type=exam]{4}
  Consider the strategic form game below: \\
  \begin{table}[h!]
    \begin{center}
    \setlength{\extrarowheight}{2pt}
    \begin{tabular}{*{4}{c|}}
      \multicolumn{2}{c}{} & \multicolumn{2}{c}{$P_2$} \\\cline{3-4}
      \multicolumn{1}{c}{} &    & X   & Y  \\\cline{2-4}
      \multirow{3}*{$P_1$}  & A & 2,3 & 6,1 \\\cline{2-4}
                            & B & 4,2 & 1,3 \\\cline{2-4}
                            & C & 3,1 & 2,4 \\\cline{2-4}
    \end{tabular}
    \end{center}
  \end{table} \\
  Suppose Player 1 plays A with probability $\alpha$,
  B with probability $\beta$,
  and C with probability $\gamma$.
  When will Player 1 be indifferent between playing X and playing Y?
  \begin{tasks}
    \task \vary{$2\alpha+4\beta+3\gamma=6\alpha+1\beta+2\gamma$}{$3\alpha+1\alpha=2\beta+3\beta=1\gamma+4\gamma$}
    \task \vary{$3\alpha+1\alpha=2\beta+3\beta=1\gamma+4\gamma$}{\correct $3\alpha+2\beta+1\gamma=1\alpha+3\beta+4\gamma$}
    \PrintSolutionsTF{\vary{}{Correct}}{}
    \task \vary{\correct $3\alpha+2\beta+1\gamma=1\alpha+3\beta+4\gamma$}{$\alpha=\beta=\gamma$}
    \PrintSolutionsTF{\vary{Correct}{}}{}
    \task \vary{$\alpha=\beta=\gamma$}{$2\alpha+4\beta+3\gamma=6\alpha+1\beta+2\gamma$}
  \end{tasks}
\end{question}

  \begin{question}[ID=5,type=exam]{4} 
  A player using a \textbf{mixed strategy} means that:
  \begin{tasks}
    \task \vary
    {some parts of their strategy are played simultaneously and other parts are played sequentially}
    {\correct they are internally uncertain about which action they will choose because they are acting randomly} % correct
    \task \vary
    {they are confused about what action their opponent is taking}
    {some parts of their strategy are played simultaneously and other parts are played sequentially}
    \task \vary
    {they will regret not having chosen their a pure strategy instead}
    {they are confused about what action their opponent is taking}
    \task \vary
    {\correct they are internally uncertain about which action they will choose because they are acting randomly} % correct
    {they will regret not having chosen their a pure strategy instead}
  \end{tasks}
\end{question}

\begin{question}[ID=6,type=exam]{4}
  A game featuring \textbf{asymmetric information}:
  \begin{tasks}
    \task \vary
    {\correct has some players who have access to private information which is not directly observable to others} % correct
    {is repeated multiple times by the same players}
    \task \vary
    {means that one player has a strategy with no equivalent strategy avaibable to any other player}
    {has Nature acting as a player even though she doesn't have any preference over outcomes}
    \task \vary
    {has Nature acting as a player even though she doesn't have any preferences}
    {means that one player has a strategy with no equivalent strategy avaibable to any other player}
    \task \vary
    {is repeated multiple times by the same players}
    {\correct has some players who have access to private information which is not directly observable to others} % correct
  \end{tasks}
\end{question}

\begin{question}[ID=7,type=exam]{4}
  \setlist{nolistsep}
  By \textbf{\vary{screening}{signaling}}:
  \begin{tasks}
    \task \vary
    {\correct a player attempts to learn about some private information held by others by designing an incentive mechanism} % correct
    {only players with the `bad' condition sort into a market}
    \task \vary
    {a player can reveal their own private information through their actions}
    {\correct a player can reveal their own private information through their actions} % correct
    \task \vary
    {only players with the `bad' condition sort into a market}
    {a player can reveal the private information held by others by designing an incentive mechanism}
    \task \vary
    {only mixed strategies will be played in equilibrium}
    {only mixed strategies will be played in equilibrium}
  \end{tasks}
\end{question}

\begin{question}[ID=8,type=exam]{4}
  Consider the strategic form game below:
  \begin{table}[h!]
    \begin{center}
    \begin{tabular}{*{5}{c|}}
      \multicolumn{2}{c}{} & \multicolumn{3}{c}{$P_2$} \\\cline{3-5}
      \multicolumn{1}{c}{} &         & Left & Middle & Right \\\cline{2-5}
      \multirow{3}*{$P_1$}& Up       & 0,1  & 9,0    & 2,3 \\\cline{2-5}
                          & Straight & 5,9  & 7,3    & 1,7 \\\cline{2-5}
                          & Down     & 7,5  & 10,10  & 3,5 \\\cline{2-5}
    \end{tabular}
    \end{center}
  \end{table} \\
  How many Nash equilibria exist in this simultaneous game,
  including both \textbf{pure} and \textbf{mixed} strategies?
  \begin{tasks}
    \task \correct One equilibrium
    \task Two equilibria
    \task Three equilibria
    \task An infinite number of equilibria
  \end{tasks}
\end{question}

\begin{question}[ID=9,type=exam]{4}
  An \textbf{information set}:
  \begin{tasks}
    \task \vary
    {is used by game theorists to signal how they want their games to be played}
    {holds all pieces of information which are publically observable to all players}
    \task \vary
    {tells a player what action to take}
    {is used by game theorists to signal how they want their games to be played}
    \task \vary
    {\correct contains all decision nodes which a player cannot tell the difference between when they reach that part of the game} % correct
    {tells a player what action to take}
    \task \vary
    {holds all pieces of information which are publically observable to all players}
    {\correct contains all decision nodes which a player cannot tell the difference between when they reach that part of the game} % correct
  \end{tasks}
\end{question}

\begin{question}[ID=10,type=exam]{4}
  Consider the following lottery: \\
  \begin{itemize}
    \item with probability 1/3 you will receive \$900.
    \item with probability 2/3 you only receive \$36.
  \end{itemize}
  Suppose someone has a risk-averse utility function of $u(\$x) = \sqrt{\$x}$.
  For what certain amount of dollars, $x$, will this person be indifferent
  between taking the certain payment with probability of 1
  and taking the lottery defined above?
  \begin{tasks}
    \task \correct \$144
    \task \$324
    \task \$468
    \task \$484
  \end{tasks}
\end{question}


\begin{question}[ID=11,type=exam]{4}
  Identify the class concept that most closely describes the situation below: \\
  Conspicuous consumption describes the phenomenon of buying flashy luxury goods with visible branding
  such as Louis Vuitton, Gucci, Prada, etc.
  in order to display the buyer's level of wealth to be able to afford such goods.
  \begin{tasks}
    \task \vary
    {Brinksmanship}{Risk sharing}
    \task \vary
    {Mixed Strategy Nash Equilibrium}{\correct Signaling}
    \task \vary
    {Risk sharing}{Brinksmanship}
    \task \vary
    {\correct Signaling}{Mixed Strategy Nash Equilibrium}
  \end{tasks}
\end{question}

\begin{question}[ID=12,type=exam]{4}
  In the \textbf{Prisoner’s Dilemma}, mutual cooperation:
  \begin{tasks}
    \task \vary
    {is a dominant strategy equilibrium}
    {\correct Pareto dominates the outcome of mutual defection}
    \task \vary
    {\correct Pareto dominates the outcome of mutual defection}
    {is stable}
    \task \vary
    {is stable}
    {is a credible threat}
    \task \vary
    {is a credible threat}
    {is a dominant strategy equilibrium}
  \end{tasks}
\end{question}

\begin{question}[ID=13,type=exam]{4}
  The Folk Theorem states that:
  \begin{tasks}
    \task \vary{\correct Any individually rational and feasible outcome can be reached in a repeated game for some sufficiently high enough discount factor.}{The Prisoners' Dilemma is the only game with a unique Nash equilibrium.}
    \task \vary{All Pareto optimal outcomes can always be reached in a Nash equilibrium.}{\correct Any individually rational and feasible outcome can be reached in a repeated game for some sufficiently high enough discount factor.}
    \task \vary{No matter how hard you try, some folks will just never cooperate}{All Pareto optimal outcomes can always be reached in a Nash equilibrium.}
    \task \vary{The Prisoners' Dilemma is the only game with a unique Nash equilibrium.}{No matter how hard you try, some folks will just never cooperate}
  \end{tasks}
\end{question}

\begin{question}[ID=14,type=exam]{4}
  Consider the Prisoners' Dilemma game with payoffs as shown in the strategic form table below: \\
  \begin{table}[!h]
    \begin{center}
    \setlength{\extrarowheight}{2pt}
    \begin{tabular}{*{4}{c|}}
      \multicolumn{2}{c}{} & \multicolumn{2}{c}{$P_2$} \\\cline{3-4}
      \multicolumn{1}{c}{} &           & Cooperate & Cheat \\\cline{2-4}
      \multirow{2}*{$P_1$} & Cooperate & 16, 16    &  8, 36 \\\cline{2-4}
                           &     Cheat & 36, 8     & 12, 12 \\\cline{2-4}
    \end{tabular}
    \end{center}
  \end{table} \\
  Suppose Player 2 is utilizing a 
    \textbf{Tit-for-Tat} strategy in which they will start off cooperating,
    and after that they will play whatever strategy their oponent used in the previous round. \\
    Which of the following represents Player 1's present value
    of cheating in the first period and then going cooperating in all following periods?
  \begin{tasks}
    \task 
    $36 + 0\delta + 0\delta^2 + 0\delta^3 + ... = 36$
    \task 
    \correct $36 + 8\delta + 16\delta^2 + 16\delta^3 + ... = 36 + 8\delta + 16\frac{\delta^2}{1-\delta}$
    \PrintSolutionsTF{Correct}{}
    \task 
    $16 + 16\delta + 16\delta^2 + 16\delta^3 + ... = \frac{16}{1-\delta}$
    \task 
    $36 + 12\delta + 12\delta^2 + 12\delta^3 + ... = 36 + \frac{12 \delta}{1-\delta}$
  \end{tasks}
\end{question}

\begin{question}[ID=15,type=exam]{4}
  Consider the Prisoners' Dilemma game with payoffs as shown in the strategic form table below: \\
  \begin{table}[!h]
    \begin{center}
    \setlength{\extrarowheight}{2pt}
    \begin{tabular}{*{4}{c|}}
      \multicolumn{2}{c}{} & \multicolumn{2}{c}{$P_2$} \\\cline{3-4}
      \multicolumn{1}{c}{} &           & Cooperate & Cheat \\\cline{2-4}
      \multirow{2}*{$P_1$} & Cooperate & 3, 3 & 1, 4 \\\cline{2-4}
                           &     Cheat & 4, 1 & 2, 2 \\\cline{2-4}
    \end{tabular}
    \end{center}
  \end{table} \\
  Suppose Player 2 is utilizing a 
    \textbf{Grim Trigger} strategy in which they will start off cooperating,
    and continue to cooperate unless their opponent has ever played Cheat,
    in which case they will play Cheat in all periods following. \\
    Which of the following represents Player 1's present value
    of cheating in the first period (and in all following periods)?
    $\delta$ is the per-period discount rate.
  \begin{tasks}
    \task 
    \correct $4 + 2\delta + 2\delta^2 + 2\delta^3 + ... = 4 + 2\frac{\delta}{1-\delta}$
    \PrintSolutionsTF{Correct}{}
    \task 
    $3 + 3\delta + 3\delta^2 + 3\delta^3 + ... = \frac{3}{1-\delta}$
    \task
    $4 + 3\delta + 3\delta^2 + 3\delta^3 + ... = 4 + 3\frac{\delta}{1-\delta}$
    \task
    $4 + 1\delta + 2\delta^2 + 2\delta^3 + ... = 4 + 1\delta + 2\frac{\delta^2}{1-\delta}$
  \end{tasks}
\end{question}
