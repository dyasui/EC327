\noindent\textbf{University Policies:\footnote{All of the following is from UO TEP's starter syllabus}}
The University of Oregon and I are dedicated to fostering inclusive learning environments for all students and welcomes students with disabilities into all of the University's educational programs. The Accessible Education Center (AEC) assists students with disabilities in reducing campus-wide and classroom-related barriers. If you have or think you have a disability (\url{https://aec.uoregon.edu/content/what-disability}) and experience academic barriers, please contact the AEC to discuss appropriate accommodations or support. Visit 360 Oregon Hall or \url{https://aec.uoregon.edu/} for more information. You can contact AEC at 541-346-1155 or via email at \href{mailto:uoaec@uoregon.edu}{uoaec@uoregon.edu}.

\vskip.15in
\noindent\textbf{Your Wellbeing and Basic Needs:}
Students often feel overwhelmed or stressed, experience anxiety or depression, struggle with relationships, or just need help navigating challenges in their life. If you're facing such challenges, you don't need to handle them on your own--there's help and support on campus. 
As your instructor if I believe you may need additional support, I will express my concerns, the reasons for them, and refer you to resources that might be helpful. It is not my intention to know the details of what might be bothering you, but simply to let you know I care and that help is available. Getting help is a courageous thing to do—for yourself and those you care about. 

University Health Services helps students cope with difficult emotions and life stressors. If you need general resources on coping with stress or want to talk with another student who has been in the same place as you, visit the Duck Nest (located in the EMU on the ground floor) and get help from one of the specially trained Peer Wellness Advocates.

University Counseling Services (UCS) has a team of dedicated staff members to support you with your concerns, many of whom can provide identity-based support. All clinical services are free and confidential. Find out more at \url{https://counseling.uoregon.edu} or by calling 541-346-3227 (anytime UCS is closed, the After-Hours Support and Crisis Line is available by calling this same number).

Being able to meet your basic needs is foundational to your success as a student at the University of Oregon. If you are having difficulty affording food, don’t have a stable, safe place to live, or are struggling to meet another need, visit the \href{https://basicneeds.uoregon.edu/}{UO Basic Needs Resource page} for information on how to get support. They have information food, housing, healthcare, childcare, transportation, technology, finances (including emergency funds), and legal support.

If your need is {\color{darkred} urgent}, please contact the \href{https://dos.uoregon.edu/help}{Care and Advocacy Program} by calling {\color{darkred}541-346-3216}, filling out the Community Care and Support form, or by scheduling an appointment with an advocate.

\vskip.15in
\noindent\textbf{Mandatory Reporter Status:}
I am an assisting employee\footnote{This means I only report any information shared to me to the administration if you request I do so.}. For information about my reporting obligations as an employee, please see Employee Reporting Obligations on the Office of Investigations and Civil Rights Compliance (OICRC) \href{https://investigations.uoregon.edu/employee-responsibilities}{website}. Students experiencing sex or gender-based discrimination, harassment or violence should call the 24-7 hotline 541-346-SAFE [7244] or visit \href{https://safe.uoregon.edu/}{safe.uoregon.edu} for help. Students experiencing all forms of prohibited discrimination or harassment may contact the Dean of Students Office at 5411-346-3216 or the non-confidential Title IX Coordinator/OICRC at 541-346-3123. Additional resources are available at UO’s \href{https://investigations.uoregon.edu/how-get-support}{How to Get Support webpage}.

\vskip.15in
\noindent\textbf{Academic Integrity:}
The University Student Conduct Code (available on the Student Conduct Code and Procedures \href{https://policies.uoregon.edu/vol-3-administration-student-affairs/ch-1-conduct/student-conduct-code}{webpage}) defines academic misconduct. Students are prohibited from committing or attempting to commit any act that constitutes academic misconduct. By way of example, students should not give or receive (or attempt to give or receive) unauthorized help on assignments or examinations without express permission from the instructor. Students should properly acknowledge and document all sources of information (e.g. quotations, paraphrases, ideas) and use only the sources and resources authorized by the instructor. If there is any question about whether an act constitutes academic misconduct, it is the students’ obligation to clarify the question with the instructor before committing or attempting to commit the act. Additional information about a common form of academic misconduct, plagiarism, is available at the Libraries' \href{https://researchguides.uoregon.edu/citing-plagiarism}{Citation and Plagiarism page}.

\vskip.15in
\noindent\textbf{Accommodations for Religious Observances:}
The University of Oregon respects the right of all students to observe their religious holidays, and will make reasonable accommodations, upon request, for these observances. If you need to be absent from a class period this term because of a religious obligation or observance, please fill out the Student Religious Accommodation Request \href{https://provost.uoregon.edu/religious-observance-accommodations-policy}{fillable PDF form} and send it to me within the first weeks of the course so we can make arrangements in advance. 

\subsection*{Expectations Regarding Diversity and Inclusion:}
  The UO Economics Department welcomes and respects diverse experiences, perspectives, and approaches. Both nationwide and at the University of Oregon, disproportionately few women and members of historically underrepresented racial and ethnic minority groups graduate with degrees in economics. All class participants are expected to communicate with respect and to avoid behaviors or contributions that undermine, demean, or marginalize others based on race, ethnicity, gender, sex, age, sexual orientation, religion, ability, or socioeconomic status.

