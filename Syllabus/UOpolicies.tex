\section*{University Policies:\footnote{Adapted from UO TEP's starter syllabus}}

The UO Economics Department welcomes and respects diverse experiences, perspectives, and approaches. Both nationwide and at the University of Oregon, disproportionately few women and members of historically underrepresented racial and ethnic minority groups graduate with degrees in economics. All class participants are expected to communicate with respect and to avoid behaviors or contributions that undermine, demean, or marginalize others based on race, ethnicity, gender, sex, age, sexual orientation, religion, ability, or socioeconomic status.

I believe that everyone deserves an equal access to higher education.
If you believe that you face specific barriers to your learning that make it more difficult to keep up with your peers, there are resources at UO to help you.

\subsection*{Accessible Education}

The \textbf{Accessible Education Center (AEC)} provides services such as:

\begin{itemize}

\item Testing accommodations such as reduced-distraction environments and official requests for longer time to take exams if necessary.

\item Other classroom accomodations like peer note-taking, accessable technology, etc.

\end{itemize}

If you have or think you have a \href{https://aec.uoregon.edu/content/what-disability}{disability} or an obstacle to your learning, please contact the AEC to discuss appropriate accommodations or support:
\begin{itemize}
\item Visit 360 Oregon Hall or \url{https://aec.uoregon.edu/} for more information.
\item Call AEC at 541-346-1155 or email to \href{mailto:uoaec@uoregon.edu}{uoaec@uoregon.edu}.
\end{itemize}

\subsection*{Mandatory Reporter Status:}

I am a \textbf{Designated Reporter}.
This means that I must report any alleged incident of discriminatory misconduct to the Office of Equal Opportunity and Access (OEOA) if I become aware of them.

\begin{itemize}

\item For the difference between designated reporters and confidential employees, please refer to the Employee Reporting Obligations on the 
 \href{https://investigations.uoregon.edu/employee-responsibilities}{Office of Investigations and Civil Rights Compliance (OICRC) website}.

\end{itemize}

If you are experiencing sex or gender-based discrimination, harassment or violence:
\begin{itemize}
\item Call the 24-7 hotline 541-346-SAFE [7244] or visit \href{https://safe.uoregon.edu/}{safe.uoregon.edu}
\end{itemize}

If you are experiencing \textit{any} form of prohibited discrimination or harassment:
\begin{itemize}
\item contact the Dean of Students Office at 5411-346-3216
\item or the non-confidential Title IX Coordinator/OICRC at 541-346-3123.
\end{itemize}

Additional resources are available at UO’s \href{https://investigations.uoregon.edu/how-get-support}{How to Get Support webpage}.

\subsection*{Academic Integrity:}
UO's \href{https://policies.uoregon.edu/vol-3-administration-student-affairs/ch-1-conduct/student-conduct-code}{student code of conduct} defines \textbf{plagiarism} as;
``Presenting another’s material as one’s own, including using another’s words, results, processes or ideas,
  in whole or in part, without giving appropriate credit.''

A (non-exhaustive) list of acts which can be considered \textbf{academic misconduct} in this class:

\begin{itemize}
\item Taking a quiz or exam for another person, or providing answers to a quiz or exam question to someone else.
\item Providing the work of another classmate, tutor, or online poster from Chegg, Course Hero, Stack Exchange, etc. as your own without proper attribution.
\item Providing an answer which was generated by an AI large language model (like ChatGPT) without a clear explanation of how it was generated.
\end{itemize}

See this class's \hyperlink{policy:AI}{AI tool use policy} for more details

\subsection*{Your Wellbeing and Basic Needs:}

Students often feel overwhelmed or stressed, experience anxiety or depression, struggle with relationships, or just need help navigating challenges in their life.
If you're facing such challenges, you don't need to handle them on your own--there's help and support on campus. 
As your instructor I am here to support you, and I will refer you to appropriate resources if I believe you may need support outside of what I can provide.
Getting help is a courageous thing to do—for yourself and those you care about. 

\begin{itemize}

\item \textbf{University Health Services}: (located in the EMU on the ground floor) Help students cope with difficult emotions and life stressors. You can talk to a specially trained Peer Wellness Advocate about problems you are facing.

\item \href{https://counseling.uoregon.edu}{\textbf{University Counseling Services (UCS)}}: Offers free and confidential clinical counseling services, wellbeing assesments, in-person, hybrid, or teletherapy options.
Call 541-346-3227 if UCS is closed to access the After-Hours Support and Crisis Line.

\item \href{https://basicneeds.uoregon.edu/}{\textbf{UO Basic Needs}}:
They have information on food, housing, healthcare, childcare, transportation, technology, finances (including emergency funds), and legal support.

\item If your need is \textbf{\color{darkred} urgent}, contact the \href{https://dos.uoregon.edu/help}{Care and Advocacy Program} by calling \textbf{\color{darkred}541-346-3216}, filling out the Community Care and Support form, or by scheduling an appointment with an advocate.

\end{itemize}

\subsection*{Accommodations for Religious Observances:}
If you need to be absent from a class period this term because of a religious obligation or observance, please fill out the Student Religious Accommodation Request \href{https://provost.uoregon.edu/religious-observance-accommodations-policy}{fillable PDF form} and send it to me within the first weeks of the course so we can make arrangements in advance. 
