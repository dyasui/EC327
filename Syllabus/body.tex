% Template Syllabus from UO econ department
% w/ minor changes 
\section*{Course Decscription}

This course introduces the field of game theory, which involves the analysis of strategic situations; where the consequences one person's choice depend on the actions of others.
The examples of settings we may discuss range from simple games like rock, paper, scissors; to more complicated games of incomplete information in the market for used cars. 
We will introduce the general concept of equilibrium and explore the different forms of equilbria in different settings.
Students will learn through in-person lectures and discussions,
and build their own skills through participation in in-class activities.
This elective is suitable for economics majors and anyone interested in understanding the strategic dynamics that shape our decisions.

\section*{Course Policies}

\subsection*{Course Pages:}

 \href{https://canvas.uoregon.edu/courses/274671}{\textbf{Canvas}}:
  The Canvas page will contain the most up to date versions of announcements, due dates, lectures, etc. 

\subsection*{Course Modality:}

This is an in-person class that will meet at the times specified in the UO Class Schedule.
The \hyperlink{sec:attendance}{attendance policy} below specifies the manner in which absences are managed.
Please consult with the Accessible Education Center if you need accommodations related to a disability.

\subsection*{Prerequisites:}

  One of EC101 or EC201.

\subsection*{Objectives:}

This course aims to help you to understand the fundamental principles of game theory, become familiar with basic classes of games, practice the mathematical skills to solve for the appropriate equilibria, apply game theoretic models to real-world interactions, and recognize limitations and critiques of the theory.

\subsection*{Textbook:}

  \textit{Games of Strategy} by Avinash Dixit, Susan Skeath, and David McAdams, published by W.W. Norton.
  E-book versions are available from the \href{https://wwnorton.com/books/Games-of-Strategy}{publisher}.
  
\subsubsection*{Additional Sources:}

  I also like Joseph E. Harrington, \textit{Games Strategies and Decision Making}, Worth Publishers, 2008.
  % Giacomo Bonanno, \textit{Game Theory}, $2^{nd}$ edition, 2018: is a free open source game theory textbook available here; \url{https://faculty.econ.ucdavis.edu/faculty/bonanno/PDF/GT_book.pdf}, but it is a lot more technical and math heavy.

 \noindent Any other readings I assign (for example articles, news stories, etc.) will be posted on Canvas.

\hypertarget{attendance policy}{\subsection*{Attendance:}}

  Regular attendance is essential and expected.
  You will demonstrate your attendence by participating in regular in-class activities and group discussions.
  If you are unable to attend class when we have an activity, you may make up for your lost participation by designing an alternative activity which you can complete on your own time in consultation with the instructor.
  Only up to 2 missed activities may be substituted for per quarter.

\hypertarget{grading_activity}{\subsubsection{In-class Activities}}

  During class times, we will sometimes play games where you will be asked to break into groups and play a short game that relates to ideas we are learning.
  You will be asked to record what happens during the activity so that we can collect and analyze the class's data.
  Your Activities grade will be determinded by your participation (i.e., were you present, respectful, and engaged) as well as a short paragraph where you debrief the activity and relate it to class content.

  \hypertarget{grading_HW}{\subsection{Homework}}

  Several homework assignments will be posted on Canvas throughout the term. You will post your submissions as \textit{a single pdf file} including all work. You will be graded according to the rubric posted on Canvas. 
  Answer keys will be automatically posted on the due date in Canvas, so no late submissions will be accepted. However, your lowest grade will be dropped at the end of the term.

 
\hypertarget{grading_exam}{\subsection{Exams:}}
  There will be one midterm in week 5 and one final exam administered at the time specified in the Registrar’s final exam schedule.
  If a student cannot attend the midterm due to extreme circumstances, a re-weighting of the student’s grade towards the final may be considered.
  To qualify for re-weighting, a request must be submitted to the instructor no later than two days after the midterm.

\section{Grading Policy:} 

\begin{center}\begin{minipage}{3.8in}\begin{flushleft}
    \hyperlink{grading_activity}{Activities} \dotfill 10\% \\
    \hyperlink{grading_HW}{Homework}         \dotfill 30\% \\
    \hyperlink{grading_exam}{Midterm Exam}   \dotfill 30\% \\
    \hyperlink{grading_exam}{Final Exam}     \dotfill 30\% \\
\end{flushleft}\end{minipage}\end{center}

