% Template Syllabus from UO econ department
% w/ minor changes 
\section*{Course Description}

This course introduces the field of game theory, which involves the analysis of strategic situations; where the consequences one person's choice depend on the actions of others.
The examples of settings we may discuss range from simple games like rock, paper, scissors; to more complicated games of incomplete information in the market for used cars. 

We will introduce the general concept of equilibrium and explore the different forms of equilbria in different settings.
Students will learn through in-person lectures and discussions,
and build their own skills through participation in in-class activities.

This elective is suitable for economics majors and anyone interested in understanding the strategic dynamics that shape our decisions.

\section{Course Policies}

\subsection{Course Pages:}

\begin{itemize}

\item \href{https://canvas.uoregon.edu/courses/274671}{\textbf{Canvas}}:
  The Canvas class page will be where you submit graded assignments and where you can check your current grade.

\item \textbf{Course Website:} \url{https://dyasui.github.io/EC327/}

  This website will be kept up to date with the current class schedule, activities, assignments, and slides.

\end{itemize}

\subsection{Course Modality:}

This is an in-person class that will meet at the times specified in the UO Class Schedule.
The \hyperlink{grade:attendence}{attendance policy} below specifies the manner in which absences are managed.
Please consult with the Accessible Education Center if you need accommodations related to a disability.

\subsection{Prerequisites:}

  One of EC101 or EC201.

\subsection{Objectives:}

This course aims to help you to understand the fundamental principles of game theory, become familiar with basic classes of games, practice the mathematical skills to solve for the appropriate equilibria, apply game theoretic models to real-world interactions, and recognize limitations and critiques of the theory.

\subsection{Textbook:}

  \textit{Games of Strategy} by Avinash Dixit, Susan Skeath, and David McAdams, \textbf{5th edition}.
  E-book versions are available from the \href{https://wwnorton.com/books/Games-of-Strategy}{publisher}.

  You will need a copy of the textbook to complete the assigned readings.
  There will also be some homework questions drawn from the book, so make sure you have an up-to-date copy.
  
  \subsubsection{Additional Sources:}

  If you want to see another introductory game theory textbook which has a slightly different focus,
I would recommend: Joseph E. Harrington, \textit{Games Strategies and Decision Making}, Worth Publishers, 2008.
  % Giacomo Bonanno, \textit{Game Theory}, $2^{nd}$ edition, 2018: is a free open source game theory textbook available here; \url{https://faculty.econ.ucdavis.edu/faculty/bonanno/PDF/GT_book.pdf}, but it is a lot more technical and math heavy.

\noindent Any other readings I assign (for example articles, news stories, etc.) will be posted on the course website.

 \hypertarget{attendance policy}{\subsection{Attendance:}}

  Regular attendance is essential and expected.
  My expectation is that you have done at least some of the assigned reading before coming to class so that you are ready to ask questions and engage in discussion.
  You will demonstrate your attendence by participating in regular in-class \hyperlink{grade:activity}{activities} and group discussions.

\section{Grading Policies:} 

\begin{center}\begin{minipage}{3.8in}\begin{flushleft}
    \hyperlink{grade:activity}{Activities} \dotfill 15\% \\
    \hyperlink{grade:HW}{Homework}         \dotfill 25\% \\
    \hyperlink{grade:quiz}{Quizzes}        \dotfill 10\% \\
    \hyperlink{grade:exam}{Midterm Exam}   \dotfill 25\% \\
    \hyperlink{grade:exam}{Final Exam}     \dotfill 25\% \\
\end{flushleft}\end{minipage}\end{center}

  \hypertarget{grade:activity}{\subsection{Activities}}

  Throughout the course on several days, you will be asked to break into groups and play a short game that relates to ideas we are learning.
  You will be asked to record what happens during the activity so that we can collect and analyze the class's data.
  Your Activities grade will be determinded by your participation (i.e., were you present, respectful, and engaged)
  as well as a short report where you debrief the activity and relate it to class content.

  If you are unable to attend class when we have an activity, you may make up for your lost participation by designing an alternative activity which you can complete on your own time in consultation with the instructor.
  Only up to 2 missed activities may be substituted for per quarter.

  \hypertarget{grade:HW}{\subsection{Homework}}

  To help practice your understanding of the material, you will complete a written homework assignment covering the corresponding chapter (~9-10).
  You will post your submissions as \textit{a single pdf file} including all work. You will be graded according to the rubric posted on Canvas. 
  Answer keys will be automatically posted on the due date in Canvas, so no late submissions will be accepted.
  However, your lowest grade will be dropped at the end of the term.

  \hypertarget{grade:quiz}{\subsection{Quizzes}}

  There will also be a number of quizzes posted on Canvas which are mostly designed to keep you up to date with the reading.
  There will be one quiz for each assigned chapter and they will be due the day before we are scheduled to begin the lecture on that new content.
  Your lowest quiz score will be dropped.
 
  \hypertarget{grade:exam}{\subsection{Exams:}}

  Exams will be in-person and closed-note.

  There will be one midterm exam around week 5 and one final exam administered at the time specified in the Registrar’s final exam schedule.
  Each exam will have a multiple-choice section (based on quiz questions) and some long-answer questions (based on homework).
  You are allowed to bring a 3x5 inch handwritten notecard (front and back sides).

  If you cannot attend the midterm due to extreme circumstances, you may choose to move the weight that would have been applied to the midterm score to your final exam instead.
  To qualify for re-weighting, a request must be submitted to the instructor no later than two days after the midterm.

