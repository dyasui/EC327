% Template Syllabus from UO econ department
% w/ minor changes 
\section{Course Policies}
\subsection*{Course Pages:}

 \href{https://canvas.uoregon.edu/courses/233504}{\textbf{Canvas}}:
  The Canvas page will contain the most up to date versions of announcements, due dates, lectures, etc. 

\subsection*{Office Hours:} 

  \textbf{In Person:} Tuesday: 2-3pm, Friday: 1:30-2:30
  in \href{https://map.uoregon.edu/4f4d713e0}{PLC 523}.
  
  \textbf{Online:} Email me ahead of time so I know to set up a meeting link.

\subsection*{Course Modality:}

This is an in-person class that will meet at the times specified in the UO Class Schedule. The Attendance policy below specifies the manner in which absences are managed. Please consult with the Accessible Education Center if you need accommodations related to a disability.

\subsection*{Objectives:}

This course aims to help you to understand the fundamental principles of game theory, become familiar with basic classes of games, practice the mathematical skills to solve for the appropriate equilibria, apply game theoretic models to real-world interactions, and recognize limitations and critiques of the theory.

\subsection*{Textbook:}

  \textit{Games of Strategy} by Avinash Dixit, Susan Skeath, and David McAdams, published by W.W. Norton.
  E-book versions are available from the \href{https://wwnorton.com/books/Games-of-Strategy}{publisher}.
  
\subsubsection*{Additional Sources:}

  I also like Joseph E. Harrington, \textit{Games Strategies and Decision Making}, Worth Publishers, 2008. Giacomo Bonanno, \textit{Game Theory}, $2^{nd}$ edition, 2018: is a free open source game theory textbook available here; \url{https://faculty.econ.ucdavis.edu/faculty/bonanno/PDF/GT_book.pdf}, but it is a lot more technical and math heavy.

 \noindent Any other readings I assign (for example articles, news stories, etc.) will be posted on Canvas.

\subsection*{Prerequisites:}

  One of EC101 or EC201.

\subsection*{Attendance:}

  Regular attendance is essential and expected. Of course, exceptional circumstances may prevent you from attending class, but be aware that you are responsible for catching up on any missed content in class. Slides and other materials will be posted on Canvas

\section{Grading Policy:} 

\begin{center} \begin{minipage}{3.8in}\begin{flushleft}
    \hyperlink{grading_activity}{Activities} \dotfill 10\% \\
    \hyperlink{grading_HW}{Homework} \dotfill 20\% \\
    \hyperlink{grading_exam}{Midterm Exam} \dotfill 30\% \\
    \hyperlink{grading_exam}{Final Exam} \dotfill 40\% \\
\end{flushleft}\end{minipage}\end{center}

\hypertarget{grading_activity}{\subsection{In-class Activities}}

  During class times, we will sometimes play games where you will be asked to participate, record outcomes, and connect them to the class concepts. These will be graded by completion. 

  \hypertarget{grading_HW}{\subsection{Homework}}

  Several homework assignments will be posted on Canvas throughout the term. You will post your submissions as \textit{a single pdf file} including all work. You will be graded according to the rubric posted on Canvas. 
  Answer keys will be automatically posted on the due date in Canvas, so no late submissions will be accepted. However, your lowest grade will be dropped at the end of the term.

 
\hypertarget{grading_exam}{\subsection{Exams:}}
  There will be one midterm in week 5 and one final exam administered at the time specified in the Registrar’s final exam schedule. If a student cannot attend the midterm due to extreme circumstances, a re-weighting of the student’s grade towards the final may be considered. To qualify for re-weighting, a request must be submitted to the instructor no later than two days after the midterm.


