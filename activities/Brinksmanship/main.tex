\documentclass[12pt]{exam}
\usepackage{amssymb,amsmath,amsfonts,mathtools} % useful math fonts and symbols
\usepackage{hyperref}
\usepackage{multicol, multirow}
\usepackage[inner=1.0in, outer=1.0in, top=1.0in, bottom=1.0in]{geometry}

\pagestyle{headandfoot}
\lhead{EC327}
\chead{Activity 7 - Cuban Missile Crisis}
\rhead{\today}
\runningheadrule 

\hypersetup{
    colorlinks=true,
    linkcolor=blue,
    filecolor=magenta,      
    urlcolor=cyan,
}

\title{Activity 7 - Cuban Missile Crisis}
\date{\today}

\begin{document}

\section*{Rules of the Game}

For this activity, you will be grouped into either a team representing the \textbf{Soviet Union} or \textbf{United States} leadership.
Among both the USSR and US, you will be randomly assigned to take on the role of either a \textbf{hawk} or \textbf{dove}.

The game has several stages. 
The US team will first discuss and vote on whether to threaten the USSR with a blockade of Cuba.
If they decide on a blockade, the USSR can respond by withdrawing their missiles, or defying the blockade.
If the USSR has defied a US blockade, use a \href{https://pickerwheel.com/tools/random-number-generator/}{random number generator} and see if the confrontation escalates to nuclear war with 60\% likelihood.
If this doesn't lead to nuclear war, the US can choose to make a new threat that nuclear war will occur with 100\% likelihood if the USSR doesn't withdraw their missiles.

The game has five outcomes: 
\begin{itemize}
  \item If the US does not decide to blockade Cuba, 
  \begin{itemize}
    \item US dove players will get -2 for avoiding nuclear war for now, but allowing the USSR to assert their power
    \item US hawk players will get -8 because they are more worried about Soviet military dominance
    \item All USSR players will get a payoff of 2 for asserting their dominance
  \end{itemize}
  \item If the US establishes the blockade and the USSR decide to withdraw,
  \begin{itemize}
    \item US doves will get 1
    \item US hawks will get 2
    \item USSR hawks will get -8 for losing face to the US
    \item USSR doves will get -4 because the withdrawal is less embarrassing to them 
  \end{itemize}
  \item If the US blockades and the USSR decide to defy the US blockade, 
  there is a chance that the confrontation will lead to nuclear war.
  In the case of nuclear war:
  \begin{itemize}
    \item All doves will get -8
    \item All hawks will get -4
  \end{itemize}
  \item If the US blockades, the USSR defies the blockade but it didn't escalate to nuclear war,
  the US threatened again and the USSR finally withdrew their missiles,
  \begin{itemize}
    \item US doves will get 0
    \item US hawks will get 1
    \item USSR doves will get -6
    \item USSR hawks will get -10
  \end{itemize}
  \item If the US blockades, the USSR defies the blockade but it didn't escalate to nuclear war,
  the US threatened again and the USSR defied the second threat, there is 100\% chance of nuclear war.
\end{itemize}

\section*{Playing the Game}

Before each decision, you will discuss amongst your team what you think your strategy should be
(but try not to let the other team hear).
Pick one person to be the president/premier who will moderate the discussion, 
one person to be the diplomat who will communicate your teams actions to the other team,
and one person to be the journalist who will take notes of your discussion and thought processes.

\section*{Discussion}

\begin{enumerate}
  \item How would you model this game in extensive form? 
  \item What theoretical tools from the class are relevant here? How did they affect your behavior?
  \item How realistic do you think the payoff values that I assigned to each outcome are? How would the results of the game change with different combinations of payoffs?
  \item What other settings can you think of which involve brinksmanship? 
\end{enumerate}

\end{document}

