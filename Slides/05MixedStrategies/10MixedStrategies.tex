
\begin{frame}
\frametitle{Internal Uncertainty}
\begin{itemize}
	\item Now that we have a good grasp of external uncertainty, in the form of states of nature, let's talk about \alert{internal uncertainty}.
	\item Internal uncertainty occurs when the players themselves do something which creates uncertainty or randomness: this basically means that the players pick their strategies randomly.
	\item Picking a strategy at random is really just a different kind of strategy, called a \alert{mixed strategy}.
\end{itemize}
\end{frame}

\begin{frame}
\frametitle{Mixed Strategies}
\begin{itemize}
	\item The kind of strategy we've been working with up until now, in which the player always does the same thing, is called a \alert{pure strategy}.
	\item A mixed strategy assigns a probability to each of a player's pure strategies. Much like a lottery, the probabilities in a mixed strategy must all be between 0 and 1, and must sum to exactly 1.
	\begin{itemize}
		\item A mixed strategy can assign 0 probability to a pure strategy. It can even assign probability 1 to a single pure strategy, and probability 0 to all others; this is still, technically, a mixed strategy, but it is a trivial one.
	\end{itemize}
	\item When a player uses a mixed strategy, it turns the \textbf{other} player's payoffs into lotteries.
\end{itemize}
\end{frame}

\begin{frame}
\frametitle{Mixed Strategies in the Deer Hunt}
\begin{itemize}
	\item Consider the Deer Hunt:
\end{itemize}
\begin{table}[h]
\centering
\begin{tabular}{cc|c|c|}
	& \multicolumn{1}{c}{} & \multicolumn{2}{c}{Ogg}\\
	& \multicolumn{1}{c}{} & \multicolumn{1}{c}{$Deer$}  & \multicolumn{1}{c}{$Rabbit$} \\\cline{3-4}
	\multirow{2}*{Igg}  & $Deer$ & $2, 2$ & $0, 1$ \\\cline{3-4}
	& $Rabbit$ & $1, 0$ & $1, 1$ \\\cline{3-4}
\end{tabular}
\end{table}
\begin{itemize}
	\item Suppose that Igg hunts Deer 3/4 of the time, and Rabbit 1/4 of the time. (A mixed strategy assigning equal probability to both pure strategies.)
	\item If Ogg hunts Deer, then 3/4 of the time, the strategy profile that actually occurs will be (Deer, Deer), and the other 1/4 it will be (Rabbit, Deer).
	\item Ogg's expected payoff from playing Deer will be $0.75(2) + 0.25(0) = 1.5$.
\end{itemize}
\end{frame}

\begin{frame}
\frametitle{Mixed Strategies in the Deer Hunt: Generalizing}
\begin{itemize}
	\item We can generalize this approach to calculate Ogg's expected payoffs from any strategy that Igg chooses to play:
	\item Suppose that Igg plays Deer with probability p, and Rabbit with probability 1 - p.
	\item Then Ogg's expected payoff from Deer is $2(p) + 0(1 - p) = 2p$, and from Rabbit, it is $1(p) + 1(1 - p) = 1$.
	\item Note that Ogg's expected payoff from Deer gets larger with p: the more likely Igg is to hunt Deer, the more attractive an option it becomes for Ogg.
\end{itemize}
\end{frame}

\begin{frame}
\frametitle{When to Play a Mixed Strategy?}
\begin{itemize}
	\item It's possible for a mixed strategy to be a best response to the other player's strategy: this is the case if and only if all of the mixed strategy's \alert{components} (pure strategies that are assigned positive probability) are best responses too.
	\item Some intuition: If a strategy is not a best response, you should not play it---even as part of a mixed strategy.
	\item If a player only has two pure strategies, it becomes simple to tell when a mixed strategy is a best response: the mixed strategy must be a mixture of those two pure strategies, and the only way that both of them are best responses is if they have equal expected payoffs.
	\item Taking the Deer Hunt as an example, the only way that it can be a best response for Ogg to play a mixed strategy is if Deer and Rabbit provide Ogg with equal expected payoffs: we must have $2p = 1$, or $p = \frac{1}{2}$.
\end{itemize}
\end{frame}

\begin{frame}
\frametitle{What Mixed Strategy to Play}
\begin{itemize}
	\item However, if any mixed strategy is a best response, then \textbf{all} mixed strategies (with the same components) are also best responses.
	\item Intuitively, if the pure strategies going into a mixed strategy are just as good as each other, then it doesn't matter what proportions you mix them in.
	\item This means that, while it's easy to solve for \textbf{when} it's rational for a player to use a mixed strategy, there's no way to solve for a particular mixed strategy that the player \textbf{should} play.
\end{itemize}
\end{frame}

\begin{frame}
\frametitle{Mixed-Strategy Nash Equilibrium}
\begin{itemize}
	\item We still have a way to solve for the Nash equilibria when the players are allowed to use mixed strategies: we will solve for the conditions under which a player would be willing to use a mixed strategy.
	\item This means that we're going to use one player's expected payoffs to solve for the \textbf{other} player's mixed strategy: it's a little bit different from what we've done before.
\end{itemize}
\end{frame}

\begin{frame}
\frametitle{MSNE in the Deer Hunt}
\begin{itemize}
	\item Returning to the Deer Hunt, let's say that Igg plays Deer with probability $p$ and Rabbit with probability $1 - p$...
	\item While Ogg plays Deer with probability $q$ and Rabbit with probability $1 - q$.
	\begin{itemize}
		\item This is simply a framework for describing each player's mixed strategies: we're saying that Igg and Ogg each play Deer some of the time (p or q) and Rabbit the rest of the time (1 - p or 1 - q).
	\end{itemize}
	\item We already saw that Ogg's expected payoffs from Deer and Rabbit are $2p$ and $1$, respectively, and that Ogg would only play a mixed strategy if $p = \frac{1}{2}$.
	\item Likewise, Igg's expected payoffs are $2q$ and $1$, and Igg will play a mixed strategy if $q = \frac{1}{2}$.
	\item The MSNE in this game can be written as \{(p, 1 - p), (q, 1 - q)\} = \{(1/2, 1/2), (1/2, 1/2)\}.
\end{itemize}
\end{frame}

\begin{frame}
\frametitle{Error-Checking}
\begin{itemize}
	\item As mentioned, this way of solving for MSNEs is unlike what we've done before---and it can be counterintuitive at first.
	\item It's important to make sure that you're setting up the equations used to solve for a player's strategy correctly:
	\begin{itemize}
		\item Remember that you are creating an equation to describe when a player is indifferent between their pure strategies: if you're trying to figure out when \textbf{Player 1} is indifferent, you need to use \textbf{Player 1's} payoffs.
		\item However, when calculating expected payoffs, the probabilities will be based on the \textbf{other} player's mixed strategy: in a game with mixed strategies, the randomness a player deals with is created by the \textbf{other} player---not themselves.
	\end{itemize}
\end{itemize}
\end{frame}

\begin{frame}
\frametitle{Another Example: Bach or Stravinsky}
\begin{table}[h]
	\centering
	\begin{tabular}{cc|c|c|}
		& \multicolumn{1}{c}{} & \multicolumn{2}{c}{Stravinsky Fan}\\
		& \multicolumn{1}{c}{} & \multicolumn{1}{c}{Bach (q)}  & \multicolumn{1}{c}{Strav. (1 - q)} \\\cline{3-4}
		\multirow{2}*{Bach Fan}  & Bach (p) & 3, 2 & 0,0 \\\cline{3-4}
		& Strav. (1 - p) & 0,0 & 2, 3 \\\cline{3-4}
	\end{tabular}
\end{table}
\begin{itemize}
	\item To begin with, we're going to set up the players' mixed strategies the same way as in the Deer Hunt (as shown in the table).
	\item The Bach Fan's expected payoffs will be $U_B(Bach) = 3q + 0(1 - q) = 3q$ and $U_B(Strav.) = 0q + 2(1 - q) = 2 - 2q$.
	\item Likewise, the Stravinsky Fan will have payoffs $U_S(Bach) = 2p$ and $U_S(Strav.) = 3 - 3p$.
\end{itemize}
\end{frame}

\begin{frame}
\frametitle{MSNE in Bach or Stravinsky}
\begin{itemize}
\item The Bach Fan will be indifferent between Bach and Stravinsky (and thus willing to play a mixed strategy) when $3q = 2 - 2q$, or when $q = \frac{2}{5}$.
\item The Stravinsky Fan will be indifferent when $2p = 3 - 3p$, or when $p = \frac{3}{5}$.
\item Thus, the MSNE in this game will be \{(0.6, 0.4), (0.4, 0.6)\}.
\item Note that the players' asymmetric preferences result in each of them buying a ticket for their more preferred concert most of the time in this MSNE.
\item If we gave them stronger preferences (i.e. increased the amount by which they prefer their favorite composer), it would amplify this effect in the MSNE.
\end{itemize}
\end{frame}

\begin{frame}
\frametitle{iClicker Q1}
\begin{itemize}
\item Consider the following game table. What are Player 1's expected payoffs, given Player 2's mixed strategy?
\end{itemize}
\begin{table}[h]
\centering
\begin{tabular}{cc|c|c|}
	& \multicolumn{1}{c}{} & \multicolumn{2}{c}{Player 2}\\
	& \multicolumn{1}{c}{} & \multicolumn{1}{c}{$Up (q)$}  & \multicolumn{1}{c}{$Down (1 - q)$} \\\cline{3-4}
	\multirow{2}*{Player 1}  & $Up (p)$ & 2, -2 & -3, 3 \\\cline{3-4}
	& $Down (1 - p)$ & -5, 5 & 1, -1 \\\cline{3-4}
\end{tabular}
\end{table}
\begin{enumerate}
\item $U_1(Up) = 5q - 3, U_1(Down) = 1 - 6q$
\item $U_1(Up) = 3 - 5q, U_1(Down) = 6q - 1$
\item $U_1(Up) = 5 - 7q, U_1(Down) = 1 - 6p$
\item $U_1(Up) = 7p - 5, U_1(Down) = 1 - 4p$
\item $U_1(Up) = 5 - 7p, U_1(Down) = 4p - 1$
\end{enumerate}
\end{frame}

\begin{frame}
\frametitle{iClicker Q2}
\begin{itemize}
\item Consider the following game table. What are \textbf{Player 2's} expected payoffs, given Player 1's mixed strategy?
\end{itemize}
\begin{table}[h]
\centering
\begin{tabular}{cc|c|c|}
& \multicolumn{1}{c}{} & \multicolumn{2}{c}{Player 2}\\
& \multicolumn{1}{c}{} & \multicolumn{1}{c}{$Up (q)$}  & \multicolumn{1}{c}{$Down (1 - q)$} \\\cline{3-4}
\multirow{2}*{Player 1}  & $Up (p)$ & 2, -2 & -3, 3 \\\cline{3-4}
& $Down (1 - p)$ & -5, 5 & 1, -1 \\\cline{3-4}
\end{tabular}
\end{table}
\begin{enumerate}
\item $U_2(Up) = 5q - 3, U_2(Down) = 1 - 6q$
\item $U_2(Up) = 3 - 5q, U_2(Down) = 6q - 1$
\item $U_2(Up) = 5 - 7q, U_2(Down) = 1 - 6p$
\item $U_2(Up) = 7p - 5, U_2(Down) = 1 - 4p$
\item $U_2(Up) = 5 - 7p, U_2(Down) = 4p - 1$
\end{enumerate}
\end{frame}

\begin{frame}
\frametitle{iClicker Q3}
\begin{itemize}
\item The correct answers to the previous two questions were:
\begin{itemize}
\item $U_1(Up) = 5q - 3, U_1(Down) = 1 - 6q$.
\item $U_2(Up) = 5 - 7p, U_2(Down) = 4p - 1$.
\end{itemize}
\item Based on this, what are $p$ and $q$ in the MSNE of this game?
\begin{enumerate}
\item $p = 4/11, q = 5/11$
\item $p = 4/11, q = 6/11$
\item $p = 6/11, q = 4/11$
\item $p = 7/11, q = 5/11$
\item $p = 7/11, q = 6/11$
\end{enumerate}
\end{itemize}
\end{frame}

\begin{frame}
\frametitle{An MSNE With Only One Mixed Strategy}
\begin{itemize}
	\item Consider the following game table:
\end{itemize}
\begin{table}[h]
	\centering
	\begin{tabular}{cc|c|c|}
		& \multicolumn{1}{c}{} & \multicolumn{2}{c}{Player 2}\\
		& \multicolumn{1}{c}{} & \multicolumn{1}{c}{$X~(q)$}  & \multicolumn{1}{c}{$Y~(1 - q)$} \\\cline{3-4}
		\multirow{2}*{Player 1}  & $A~(p)$ & 2, 2 & 3, 2 \\\cline{3-4}
		& $B~(1 - p)$ & 4, 3 & 0, 0 \\\cline{3-4}
	\end{tabular}
\end{table}
\begin{itemize}
	\item The players' expected payoffs are:
	\begin{itemize}
		\item $U_1(A) = 2q + 3(1 - q) = 2q + 3 - 3q = 3 - q$.
		\item $U_1(B) = 4q + 0(1 - q) = 4q$.
		\item $U_2(X) = 2p + 3(1 - p) = 2p + 3 - 3p = 3 - p$.
		\item $U_2(Y) = 2p + 0(1 - p) = 2p$.
	\end{itemize}
\end{itemize}
\end{frame}

\begin{frame}
\frametitle{An MSNE With Only One Mixed Strategy}
\begin{itemize}
\item Based on this, the conditions under which each player will use a mixed strategy are:
\end{itemize}
\begin{align*}
Player~1: && Player~2:&\\
3 - q &= 4q & 3 - p &= 2p\\
3 &= 5q & 3 &= 3p\\
q &= 3/5 & p &= 1
\end{align*}
\begin{itemize}
\item We've never seen anything like $p = 1$ in this context before...
\item $p = 1$ tells us that Player 2 will only play a mixed strategy if Player 1 plays the mixed strategy where p = 1...in other words, if they only play A, which isn't really a mixed strategy at all.
\item This usually occurs when one strategy \textbf{weakly} dominates another.
\end{itemize}
\end{frame}

\begin{frame}
\frametitle{An MSNE With Only One Mixed Strategy}
\begin{itemize}
	\item We can still approach this the same way that we have in the past:
	\item Suppose that in the MSNE, Player 1 plays a (non-trivial) mixed strategy. Then Player 2 must also play a mixed strategy, in which q = 3/5.
	\begin{itemize}
		\item But Player 2 will only play a mixed strategy if Player 1 plays the mixed strategy where p = 1...which is a trivial mixed strategy. This is a contradiction, and it means that there is no MSNE where Player 1 plays a non-trivial mixed strategy.
	\end{itemize}
\end{itemize}
\end{frame}

\begin{frame}
\frametitle{An MSNE With Only One Mixed Strategy}
\begin{itemize}
\item Approach it the other way next: Suppose Player 2 plays a non-trivial mixed strategy. Then Player 1 must play A as a pure strategy.
\begin{itemize}
	\item Player 2 will play A if $3 - q \geq 4q$, i.e. if $3/5 \geq q$.
\end{itemize}
\item This lets Player 2 play a non-trivial mixed strategy! There is no contradiction here.
\item There are a range of MSNEs here: all strategy profiles of the form \{(1, 0), (q, 1 - q)\}, in which $q \in (0, 3/5]$, are MSNEs.
\item There are also two trivial MSNEs, \{(1, 0), (0, 1)\} and \{(0, 1), (1, 0)\}, which are really just the pure-strategy Nash equilibria (A, Y) and (B, X) expressed in the form of an MSNE.
\item We will cover a more methodical way to find MSNEs like this later.
\end{itemize}
\end{frame}

\begin{frame}
\frametitle{Absence of MSNEs}
\begin{itemize}
	\item Let us return to the Prisoner's Dilemma and check for MSNEs:
\end{itemize}
\begin{table}[h]
	\centering
	\begin{tabular}{cc|c|c|}
		& \multicolumn{1}{c}{} & \multicolumn{2}{c}{Luca}\\
		& \multicolumn{1}{c}{} & \multicolumn{1}{c}{$Testify~(q)$}  & \multicolumn{1}{c}{$Keep~Quiet~(1-q)$} \\\cline{3-4}
		\multirow{2}*{Guido}  & $Testify~(p)$ & $-10,-10$ & $0,-20$ \\\cline{3-4}
		& $Keep~Quiet~(1-p)$ & $-20,0$ & $-1,-1$ \\\cline{3-4}
	\end{tabular}
\end{table}
\begin{itemize}
	\item Guido and Luca's expected payoffs are:
	\begin{itemize}
		\item $U_G(Testify) = -10q + 0(1 - q) = -10q$.
		\item $U_G(Keep Quiet) = -20q + (-1)(1 - q) = -1 - 19q$.
		\item $U_L(Testify) = -10p + 0(1 - p) = -10p$.
		\item $U_L(Keep Quiet) = -20p + (-1)(1 - p) = -1 - 19p$.
	\end{itemize}
\end{itemize}
\end{frame}

\begin{frame}
\frametitle{Absence of MSNEs}
\begin{itemize}
\item Guido will play a mixed strategy if:
\end{itemize}
\begin{align*}
-10q &= -1 - 19q\\
9q &= -1\\
q &= -1/9
\end{align*}
\begin{itemize}
\item But -1/9 is not a valid probability...
\item We could also note that if $q\in [0, 1]$, which is the range for valid probabilities, $-10q$ is always greater than $-1 - 19q$. In other words, as we saw weeks ago, $Testify$ strictly dominates $Keep~Quiet$...so why would Guido mix between the two of them?
\item The same logic applies to Luca as well: neither Guido or Luca would ever wish to play a mixed strategy.
\end{itemize}
\end{frame}

\begin{frame}
\frametitle{Getting Bad Probabilities}
\begin{itemize}
	\item If you've set up the expected-payoff equation, and solved for a player's mixed strategy, and you find that the probability is less than 0, or more than 1...
	\item \textbf{It means something is wrong.} Probability can only be between 0 and 1 (inclusive).
	\item First of all, double-check your math---it could be an algebra error.
	\item But if you're confident in your math, this means that there is \textbf{no way that the player would ever play a mixed strategy}: in fact, they have a strictly dominated strategy.
	\item There will be no MSNE where this player uses a mixed strategy---but there might be MSNEs where the other player does, so you should still check that.
\end{itemize}
\end{frame}

\begin{frame}
\frametitle{MSNE in a Larger Game}
\begin{itemize}
	\item Suppose that we have this 3$\times$2 game:
\end{itemize}
\begin{table}[h]
\centering
\begin{tabular}{cr|c|c|}
	& \multicolumn{1}{c}{} & \multicolumn{2}{c}{Player 2}\\
	& \multicolumn{1}{c}{} & \multicolumn{1}{c}{X (r)}  & \multicolumn{1}{c}{Y (1 - r)} \\\cline{3-4}
	\multirow{3}*{Player 1}  & A (p) & 2, 1 & 0, 1 \\\cline{3-4}
	& B (q) & 1, 2 & 2, 0 \\\cline{3-4}
	& C (1 - p - q) & 0, 0 & 3, 2 \\\cline{3-4}
\end{tabular}
\end{table}
\begin{itemize}
	\item First, note that in this game, Player 1's mixed strategy uses probabilities p, q, and 1 - p - q, since they have three pure strategies.
	\item As a rule of thumb, a player's mixed strategy will need one variable less than their number of strategies.
\end{itemize}
\end{frame}

\begin{frame}
\frametitle{MSNE in a Larger Game}
\begin{itemize}
	\item To begin with, let's put together Player 1's expected payoffs, of which there will be three:
	\begin{itemize}
		\item $U_1(A) = 2r + 0 = 2r$.
		\item $U_1(B) = 1r + 2(1 - r) = 2 - r$.
		\item $U_1(C) = 0 + 3(1 - r) = 3 - 3r$.
	\end{itemize}
	\item Next, let's see what it would take to get Player 1 to mix different pairs of strategies:
	\begin{itemize}
		\item A and B: $2r = 2 - r \implies r = \frac{2}{3}$.
		\item A and C: $2r = 3 - 3r \implies r = \frac{3}{5}$.
		\item B and C: $2 - r = 3 - 3r \implies r = \frac{1}{2}$.
	\end{itemize}
	\item Note that each pair of strategies requires a different value of $r$: there is no mixed strategy for Player 2 that would make Player 1 willing to mix all three of their pure strategies.
\end{itemize}
\end{frame}

\begin{frame}
\frametitle{MSNE in a Larger Game}
\begin{itemize}
	\item Let's check Player 2's expected payoffs next:
	\begin{itemize}
		\item $U_2(X) = 1p + 2q + 0$.
		\item $U_2(Y) = 1p + 0 + 2(1 - p - q)$.
	\end{itemize}
	\item So Player 2 will play a mixed strategy if $p + 2q = p + 2(1 - p - q) \implies q = 1 - p - q$.
	\item There are two ways that this can be true: Either Player 1 plays B and C with equal probability (and we know from earlier that they would \textbf{only} be playing these two, not A), or Player 1 plays A only, and B and C not at all. 
\end{itemize}
\end{frame}

\begin{frame}
\frametitle{MSNE in a Larger Game}
\begin{itemize}
	\item So, one type of MSNE is where Player 1 only plays A: this requires $2r \geq 2 - r$ and $2r \geq 3 - 3r$, which imply that $r \geq \frac{2}{3}$ and $r \geq \frac{3}{5}$.
	\begin{itemize}
		\item MSNE: \{(1, 0, 0), (r, 1 - r)\}, where $r \geq \frac{2}{3}$.
	\end{itemize}
	\item And the other type of MSNE is where Player 1 plays B and C with equal (1/2) probability, and Player 2 plays X and Y with equal (1/2) probability.
	\begin{itemize}
		\item MSNE: \{(0, 1/2, 1/2), (1/2, 1/2)\}
	\end{itemize}
\end{itemize}
\end{frame}

