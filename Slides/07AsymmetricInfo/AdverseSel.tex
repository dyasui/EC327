
\begin{frame}{Adverse Selection - Definition}
  \begin{block}{Adverse Selection}
  When one player knows something about the outcomes that others don't 
  and direct communication will not \textit{credibly} signal their information,
  there can be separating equilibria in which only those players
  with the 'undesirable' states of information will 
  self-\textit{select} into engaging in the market.
  \end{block}
\end{frame}

% - - - - - - - - - - - - - - - - - - - - - - - - - - - - - - - - - - - - - - -

\begin{frame}{Adverse Selection - Examples}
  \underline{\Large{Insurance Markets}}
  \begin{itemize}
    \item Potential buyers of insurance have different risk levels
    that they know about themselves but are not easily (or legally)
    observable to insurance plan providers.
    \begin{itemize}
      \item Underlying health conditions, riskier driving habits, etc.
    \end{itemize}
    \item Insurance providers have to pay out more often
    on these riskier customers.
    \item However, riskier customers are the exact types 
    who will find insurance plans more attractive.
  \end{itemize}
\end{frame}

% - - - - - - - - - - - - - - - - - - - - - - - - - - - - - - - - - - - - - - -

\begin{frame}{Adverse Selection - Examples}
  \underline{\Large{Insurance Markets}}
  \begin{itemize}
    \item What strategies could an insurance provider take 
    to \textit{screen} out risky customers from safe customers?
    \item Any ideas?
  \end{itemize}
\end{frame}

% - - - - - - - - - - - - - - - - - - - - - - - - - - - - - - - - - - - - - - -

\begin{frame}{Adverse Selection - Examples}
  \underline{\Large{Insurance Markets}}
  \begin{itemize}
    \item Suppose there are only two categories of risk level.
    \item An insurance provider could offer two different plans:
    \begin{itemize}
      \item Plan 1: has a lower premium
      but covers a lower percent of the customer's loss.
      \item Plan 2: has a higher premium 
      but covers a higher percent of the loss.
    \end{itemize}
  \end{itemize}
\end{frame}

% - - - - - - - - - - - - - - - - - - - - - - - - - - - - - - - - - - - - - - 

\begin{frame}{Adverse Selection - Examples}
  \underline{\Large{Insurance Markets}}
  \begin{itemize}
    \item How should the insurance provider structure the two plans 
    so that a \textit{separating equilibrium} is achieved 
    in which all risky types choose a different plan than the safe types?
    \begin{itemize}
      \item Make Plan 2 only attractive to safe types by setting the price higher than the risky types' willingness to pay for the extra security.
      \item and Plan 2 a better option for safe types than Plan 1.
      \begin{itemize}
        \item We call this \alert{incentive compatible}.
      \end{itemize}
      \item Make Plan 1 a better option for risky types than going without insturance altogether
      \begin{itemize}
        \item We call this \alert{individually rational}.
      \end{itemize}
    \end{itemize}
  \end{itemize}
\end{frame}

% - - - - - - - - - - - - - - - - - - - - - - - - - - - - - - - - - - - - - - -

\begin{frame}{Adverse Selection - Market for Lemons}
  In economics, one of the most famous examples of adverse selection 
  comes from George Akerlof's 1970 paper, 
  "The Market for Lemons: Qualitative Uncertainty and the Market Mechanism".
  \begin{itemize}
    \item He analyses the used car market, 
    in which crappy cars are called 'lemons'
  \end{itemize}
\end{frame}

% - - - - - - - - - - - - - - - - - - - - - - - - - - - - - - - - - - - - - - -

\begin{frame}{Adverse Selection - Market for Lemons}
  \begin{itemize}
    \item Suppose there are only two types of cars:
    \begin{itemize}
      \item good quality cars are valued at \$12,500 to the seller
      \item and lemons are worth \$3,000. 
    \end{itemize}
    \item Suppose a potential buyer would be willing to pay more than these values. 
    \begin{itemize}
      \item He would be willing to pay \$16,000 for a car he knows is good
      \item and \$6,000 for a car he knows is a lemon.
    \end{itemize}
  \end{itemize}
\end{frame}

% - - - - - - - - - - - - - - - - - - - - - - - - - - - - - - - - - - - - - - -

\begin{frame}{Adverse Selection - Market for Lemons}
  In a perfectly competitive market with perfect info
  and a large number of buyers:
  \begin{itemize}
    \item buyer competition will drive up prices to 
    \$16,000 for good cars and \$6,000 for lemons.
  \end{itemize}
\end{frame}

% - - - - - - - - - - - - - - - - - - - - - - - - - - - - - - - - - - - - - - -

\begin{frame}{Adverse Selection - Market for Lemons}
  However, buyers don't know the true value of any car by looking at them,
  but the seller knows whether their car is a lemon or not.
  \begin{itemize}
    \item Now we have a market with \alert{aymmetric info}
    \item When the type of car is unobservable,
    there can only be \textit{one price} in the market
  \end{itemize}
\end{frame}

% - - - - - - - - - - - - - - - - - - - - - - - - - - - - - - - - - - - - - - -

\begin{frame}{Adverse Selection - Market for Lemons}
   Suppose that a fraction $f$  of cars are lemons \\ 
   Draw the extensive form game tree
\end{frame}

% - - - - - - - - - - - - - - - - - - - - - - - - - - - - - - - - - - - - - - -

\begin{frame}[plain]{}
  
\end{frame}

% - - - - - - - - - - - - - - - - - - - - - - - - - - - - - - - - - - - - - - -

\begin{frame}{Market for Lemons - Buyer's Perspective}
  For the buyer: \\ 
  write out the expected utility of buying a car at price $p$:
\end{frame}

% - - - - - - - - - - - - - - - - - - - - - - - - - - - - - - - - - - - - - - -

\begin{frame}{Market for Lemons - Seller's Perspective}
  For the seller: \\ 
  \begin{itemize}
    \item the expected utility of selling a lemon is:
    \vspace{15mm}
    \item the expected utility of selling a good car is:
  \end{itemize}
\end{frame}

% - - - - - - - - - - - - - - - - - - - - - - - - - - - - - - - - - - - - - - -

\begin{frame}{Market for Lemons - Market Clearing}
  When will all buyers and sellers want to trade?
  \vspace{30mm}
\end{frame}

% - - - - - - - - - - - - - - - - - - - - - - - - - - - - - - - - - - - - - - -

\begin{frame}{Market for Lemons - Market Unraveling}
  When will there be a \textit{pooling equilibrium}
  where only one type of car is sold?
  \vspace{30mm} \\
  What type of car is sold in this equilibrium?
\end{frame}


% - - - - - - - - - - - - - - - - - - - - - - - - - - - - - - - - - - - - - - -

\begin{frame}{Market for Lemons - Conclusions}
  \begin{quote}
    \small{
    "Verbal declarations are costless and therefore useless. Anyone can lie about why he is selling the car. 
    One can offer to let the buyer have the car checked. The lemon owner can make the same offer. It’s a bluff. If called, nothing is lost. Besides, such checks are costly. \\
    Reliability reports from the owner’s mechanic are untrustworthy. The clever nonlemon owner might pay for the checkup but let the purchaser choose the inspector. The problem for the owner, then, is to keep the inspection cost down. 
    Guarantees do not work. The seller may move to Cleveland, leaving no forwarding address." 
    }
  \end{quote} 
   \footnotesize{- A. Michael Spence, \textit{Market Signaling: Information Transfer in Hiring and Related Screening Processes}}
\end{frame}

% - - - - - - - - - - - - - - - - - - - - - - - - - - - - - - - - - - - - - - -

\begin{frame}{Market for Lemons - Discussion}
  \begin{itemize}
  \item How well do you think that the used car market in the real world  
  reflects the Market for Lemons model?
  \item What other factors could there be that allow used car sellers 
  to \textit{credibly signal} the quality of their cars?
  \end{itemize}
\end{frame}
