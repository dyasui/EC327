
% - - - - - - - - - - - - - - - - - - - - - - - - - - - - - - - - - - - - - - -
\begin{frame}{What is Asymmetric Info?}
  \begin{itemize}
    \item We already learned about \textit{symmetric} uncertainty
    in the models where \texttt{Nature} makes a play
    that \textit{neither} player can observe.
    \item But sometimes one player will know some things that other do not.
  \end{itemize}
  \begin{block}{Asymmetric Information}
    describes situations in which some players have \alert{private information}
    that is not accessible to other players.
  \end{block} 
\end{frame}

% - - - - - - - - - - - - - - - - - - - - - - - - - - - - - - - - - - - - - - -

\begin{frame}{What is Asymmetric Info?}
  If you are \alert{better informed} than others:
  \begin{itemize}
    \item You might be able to 
    \textit{conceal} or \textit{reveal misleading} information strategically
    in order to manipulate the beliefs of others about you 
    \item You might instead want to \textit{selectively reveal} the truth
    if it helps you.
  \end{itemize}
  If you are \alert{less informed} than other players:
  \begin{itemize}
    \item You might want to \textit{filter out the truth} from lies or misinformation.
    \item You could instead strategically \textit{remain ignorant} 
    in order to claim "credible deniability".
  \end{itemize}
\end{frame}

% - - - - - - - - - - - - - - - - - - - - - - - - - - - - - - - - - - - - - - -

\begin{frame}{Behaviors in Asymmetric Info Games}
  \begin{block}{Cheap Talk}
    I could let people in on my private info by directly talking to them.
    But if they know that I have potential incentives to \textit{lie},
    they might not believe my \alert{\textit{cheap talk}}.
  \end{block}
  \begin{quote}
    Actions Speak Louder Than Words 
  \end{quote}
\end{frame}

% - - - - - - - - - - - - - - - - - - - - - - - - - - - - - - - - - - - - - - -

\begin{frame}{Behaviors in Asymmetric Info Games}
  \begin{block}{Signaling}
    When I know something about myself 
    that would benefit me if \textit{others} knew,
    I might send a \alert{signal} through my actions
  \end{block}
  Examples:
  \begin{itemize}
    \item
    A 4.0 GPA might signal to potential employers that you are hard-working.
    \item
    If you're in the market for a product and you're uncertain of its quality,
    a money-back guarantee might \textit{signal} that it works.
  \end{itemize}
\end{frame}

% - - - - - - - - - - - - - - - - - - - - - - - - - - - - - - - - - - - - - - -

\begin{frame}{Behaviors in Asymmetric Info Games}
  \begin{block}{Screening}
    When I want to know something about \textit{someone else's} private info,
    I might get them to take an action that would \alert{screen} out 
    people of different \textit{types}.
  \end{block}
  Examples:
  \begin{itemize}
    \item 
    An employer might not know if a job candidate is a \textit{lazy}
    or \textit{industrious} type of worker,
    but they could try to screen out the \textit{lazy} ones
    by requiring a portfolio of previous work.
  \end{itemize}
\end{frame}

% - - - - - - - - - - - - - - - - - - - - - - - - - - - - - - - - - - - - - - -

\begin{frame}{Effectiveness of Different Communication Strategies}
  When are different strategies effective in actually revealing private info?
  \begin{itemize}
    \item 
    Sometimes direct communication works when players' interests align.
    But trust might break down when there are incentives to send false messages.
    \item 
    A signal is only effective if not all types take the same action.
    We'll discuss breakdowns in signaling using the ideas of 
    \alert{Separating} vs \alert{Pooling} equilibria
  \end{itemize}
\end{frame}

% - - - - - - - - - - - - - - - - - - - - - - - - - - - - - - - - - - - - - - -

\begin{frame}{Asymmetric Info in Market Games}
  \begin{itemize}
    \item 
    In 201 or 311 you may have learned about the 
    \textbf{perfectly competitive} markets model. \\
    \item 
    One of the assumptions of that model is \alert{perfect information}. \\ 
    \item 
    When this assumption breaks, we might see \alert{Adverse Selection}
    or other types of market failures.
\end{itemize}
\end{frame}
