\begin{frame}{Complete Information}
  \alert{Complete information} is a restrictive assumption
  \begin{itemize}
    \item all players know and agree on rules of the game
    \item all players know their own preferences and everyone else's
    \item common knowledge of how strategies map onto preferences
  \end{itemize}
  Now we'll see how to relax this.
\end{frame}

\begin{frame}{How to Strategize around Uncertainty}
  Strategic behavior may help to deal with uncertainty:
  \begin{itemize}
    \item Risk-averse players might choose to spread external risks out
    \begin{itemize}
      \item Insurance policies, etc.
    \end{itemize} 
    \item Uninformed players might try to learn hidden information from actions
    \begin{itemize}
      \item `weed-out' classes, job applicant screening
    \end{itemize} 
    \item Players with hidden information might try to communicate
    \begin{itemize}
      \item When do actions speak louder than words?
    \end{itemize}
  \end{itemize}
\end{frame}

% - - - - - - - - - - - - - - - - - - - - - - - - - - - - - - - - - - - - - - -

\begin{frame}{What is Asymmetric Info?}
  \begin{itemize}
    \item \textit{symmetric} uncertainty is where everyone is in the dark
    \begin{itemize}
      \item External uncertainty is decided by Nature
      \item No player in the game knows what will happen (but they still have beliefs)
    \end{itemize}
    \item But sometimes one player will know some things that other do not.
  \end{itemize}
  \begin{block}{Asymmetric Information}
    describes situations in which some players have \alert{private information}
    that is not accessible to other players.
  \end{block} 
\end{frame}

% - - - - - - - - - - - - - - - - - - - - - - - - - - - - - - - - - - - - - - -

\begin{frame}{What is Asymmetric Info?}
  If you are \alert{better informed} than others:
  \begin{itemize}
    \item You might be able to 
    \textit{conceal} or \textit{reveal misleading} information strategically
    \item You might instead want to \textit{selectively reveal} the truth
  \end{itemize}
  If you are \alert{less informed} than other players:
  \begin{itemize}
    \item You might want to \textit{filter out the truth} from lies or misinformation.
    \item You could instead strategically \textit{remain ignorant} 
    in order to claim "credible deniability".
  \end{itemize}
\end{frame}

% - - - - - - - - - - - - - - - - - - - - - - - - - - - - - - - - - - - - - - -

\begin{frame}{Behaviors in Asymmetric Info Games}
  \begin{block}{Cheap Talk}
    I could let people in on my private info by directly talking to them.
    But if they know that I have potential incentives to \textit{lie},
    they might not believe my \alert{\textit{cheap talk}}.
  \end{block}
  \begin{quote}
    Actions Speak Louder Than Words 
  \end{quote}
\end{frame}

% - - - - - - - - - - - - - - - - - - - - - - - - - - - - - - - - - - - - - - -

\begin{frame}{Behaviors in Asymmetric Info Games}
  \begin{block}{Signaling}
    When I know something about myself 
    that would benefit me if \textit{others} knew,
    I might send a \alert{signal} through my actions
  \end{block}
  Examples:
  \begin{itemize}
    \item
    A 4.0 GPA might signal to potential employers that you are hard-working.
    \item
    If you're in the market for a product and you're uncertain of its quality,
    a money-back guarantee might \textit{signal} that it works.
  \end{itemize}
\end{frame}

% - - - - - - - - - - - - - - - - - - - - - - - - - - - - - - - - - - - - - - -

\begin{frame}{Behaviors in Asymmetric Info Games}
  \begin{block}{Screening}
    When I want to know something about \textit{someone else's} private info,
    I might get them to take an action that would \alert{screen} out 
    people of different \textit{types}.
  \end{block}
  Examples:
  \begin{itemize}
    \item 
    An employer might not know if a job candidate is a \textit{lazy}
    or \textit{industrious} type of worker,
    but they could try to screen out the \textit{lazy} ones
    by requiring a portfolio of previous work.
  \end{itemize}
\end{frame}

% - - - - - - - - - - - - - - - - - - - - - - - - - - - - - - - - - - - - - - -

\begin{frame}{Effectiveness of Different Communication Strategies}
  When are different strategies effective in actually revealing private info?
  \begin{itemize}
    \item 
    Sometimes direct communication works when players' interests align.
    But trust might break down when there are incentives to send false messages.
    \item 
    A signal is only effective if not all types take the same action.
    We'll discuss breakdowns in signaling using the ideas of 
    \alert{Separating} vs \alert{Pooling} equilibria
  \end{itemize}
\end{frame}

% - - - - - - - - - - - - - - - - - - - - - - - - - - - - - - - - - - - - - - -

\begin{frame}{Asymmetric Info in Market Games}
  \begin{itemize}
    \item 
    In 201 or 311 you may have learned about the 
    \textbf{perfectly competitive} markets model. \\
    \item 
    One of the assumptions of that model is \alert{perfect information}. \\ 
    \item 
    When this assumption breaks, we might see \alert{Adverse Selection}
    or other types of market failures.
\end{itemize}
\end{frame}
