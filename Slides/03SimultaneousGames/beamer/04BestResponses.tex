
\begin{frame}
\frametitle{iClicker Poll}
\begin{table}[h]
	\centering
	\begin{tabular}{cc|c|c|c|c|}
		& \multicolumn{1}{c}{} & \multicolumn{4}{c}{P$_2$}\\
		& \multicolumn{1}{c}{} & \multicolumn{1}{c}{$a$} & \multicolumn{1}{c}{$b$} & \multicolumn{1}{c}{$c$} & \multicolumn{1}{c}{$d$} \\\cline{3-6}
		\multirow{4}*{P$_1$}  & $A$ & 1, 0 & 2, 1 & 3, 1 & 4, 2 \\\cline{3-6}
		& $B$ & 3, 1 & 2, 2 & 2, 0 & 3, 0 \\\cline{3-6}
		& $C$ & 3, 1 & 4, 0 & 1, 0 & 2, 0 \\\cline{3-6}
		& $D$ & 4, 2 & 3, 0 & 2, 1 & 1, 1 \\\cline{3-6}
	\end{tabular}
\end{table}
\begin{itemize}
	\item Suppose that you KNEW, without a doubt, that Player 2 would play $b$. If you were Player 1, what strategy would you choose?
	\begin{enumerate}
		\item A
		\item B
		\item C
		\item D
		\item I would be indifferent between two or more strategies.
	\end{enumerate}
\end{itemize}
\end{frame}

\begin{frame}
\frametitle{iClicker Poll}
\begin{table}[h]
\centering
\begin{tabular}{cc|c|c|c|c|}
	& \multicolumn{1}{c}{} & \multicolumn{4}{c}{P$_2$}\\
	& \multicolumn{1}{c}{} & \multicolumn{1}{c}{$a$} & \multicolumn{1}{c}{$b$} & \multicolumn{1}{c}{$c$} & \multicolumn{1}{c}{$d$} \\\cline{3-6}
	\multirow{4}*{P$_1$}  & $A$ & 1, 0 & 2, 1 & 3, 1 & 4, 2 \\\cline{3-6}
	& $B$ & 3, 1 & 2, 2 & 2, 0 & 3, 0 \\\cline{3-6}
	& $C$ & 3, 1 & 4, 0 & 1, 0 & 2, 0 \\\cline{3-6}
	& $D$ & 4, 2 & 3, 0 & 2, 1 & 1, 1 \\\cline{3-6}
\end{tabular}
\end{table}
\begin{itemize}
\item Suppose that you KNEW, without a doubt, that Player 1 would play $C$. If you were Player 2, what strategy would you choose?
\begin{enumerate}
	\item a
	\item b
	\item c
	\item d
	\item I would be indifferent between two or more strategies.
\end{enumerate}
\end{itemize}
\end{frame}

\begin{frame}
\frametitle{iClicker Poll}
\begin{table}[h]
\centering
\begin{tabular}{cc|c|c|c|c|}
	& \multicolumn{1}{c}{} & \multicolumn{4}{c}{P$_2$}\\
	& \multicolumn{1}{c}{} & \multicolumn{1}{c}{$a$} & \multicolumn{1}{c}{$b$} & \multicolumn{1}{c}{$c$} & \multicolumn{1}{c}{$d$} \\\cline{3-6}
	\multirow{4}*{P$_1$}  & $A$ & 1, 0 & 2, 1 & 3, 1 & 4, 2 \\\cline{3-6}
	& $B$ & 3, 1 & 2, 2 & 2, 0 & 3, 0 \\\cline{3-6}
	& $C$ & 3, 1 & 4, 0 & 1, 0 & 2, 0 \\\cline{3-6}
	& $D$ & 4, 2 & 3, 0 & 2, 1 & 1, 1 \\\cline{3-6}
\end{tabular}
\end{table}
\begin{itemize}
\item Suppose that you KNEW, without a doubt, that Player 2 would play $a$. If you were Player 1, what strategy would you choose?
\begin{enumerate}
	\item A
	\item B
	\item C
	\item D
	\item I would be indifferent between two or more strategies.
\end{enumerate}
\end{itemize}
\end{frame}

\begin{frame}
\frametitle{iClicker Poll}
\begin{table}[h]
\centering
\begin{tabular}{cc|c|c|c|c|}
	& \multicolumn{1}{c}{} & \multicolumn{4}{c}{P$_2$}\\
	& \multicolumn{1}{c}{} & \multicolumn{1}{c}{$a$} & \multicolumn{1}{c}{$b$} & \multicolumn{1}{c}{$c$} & \multicolumn{1}{c}{$d$} \\\cline{3-6}
	\multirow{4}*{P$_1$}  & $A$ & 1, 0 & 2, 1 & 3, 1 & 4, 2 \\\cline{3-6}
	& $B$ & 3, 1 & 2, 2 & 2, 0 & 3, 0 \\\cline{3-6}
	& $C$ & 3, 1 & 4, 0 & 1, 0 & 2, 0 \\\cline{3-6}
	& $D$ & 4, 2 & 3, 0 & 2, 1 & 1, 1 \\\cline{3-6}
\end{tabular}
\end{table}
\begin{itemize}
\item Suppose that you KNEW, without a doubt, that Player 1 would play $D$. If you were Player 2, what strategy would you choose?
\begin{enumerate}
	\item A
	\item B
	\item C
	\item D
	\item I would be indifferent between two or more strategies.
\end{enumerate}
\end{itemize}
\end{frame}

\begin{frame}
\frametitle{Best Responses}
\begin{itemize}
	\item If you know what strategy the other player will choose, then you can easily figure out what your best option (or options) are.
	\item Of course, you don't actually know what the other player will do when you choose your strategy in a game like this---but thinking about the game this way makes it easy to find NEs.
	\item A strategy $s_i$ is a \myul{best response} to another player's strategy $s_{-i}$ if and only if it provides the highest payoff possible when the other player chooses $s_{-i}$.
\end{itemize}
\end{frame}

\begin{frame}
\frametitle{Some Notes on Best Responses}
\begin{itemize}
	\item The phrase ``to another player's strategy" in the definition of a best response is \textbf{important}.
	\item A strategy can only be a best response \textbf{to some strategy of the other player}. There is no such thing as a strategy which is just ``a best response."
	\item When dealing with strategic-form games with game tables, there is always at least one best response to another player's strategy---and there may be multiple, if there is more than one strategy which provides the best payoff.
\end{itemize}
\end{frame}

\begin{frame}
\frametitle{Best Responses in Practice}
\begin{itemize}
	\item It's easy to depict best responses in a game table: we can go through each strategy in the game, and mark each strategy which is a best response to them.
	\item We do this by marking the payoffs---however, it's important to understand that it's not the payoff itself which is a best response--we're just using them as a convenient way to depict where the best responses are.
	\begin{itemize}
		\item In other words: don't try to describe the best response using the payoff. The best response is always the strategy.
	\end{itemize}
\end{itemize}
\end{frame}

\begin{frame}
\frametitle{Example: Best Responses in Game Table}
\begin{table}[h]
	\centering
	\begin{tabular}{cc|c|c|c|c|}
		& \multicolumn{1}{c}{} & \multicolumn{4}{c}{P$_2$}\\
		& \multicolumn{1}{c}{} & \multicolumn{1}{c}{$a$} & \multicolumn{1}{c}{$b$} & \multicolumn{1}{c}{$c$} & \multicolumn{1}{c}{$d$} \\\cline{3-6}
		\multirow{4}*{P$_1$}  & $A$ & 1, 0 & 2, 1 & 3, 1 & 4, 2 \\\cline{3-6}
		& $B$ & 3, 1 & 2, 2 & 2, 0 & 3, 0 \\\cline{3-6}
		& $C$ & 3, 1 & 4, 0 & 1, 0 & 2, 0 \\\cline{3-6}
		& $D$ & 4, 2 & 3, 0 & 2, 1 & 1, 1 \\\cline{3-6}
	\end{tabular}
\end{table}
\begin{itemize}
	\item To find Player 1's best responses, we will go through each column of the table (i.e. each of Player 2's strategies) and look for which of Player 1's strategies gives Player 1 the best payoff in that column.
\end{itemize}
\end{frame}

\begin{frame}
\frametitle{Example: Best Responses in Game Table}
\begin{table}[h]
\centering
\begin{tabular}{cc|c|c|c|c|}
	& \multicolumn{1}{c}{} & \multicolumn{4}{c}{P$_2$}\\
	& \multicolumn{1}{c}{} & \multicolumn{1}{c}{$a$} & \multicolumn{1}{c}{$b$} & \multicolumn{1}{c}{$c$} & \multicolumn{1}{c}{$d$} \\\cline{3-6}
	\multirow{4}*{P$_1$}  & $A$ & 1, 0 & 2, 1 & 3, 1 & 4, 2 \\\cline{3-6}
	& $B$ & 3, 1 & 2, 2 & 2, 0 & 3, 0 \\\cline{3-6}
	& $C$ & 3, 1 & 4, 0 & 1, 0 & 2, 0 \\\cline{3-6}
	& $D$ & 4, 2 & 3, 0 & 2, 1 & 1, 1 \\\cline{3-6}
\end{tabular}
\end{table}
\begin{itemize}
	\item Likewise, to find Player 2's best responses, we will go through each \textbf{row} of the table (each of Player 1's strategies) and look for which of Player 2's strategies gives \textbf{Player 2} the best payoff in that column.
	\begin{itemize}
		\item The bolded parts are important: you have to look at Player 1's payoffs to find Player 1's best responses, and Player 2's payoffs to find Player 2's best responses.
	\end{itemize}
\end{itemize}
\end{frame}

\begin{frame}
\frametitle{Another Example: Best Responses in the Deer Hunt}
\begin{itemize}
	\item Recall the Deer Hunt game, in which two cavemen each decide whether to hunt Deer or Rabbit.
	\item We couldn't find a Nash Equilibrium to this game using elimination methods, because there was nothing to eliminate: no strategies are strictly dominated, and there are no strategies which are non-best-responses.
	\item Here are Igg's best responses...
\end{itemize}
\begin{table}[h]
\centering
\begin{tabular}{cc|c|c|}
	& \multicolumn{1}{c}{} & \multicolumn{2}{c}{Ogg}\\
	& \multicolumn{1}{c}{} & \multicolumn{1}{c}{$Deer$}  & \multicolumn{1}{c}{$Rabbit$} \\\cline{3-4}
	\multirow{2}*{Igg}  & $Deer$ & $2,2$ & $0,1$ \\\cline{3-4}
	& $Rabbit$ & $1,0$ & $1,1$ \\\cline{3-4}
\end{tabular}
\end{table}
\end{frame}

\begin{frame}
\frametitle{Another Example: Best Responses in the Deer Hunt}
\begin{itemize}
	\item And here are Ogg's best responses...
\end{itemize}
\begin{table}[h]
\centering
\begin{tabular}{cc|c|c|}
	& \multicolumn{1}{c}{} & \multicolumn{2}{c}{Ogg}\\
	& \multicolumn{1}{c}{} & \multicolumn{1}{c}{$Deer$}  & \multicolumn{1}{c}{$Rabbit$} \\\cline{3-4}
	\multirow{2}*{Igg}  & $Deer$ & $2,2$ & $0,1$ \\\cline{3-4}
	& $Rabbit$ & $1,0$ & $1,1$ \\\cline{3-4}
\end{tabular}
\end{table}
\end{frame}

\begin{frame}
\frametitle{Another Example: Non-Unique Best Responses}
\begin{itemize}
	\item As mentioned earlier, best responses don't have to be unique.
\end{itemize}
\begin{table}[h]
\centering
\begin{tabular}{cc|c|c|}
	& \multicolumn{1}{c}{} & \multicolumn{2}{c}{Michael}\\
	& \multicolumn{1}{c}{} & \multicolumn{1}{c}{$Swerve$}  & \multicolumn{1}{c}{$Straight$} \\\cline{3-4}
	\multirow{2}*{Eleanor}  & $Swerve$ & $\underline{1}, 1$ & $\underline{1}, 1$ \\\cline{3-4}
	& $Straight$ & $\underline{1}, 1$ & $0, 0$ \\\cline{3-4}
\end{tabular}
\end{table}
\begin{itemize}
	\item Here, for both players, both Straight and Swerve are best responses to Swerve. When the other player chooses Swerve, both strategies provide payoff of 1.
\end{itemize}
\end{frame}

\begin{frame}
\frametitle{iClicker Q1}
\begin{table}[h]
	\centering
	\begin{tabular}{cc|c|c|c|}
		& \multicolumn{1}{c}{} & \multicolumn{3}{c}{P$_2$}\\
		& \multicolumn{1}{c}{} & \multicolumn{1}{c}{$X$} & \multicolumn{1}{c}{$Y$} & \multicolumn{1}{c}{$Z$} \\\cline{3-5}
		\multirow{3}*{P$_1$}  & $A$ & 3, 3 & 2, 2 & 1, 1 \\\cline{3-5}
		& $B$ & 4, 2 & 1, 1 & 2, 2 \\\cline{3-5}
		& $C$ & 1, 1 & 2, 2 & 3, 1  \\\cline{3-5}
	\end{tabular}
\end{table}
\begin{itemize}
	\item In the game shown above, what is Player 1's best response to X?
	\begin{enumerate}
		\item A
		\item B
		\item C
		\item A and B are both best responses.
		\item A and C are both best responses.
	\end{enumerate}
\end{itemize}
\end{frame}

\begin{frame}
\frametitle{iClicker Q2}
\begin{table}[h]
\centering
\begin{tabular}{cc|c|c|c|}
	& \multicolumn{1}{c}{} & \multicolumn{3}{c}{P$_2$}\\
	& \multicolumn{1}{c}{} & \multicolumn{1}{c}{$X$} & \multicolumn{1}{c}{$Y$} & \multicolumn{1}{c}{$Z$} \\\cline{3-5}
	\multirow{3}*{P$_1$}  & $A$ & 3, 3 & 2, 2 & 1, 1 \\\cline{3-5}
	& $B$ & 4, 2 & 1, 1 & 2, 2 \\\cline{3-5}
	& $C$ & 1, 1 & 2, 2 & 3, 1  \\\cline{3-5}
\end{tabular}
\end{table}
\begin{itemize}
\item In the game shown above, what is Player 1's best response to \textbf{Y}?
\begin{enumerate}
	\item A
	\item B
	\item C
	\item A and B are both best responses.
	\item A and C are both best responses.
\end{enumerate}
\end{itemize}
\end{frame}

\begin{frame}
\frametitle{iClicker Q3}
\begin{table}[h]
\centering
\begin{tabular}{cc|c|c|c|}
	& \multicolumn{1}{c}{} & \multicolumn{3}{c}{P$_2$}\\
	& \multicolumn{1}{c}{} & \multicolumn{1}{c}{$X$} & \multicolumn{1}{c}{$Y$} & \multicolumn{1}{c}{$Z$} \\\cline{3-5}
	\multirow{3}*{P$_1$}  & $A$ & 3, 3 & 2, 2 & 1, 1 \\\cline{3-5}
	& $B$ & 4, 2 & 1, 1 & 2, 2 \\\cline{3-5}
	& $C$ & 1, 1 & 2, 2 & 3, 1  \\\cline{3-5}
\end{tabular}
\end{table}
\begin{itemize}
\item In the game shown above, what is \textbf{Player 2's} best response to \textbf{C}?
\begin{enumerate}
	\item X
	\item Y
	\item Z
	\item X and Y are both best responses.
	\item X and Z are both best responses.
\end{enumerate}
\end{itemize}
\end{frame}

\begin{frame}
\frametitle{Nash Equilibrium from Best Responses}
\begin{itemize}
	\item Recall the various definitions of a Nash Equilibrium:
	\begin{itemize}
		\item a strategy profile such that no player can obtain a larger payoff by \textit{unilaterally deviating} (changing only their own strategy).
		\item A strategy profile such that no single player can make themselves better off by changing only their own strategy.
		\item A strategy profile such that, after the game is played, each player is satisfied that they could not have made a better decision.
	\end{itemize}
	\item Another definition we can use now is ``A strategy profile such that each player's strategy is a best responses to the other player's strategy."
	\item ``Playing a best response" is equivalent to ``cannot obtain a larger payoff by unilaterally deviating," or any of the other ways to describe this condition.
\end{itemize}
\end{frame}

\begin{frame}
\frametitle{Nash Equilibrium from Best Responses}
\begin{itemize}
	\item If we find each player's best responses in a game table, and do it using the same table for each player...
\end{itemize}
\begin{table}[h]
\centering
\begin{tabular}{cc|c|c|c|c|}
	& \multicolumn{1}{c}{} & \multicolumn{4}{c}{P$_2$}\\
	& \multicolumn{1}{c}{} & \multicolumn{1}{c}{$a$} & \multicolumn{1}{c}{$b$} & \multicolumn{1}{c}{$c$} & \multicolumn{1}{c}{$d$} \\\cline{3-6}
	\multirow{4}*{P$_1$}  & $A$ & 1, 0 & 2, 1 & \textbf{3}, 1 & \colorbox{blue!25}{\textbf{4}, \textbf{2}} \\\cline{3-6}
	& $B$ & 3, 1 & 2, \textbf{2} & 2, 0 & 3, 0 \\\cline{3-6}
	& $C$ & 3, \textbf{1} & \textbf{4}, 0 & 1, 0 & 2, 0 \\\cline{3-6}
	& $D$ & \colorbox{blue!25}{\textbf{4}, \textbf{2}} & 3, 0 & 2, 1 & 1, 1 \\\cline{3-6}
\end{tabular}
\end{table}
\begin{itemize}
	\item Then any cell of the table in which \textbf{all payoffs are marked to indicate a best response} represents a NE.
	\item Here, the two NEs are (D, a) and (A, d), which we could never have found from elimination.
\end{itemize}
\end{frame}

\begin{frame}
\frametitle{Another Example: Nash Equilibrium in the Deer Hunt}
\begin{itemize}
	\item We did already find the NEs of the Deer Hunt, but we had to go through and check all four strategy profiles. We can do it much faster by just using best responses:
\end{itemize}
\begin{table}[h]
\centering
\begin{tabular}{cc|c|c|}
	& \multicolumn{1}{c}{} & \multicolumn{2}{c}{Ogg}\\
	& \multicolumn{1}{c}{} & \multicolumn{1}{c}{$Deer$}  & \multicolumn{1}{c}{$Rabbit$} \\\cline{3-4}
	\multirow{2}*{Igg}  & $Deer$ & \colorbox{blue!25}{\textbf{2,2}} & 0,1 \\\cline{3-4}
	& $Rabbit$ & 1,0 & \colorbox{blue!25}{\textbf{1,1}} \\\cline{3-4}
\end{tabular}
\end{table}
\end{frame}

\begin{frame}
\frametitle{Another Example: When Elimination Does Nothing}
\begin{itemize}
	\item This game has \textbf{absolutely no strategies} that can be eliminated: none are strictly dominated or non-rationalizable.
	\item We can still find the NEs (of which there are quite a few) using best responses:
\end{itemize}
\begin{table}[h]
	\centering
	\begin{tabular}{cc|c|c|c|c|}
		& \multicolumn{1}{c}{} & \multicolumn{4}{c}{P$_2$}\\
		& \multicolumn{1}{c}{} & \multicolumn{1}{c}{$a$} & \multicolumn{1}{c}{$b$} & \multicolumn{1}{c}{$c$} & \multicolumn{1}{c}{$d$}  \\\cline{3-6}
		\multirow{5}*{P$_1$}  & $A$ & 1, 1 & 2, 2 & 2, 2 & 2, 1 \\\cline{3-6}
		& $B$ & 1, 3 & 1, 3 & 2, 2 & 2, 3 \\\cline{3-6}
		& $C$ & 1, 2 & 2, 4 & 1, 3 & 2, 3 \\\cline{3-6}
		& $D$ & 3, 2 & 2, 3 & 1, 4 & 2, 2 \\\cline{3-6}
	\end{tabular}
\end{table}
\end{frame}

\begin{frame}
\frametitle{iClicker Q4}
\begin{table}[h]
\centering
\begin{tabular}{cc|c|c|c|}
	& \multicolumn{1}{c}{} & \multicolumn{3}{c}{P$_2$}\\
	& \multicolumn{1}{c}{} & \multicolumn{1}{c}{$X$} & \multicolumn{1}{c}{$Y$} & \multicolumn{1}{c}{$Z$} \\\cline{3-5}
	\multirow{3}*{P$_1$}  & $A$ & 3, 3 & 2, 2 & 1, 1 \\\cline{3-5}
	& $B$ & 4, 2 & 1, 1 & 2, 2 \\\cline{3-5}
	& $C$ & 1, 1 & 2, 2 & 3, 1  \\\cline{3-5}
\end{tabular}
\end{table}
\begin{itemize}
\item In the game shown above, which of the following are the Nash Equilibria? (You will need to find best responses for both players.)
\begin{enumerate}
	\item (A, X)
	\item (A, Y)
	\item (B, X)
	\item (C, Z)
	\item More than one of the above.
\end{enumerate}
\end{itemize}
\end{frame}

\begin{frame}
\frametitle{Best Responses and Non-Rationalizability}
\begin{itemize}
	\item Finding best responses first makes it a lot easier to search for non-rationalizable strategies.
	\item Recall that a non-rationalizable strategy can also be called a non-best response: if none of the payoffs of a strategy are marked to indicate that it is a best-response, it is non-rationalizable. In other words:\begin{itemize}
		\item Any row in which none of Player 1's payoffs are marked to indicate a best response, is non-rationalizable.
		\item Any column in which none of Player 2's payoffs are marked to indicate a best response, is non-rationalizable.
	\end{itemize}
\end{itemize}
\end{frame}

\begin{frame}
\frametitle{Example: Non-Rationalizability from Best Responses}
\begin{itemize}
\item Finding best responses first makes it a lot easier to search for non-rationalizable strategies.
\item Recall that a non-rationalizable strategy can also be called a non-best response: if none of the payoffs of a strategy are marked to indicate that it is a best-response, it is non-rationalizable. In other words:\begin{itemize}
	\item Any row in which none of Player 1's payoffs are marked to indicate a best response, is non-rationalizable.
	\item Any column in which none of Player 2's payoffs are marked to indicate a best response, is non-rationalizable.
\end{itemize}
\end{itemize}
\end{frame}

\begin{frame}
\frametitle{Example: Non-Rationalizability from Best Responses}
\begin{table}[h]
	\centering
	\begin{tabular}{cc|c|c|c|c|}
		& \multicolumn{1}{c}{} & \multicolumn{4}{c}{P$_2$}\\
		& \multicolumn{1}{c}{} & \multicolumn{1}{c}{$a$} & \multicolumn{1}{c}{$b$} & \multicolumn{1}{c}{$c$} & \multicolumn{1}{c}{$d$} \\\cline{3-6}
		\multirow{4}*{P$_1$}  & $A$ & 1, 0 & 2, 1 & \textbf{3}, 1 & \textbf{4}, \textbf{2} \\\cline{3-6}
		& $B$ & 3, 1 & 2, \textbf{2} & 2, 0 & 3, 0 \\\cline{3-6}
		& $C$ & 3, \textbf{1} & \textbf{4}, 0 & 1, 0 & 2, 0 \\\cline{3-6}
		& $D$ & \textbf{4}, \textbf{2} & 3, 0 & 2, 1 & 1, 1 \\\cline{3-6}
	\end{tabular}
\end{table}
\begin{itemize}
	\item We found these best responses earlier (as well as the NEs of this game): note that in the row for Player 1's strategy B, none of Player 1's strategies are marked.
	\item Likewise, none of Player 2's payoffs are marked in column c. B and c are non-rationalizable strategies.
\end{itemize}
\end{frame}

\begin{frame}
\frametitle{Example: Non-Rationalizability from Best Responses}
\begin{table}[h]
	\centering
	\begin{tabular}{cc|c|c|c|c|}
		& \multicolumn{1}{c}{} & \multicolumn{4}{c}{P$_2$}\\
		& \multicolumn{1}{c}{} & \multicolumn{1}{c}{$a$} & \multicolumn{1}{c}{$b$} & \multicolumn{1}{c}{$c$} & \multicolumn{1}{c}{$d$} \\\cline{3-6}
		\multirow{4}*{P$_1$}  & $A$ & 1, 0 & 2, 1 & \textbf{3}, 1 & \textbf{4}, \textbf{2} \\\cline{3-6}
		& $B$ & 3, 1 & 2, \textbf{2} & 2, 0 & 3, 0 \\\cline{3-6}
		& $C$ & 3, \textbf{1} & \textbf{4}, 0 & 1, 0 & 2, 0 \\\cline{3-6}
		& $D$ & \textbf{4}, \textbf{2} & 3, 0 & 2, 1 & 1, 1 \\\cline{3-6}
	\end{tabular}
\end{table}
\begin{itemize}
	\item Performing IENBR reveals additional non-rationalizable strategies:
	\begin{itemize}
		\item Once we eliminate B, b becomes non-rationalizable.
		\item Once we eliminate b, C becomes non-rationalizable.
	\end{itemize}
\end{itemize}
\end{frame}



\begin{frame}
\frametitle{Classifying NEs}
\begin{itemize}
	\item Nash Equilibria may be either \underline{strict} or \underline{weak}.
	\item A Nash Equilibrium is strict if and only if each player would receive a \textbf{smaller} payoff by changing their own strategy.
	\item If a Nash Equilibrium is not strict---meaning that at least one player could change their own strategy and receive an equal (but not larger) payoff---it is weak.
\end{itemize}
\end{frame}

\begin{frame}
\frametitle{Intuition on Strict vs. Weak Equilibria}
\begin{itemize}
\item In any Nash Equilibrium, no player has a reason to change their own strategy---they cannot get a higher payoff this way.
\item Strict Nash Equilibria go a little further: not only does no player have a reason to change their own strategy, they also have a reason \textbf{not} to, because any other strategy would provide them a worse payoff.
\item If a Nash Equilibrium is weak, it means that some player could change their strategy, and get exactly the same payoff they already were. They have no reason to do this, but also no reason \textbf{not} to.
\item We can also say that a strict Nash Equilibrium is one where each player is playing a strategy which is a \textbf{unique} best response to the strategies chosen by other players.
\end{itemize}
\end{frame}

\begin{frame}
\frametitle{Deer Hunt: Strict Nash Equilibria}
\begin{table}[h]
\centering
\begin{tabular}{cc|c|c|}
& \multicolumn{1}{c}{} & \multicolumn{2}{c}{Ogg}\\
& \multicolumn{1}{c}{} & \multicolumn{1}{c}{$Deer$}  & \multicolumn{1}{c}{$Rabbit$} \\\cline{3-4}
\multirow{2}*{Igg}  & $Deer$ & $\underline{2},\underline{2}$ & $0,1$ \\\cline{3-4}
& $Rabbit$ & $1,0$ & $\underline{1},\underline{1}$ \\\cline{3-4}
\end{tabular}
\end{table}
\begin{itemize}
\item Here, note that at each Nash Equilibrium, each player has no other strategy providing the same payoff. This is a strict Nash Equilibrium.
\end{itemize}
\end{frame}

\begin{frame}
\frametitle{Not Quite the Trolley Problem: Weak Nash Equilibria}
\begin{table}[h]
\centering
\begin{tabular}{cc|c|c|}
& \multicolumn{1}{c}{} & \multicolumn{2}{c}{Michael}\\
& \multicolumn{1}{c}{} & \multicolumn{1}{c}{$Swerve$}  & \multicolumn{1}{c}{$Straight$} \\\cline{3-4}
\multirow{2}*{Eleanor}  & $Swerve$ & $\underline{1}, \underline{1}$ & $\underline{1}, \underline{1}$ \\\cline{3-4}
& $Straight$ & $\underline{1}, \underline{1}$ & $0, 0$ \\\cline{3-4}
\end{tabular}
\end{table}
\begin{itemize}
\item However, in this game, each Nash Equilibrium features at least one player who could still get the same payoff if they change their strategy.
\end{itemize}
\end{frame}

\begin{frame}
\frametitle{Classifying Games Based on NEs}
\begin{itemize}
	\item Now that we've talked about several ways to find a game's NEs, we can start to talk about classifying games using them.
\end{itemize}
\end{frame}

\begin{frame}
\frametitle{Prisoners' Dilemmas}
\begin{table}[h]
	\centering
	\begin{tabular}{cc|c|c|}
		& \multicolumn{1}{c}{} & \multicolumn{2}{c}{Luca}\\
		& \multicolumn{1}{c}{} & \multicolumn{1}{c}{$Testify$}  & \multicolumn{1}{c}{$Keep~Quiet$} \\\cline{3-4}
		\multirow{2}*{Guido}  & $Testify$ & $-10,-10$ & $0,-20$ \\\cline{3-4}
		& $Keep~Quiet$ & $-20,0$ & $-1,-1$ \\\cline{3-4}
	\end{tabular}
\end{table}
\begin{itemize}
	\item So far, I've only used the payoffs above to describe the Prisoner's Dilemma.
	\item But we can change those payoffs---we can also change the story behind the game, and the names of the players and strategies, and it would still count as a Prisoner's Dilemma.
	\item In general, a Prisoner's Dilemma is any game in which:
	\begin{itemize}
		\item The players have the same strategies, A and B.
		\item A strictly dominates B, making (A, A) the only NE.
		\item But (B, B) is better for both players than (A, A).
	\end{itemize}
\end{itemize}
\end{frame}

\begin{frame}
\frametitle{Other Representations of the Prisoners' Dilemmas}
\begin{table}[h]
	\centering
	\begin{tabular}{cc|c|c|}
		& \multicolumn{1}{c}{} & \multicolumn{2}{c}{Luca}\\
		& \multicolumn{1}{c}{} & \multicolumn{1}{c}{$Testify$}  & \multicolumn{1}{c}{$Keep~Quiet$} \\\cline{3-4}
		\multirow{2}*{Guido}  & $Testify$ & $1, 1$ & $3, 0$ \\\cline{3-4}
		& $Keep~Quiet$ & $0, 3$ & $2, 2$ \\\cline{3-4}
	\end{tabular}
\end{table}
\begin{table}[h]
	\centering
	\begin{tabular}{cc|c|c|}
		& \multicolumn{1}{c}{} & \multicolumn{2}{c}{$P_2$}\\
		& \multicolumn{1}{c}{} & \multicolumn{1}{c}{$A$}  & \multicolumn{1}{c}{$B$} \\\cline{3-4}
		\multirow{2}*{$P_1$}  & $A$ & $0, 4$ & $2, 3$ \\\cline{3-4}
		& $B$ & $-1, 6$ & $1, 5$ \\\cline{3-4}
	\end{tabular}
\end{table}
\end{frame}

\begin{frame}
\frametitle{Coordination Games}
\begin{itemize}
	\item A \myul{coordination game} is a game in which the players all have the same strategy sets, and the NEs are all of the strategy profiles where the players choose the same strategy.
	\item The Deer Hunt is a coordination game.
\end{itemize}
\begin{table}[h]
\centering
\begin{tabular}{cc|c|c|}
	& \multicolumn{1}{c}{} & \multicolumn{2}{c}{Ogg}\\
	& \multicolumn{1}{c}{} & \multicolumn{1}{c}{$Deer$}  & \multicolumn{1}{c}{$Rabbit$} \\\cline{3-4}
	\multirow{2}*{Igg}  & $Deer$ & \colorbox{blue!25}{\textbf{2,2}} & 0,1 \\\cline{3-4}
	& $Rabbit$ & 1,0 & \colorbox{blue!25}{\textbf{1,1}} \\\cline{3-4}
\end{tabular}
\end{table}
\end{frame}

\begin{frame}
\frametitle{Other Coordination Games}
\begin{table}[h]
	\caption*{Bach or Stravinsky?}
	\centering
	\begin{tabular}{cc|c|c|}
		& \multicolumn{1}{c}{} & \multicolumn{2}{c}{Stravinsky Fan}\\
		& \multicolumn{1}{c}{} & \multicolumn{1}{c}{Bach}  & \multicolumn{1}{c}{Strav.} \\\cline{3-4}
		\multirow{2}*{Bach Fan}  & Bach & \colorbox{blue!25}{\textbf{3, 2}} & 0,0 \\\cline{3-4}
		& Strav. & 0,0 & \colorbox{blue!25}{\textbf{2, 3}} \\\cline{3-4}
	\end{tabular}
\end{table}
\begin{table}[h]
	\centering
	\begin{tabular}{cc|c|c|c|}
		& \multicolumn{1}{c}{} & \multicolumn{3}{c}{$P_2$}\\
		& \multicolumn{1}{c}{} & \multicolumn{1}{c}{A}  & \multicolumn{1}{c}{B}  & \multicolumn{1}{c}{C} \\\cline{3-5}
		\multirow{2}*{$P_1$}  & A & \colorbox{blue!25}{\textbf{1, 1}} & 1, 0 & 3, 0 \\\cline{3-5}
		& B & 0, 1 & \colorbox{blue!25}{\textbf{2, 2}} & 2, 1 \\\cline{3-5}
		& B & 0, 3 & 1, 2 & \colorbox{blue!25}{\textbf{4, 4}} \\\cline{3-5}
	\end{tabular}
\end{table}
\end{frame}


\begin{frame}
\frametitle{Anti-Coordination Games}
\begin{itemize}
	\item An \myul{anti-coordination game} is a game in which the players all have the same strategy sets, but the NEs are all of the strategy profiles where the players choose \textbf{different} strategies.
\end{itemize}
\begin{table}[h]
	\caption*{Chicken}
	\centering
	\begin{tabular}{cc|c|c|}
		& \multicolumn{1}{c}{} & \multicolumn{2}{c}{Buzz}\\
		& \multicolumn{1}{c}{} & \multicolumn{1}{c}{Straight}  & \multicolumn{1}{c}{Swerve} \\\cline{3-4}
		\multirow{2}*{Jim}  & Straight & -10, -10 & \colorbox{blue!25}{\textbf{5, -1}} \\\cline{3-4}
		& Swerve & \colorbox{blue!25}{\textbf{-1, 5}} & 0, 0 \\\cline{3-4}
	\end{tabular}
\end{table}
\end{frame}

\begin{frame}
\frametitle{Symmetric Games}
\begin{itemize}
	\item A \myul{symmetric game} is a game which is \textbf{indifferent to an exchange of players}---in other words, a game where the players are interchangeable.
	\item Consider the Prisoner's Dilemma: if we swap the players' names, and their positions in the game table, and the order of their payoffs, we get the same game that we started with.
	\item A two-player game with a game table is symmetric if:
	\begin{itemize}
		\item The players have the same strategy sets.
		\item In the on-diagonal cells of the game table, the players receive equal payoffs.
		\item In the ``mirrored" off-diagonal cells of the game table, the players' payoffs are reversed.
	\end{itemize}
	\item Of the games we've just looked at, the Deer Hunt, the 3x3 coordination game, and Chicken are symmetric games. Bach and Stravinsky is not.
\end{itemize}
\end{frame}

