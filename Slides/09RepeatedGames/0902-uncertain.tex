

\begin{frame}{Back to the General Form Prisoners' Dilemma}
  \begin{table}[!h]
    \centering
    \begin{tabular}{*{4}{c|}}
      \multicolumn{2}{c}{} & \multicolumn{2}{c}{Column} \\ \cline{3-4}
      \multicolumn{1}{c}{} &         & Defect  & Cooperate \\ \cline{2-4}
      \multirow{2}*{Row} &    Defect & $D$,$D$ & $H$,$L$   \\ \cline{2-4}
                         & Cooperate & $L$,$H$ & $C$,$C$   \\ \cline{2-4} 
    \end{tabular} 
  \end{table} 
  Now let's suppose that this game is repeated for an \textbf{infinite number of stages}.
\end{frame}

\begin{frame}{Extending Plays to Infinity}
  Suppose the game is in the `good' equilibrium where all players always play \textit{Cooperate}.
  \begin{itemize}
    \item What is \textbf{present value} from this equilibrium? 
    \begin{align*}
      pv\left(\{ C \}_{t=1}^{\infty}\right) & = C + \delta C + \delta^2 C + \delta^3 C + ... \\ 
         & = C \sum_{t=1}^{\infty} \delta^{t-1} \\
         & = C \frac{1}{1-\delta}
    \end{align*}
  \end{itemize}
\end{frame}

\begin{frame}{Extending Plays to Infinity}
   Let's extend the \textit{Punisher} strategy we had from the two-stage game 
   into the \textit{Grim Trigger strategy} of the general infinite horizon game:
  \begin{align*}
    \begin{cases}
      \text{In stage } 1 & : \textit{Cooperate} \\ 
      \text{In stage } t \geq 2 & : 
      \begin{cases}
        \textit{Cooperate} \text{  if only cooperation has happened so far} \\ 
        \textit{Defect} \text{  if anyone has \textit{ever} Defected in the past}
      \end{cases}
    \end{cases}
  \end{align*}
\end{frame}

\begin{frame}{Grim Trigger SPNE in Repeated PD}
  Is both players playing \textit{Grim Trigger} stable?
  \begin{itemize}
    \item Does a player have an incentive to \textit{Defect} against \textit{Grim Trigger}:
    \vspace{-5mm}
    \begin{align*}
      pv(Always~Coop) & \geq pv(Defect~once) \\
      C + \delta C + \delta^2 C + ...      & \geq H + \delta D + \delta^2 D + ...\\
      C + C \sum_{t=2}^{\infty} \delta^{t} & \geq H + D \sum_{t=2}^{\infty} \delta^{t} \\ 
      C + C \delta \sum_{t=2}^{\infty} \delta^{t-1} & \geq H + D \delta \sum_{t=2}^{\infty} \delta^{t-1} \\ 
      C + \frac{\delta C}{1 - \delta} & \geq H + \frac{\delta D}{1 - \delta} \\
      \delta & \geq \frac{H - C}{H - D}
    \end{align*}
  \end{itemize}
\end{frame}

\begin{frame}{Grim Trigger SPNE in Repeated PD}
  How do we interpret this statement:
  $$\text{Cooperation is stable when } \delta \geq \frac{H - C}{H - D}  $$ 
  \begin{itemize}
    \item Recall that the definition of the Prisoners' Dilemma was that $H > C > D > L$
    \item So this means $\frac{H - C}{H - D}$ is positive and less than 1 
    \item As the $H - C$, the relative benefit of defecting increases,
    it gets harder to sustain cooperation
    \item It also gets harder to sustain cooperation as the relative penalty of defecting, $H - D$, shrinks 
  \end{itemize}
\end{frame}

\begin{frame}{Other Strategies in Repeated Games}
  So far we've only looked at one example of a type of strategy in repeated game, 
  \textit{Grim Trigger}. 
  \begin{itemize}
    \item \underline{Can you think of some others?}
    \begin{itemize}
      \item Recall that a complete strategy for a repeated game needs: 
      \begin{itemize}
        \item An initial move at $t=1$
        \item A plan of action for \textit{every} possible history in \textit{every} later stage $t\geq2$
      \end{itemize}
      \item Ideally you would be able to tell a computer how to implement your strategy
    \end{itemize}
  \end{itemize}
\end{frame}

\begin{frame}{Other Strategies in Repeated Games}
  Telling a computer how to implement strategies is exactly what Robert Axelrod did in a famous tournament in 1980.
  \begin{itemize}
    \item He invited people to submit their programs which would play 200 rounds of the prisoners' dilemma against each other 
    \item The winning program was the one which had the highest total score after playing 200 rounds against all other programs
    \item What types of strategies do you think would succeed?
  \end{itemize}
\end{frame}

\begin{frame}{An Unexpected Winner}
  The winning program was named \texttt{TIT FOR TAT} \\ 
  Surprisingly, it was fairly simple:
  \begin{align*}
    \begin{cases}
      \text{In stage } 1 & : \textit{Cooperate} \\ 
      \text{In stage } t \geq 2 & : 
      \begin{cases}
        \text{repeat what the other player did in } t-1 \\
      \end{cases}
    \end{cases}
  \end{align*}
\end{frame}

\begin{frame}{Tit-for-Tat}
  Like \textit{Grim Trigger}, \textit{Tit-for-Tat} can punish other players for defecting. 
  \begin{itemize}
    \item If a player plays \textit{Defect}, it will copy them with \textit{Defect} next round 
  \end{itemize}
  But unlike \textit{Grim Trigger} it has a short memory; or is very forgiving
  \begin{itemize}
    \item If the player who defected goes back to playing cooperatively, 
    \textit{Tit-for-Tat} will go back to cooperating too
  \end{itemize}
\end{frame}

\begin{frame}{Axelrod's Tournament}
  If you want to learn more: 
  \begin{itemize}
    \item Read the original paper: 

    {
    \footnotesize
    Axelrod, Robert; Hamilton, William D. (27 March 1981), "The Evolution of Cooperation" (PDF), \textit{Science}, 211 (4489): 1390–96
    }

    \item The 1984 Book \textit{The Evolution of Cooperation}, Basic Books

    \item Run the tournament yourself in python!
    \url{https://github.com/Axelrod-Python/Axelrod}
    
    \item Play this fun and short web game! 
    \url{https://ncase.me/trust/}

  \end{itemize}
\end{frame}
